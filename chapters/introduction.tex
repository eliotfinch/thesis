% Chapter
\chapter[Introduction]{Introduction\\
}
\label{ch:introduction}

\prettyref{ch:introduction} is a review Chapter, and no original work is presented.

\section{Linearized gravity\label{sec:LinGra}}

Gravitational waves can be thought of as travelling waves of space-time perturbations.
They are a straightforward consequence of the existence a speed limit for the propagation of physical influences, when their geometric interpretation is contextualized in \gr. 
Starting from the metric $g_{\mu\nu}$, it must be a solution of Einstein's fields equation \cite{Einstein:1913:EVRb}

\begin{equation}
	G_{\mu\nu}=R_{\mu\nu}-\frac{1}{2}g_{\mu\nu}R=\frac{8\pi G}{c^{4}}T_{\mu\nu}\label{eq:Einstein}
\end{equation}

where Ricci tensor $R_{\mu\nu}$ and scalar curvature $R$ involve up to second order derivatives of $g_{\mu\nu}$.
The energy-momentum tensor $T_{\mu\nu}$ is associated to matter and radiation distribution of the system. The non-linear nature of such equations makes it challenging to find exact solutions.
However, we are interested in a perturbative solution of \eqref{eq:Einstein}, with respect to the one in absence of matter and radiation (i.e. $T_{\mu\nu}=0$), the flat space-time metric $\eta_{\mu\nu}$
\begin{align}
	g_{\mu\nu} & =\eta_{\mu\nu}+h_{\mu\nu} \label{eq:hdefinition}\\
	\left|h_{\mu\nu}\right| & \ll1
\end{align}
As a consequence, \gr invariance under diffeomorphisms $F$
\begin{equation}
x^{\mu}\rightarrow x^{\prime\mu}=F^{\mu}\left(x\right)
\end{equation}
is restricted to an appropriate set of reference frames, everyone exhibiting \emph{small} $h_{\mu\nu}$. 
The result is the Poincarè group, with the addition of \emph{small} diffeomorphisms

\begin{equation}
	x^{\mu}\rightarrow x^{\prime\mu}=x^{\mu}+\xi^{\mu}\left(x\right) \label{eq:linEE_invariance}
\end{equation}

The resulting classical field theory degree of freedom is $h_{\mu\nu}$, and its gauge invariance is
\begin{equation}
	h_{\mu\nu}^{\prime}=h_{\mu\nu}-\left(\partial_{\nu}\xi_{\mu}+\partial_{\mu}\xi_{\nu}\right)\label{eq:h_gauge_invariance}
\end{equation}
The differential operator $G_{\mu\nu}$ is linearized through the expansion for $g_{\mu\nu}$, and \eqref{eq:Einstein} becomes a gauge invariant equations of motion for $h_{\mu\nu}$
\begin{align}
	\square\bar{h}_{\mu\nu}+\eta_{\mu\nu}\partial^{\rho}\partial^{\sigma}\bar{h}_{\rho\sigma}-\partial^{\rho}\partial_{\nu}\bar{h}_{\mu\rho}-\partial^{\rho}\partial_{\mu}\bar{h}_{\nu\rho} & =-\frac{16\pi G}{c^{4}}T_{\mu\nu}\label{eq:linEE_gauge_inv}\\
	\bar{h}_{\mu\nu} & =h_{\mu\nu}-\frac{1}{2}h_{\alpha}^{\alpha} 
\end{align}
In the so-called \emph{Lorenz gauge}
\begin{align}
	\partial^{\mu}\bar{h}_{\mu\nu} & =0\label{eq:Lorenz-gauge}
\end{align}
the three terms on the \lhs in \eqref{eq:linEE_gauge_inv} vanish, and $\square\bar{h}_{\mu\nu}$ satisfies a 4-dimensional tensor wave equation
\begin{equation}
	\square\bar{h}_{\mu\nu}=-\frac{16\pi G}{c^{4}}T_{\mu\nu}\label{eq:linear-EE}
\end{equation}

By separating the background metric from the freely propagating waves, we recover the energy-momentum tensor conservation, as a gauge-consistency condition 
\begin{equation}
	\partial^{\mu}T_{\mu\nu}=-\frac{c^{4}}{16\pi G}\partial^{\alpha}\partial_{\alpha}\partial^{\mu}\bar{h}_{\mu\nu}=0\label{eq:conservedTmunu}
\end{equation}
By contrast, in full \gr we have a non-conserved energy-momentum
tensor, as shown with the introduction of covariant derivatives 
\begin{equation}
	 D^{\mu}T_{\mu\nu} =\partial^{\mu}T_{\mu\nu}-\Gamma_{\mu\nu}^{\lambda}T_{\lambda}^{\mu}+\Gamma_{\mu\lambda}^{\mu}T_{\nu}^{\lambda}=0
\end{equation}
This is because in non-linear regime matter and radiation exchange energy and momentum with the gravitational field, too.

In summary the linearization in~\eqref{eq:hdefinition} ---and subsequent conservation in~\eqref{eq:conservedTmunu}--- prescribes that \gw sources interact and evolve in a reference spacetime $\eta_{\mu\nu}$ through a well-defined and conserved energy-momentum tensor.
It is a known result in literature (see, e.g.,~\cite{maggiore2008gravitational} or~\cite{2017grav.book.....M} for extensive discussions on the topic) that the background metric does not have to be necessarily flat. Analogue results to the above hold in the presence of additional large-scale low-frequency background gravitational fields, too.
They effectively decouple in the linearized theory, and act purely as a background metric which $h_{\mu\nu}$ propagates through.
This is an important feature of this framework, since both cosmological expansion and gravitational-wave lensing (relevant mechanisms for the following sections) can be described as such.

Far from the emitting sources, test masses are affected by the metric perturbation
$g_{\mu\nu}=\eta_{\mu\nu}+h_{\mu\nu}$ satisfying
\begin{equation}
\square\bar{h}_{\mu\nu}=0
\end{equation}
whose solution are free waves propagating at light speed. They originate as prescribed by the integral of the inhomogeneous \eqref{eq:linear-EE} over the source volume $\mathcal{V}$
\begin{equation}
\bar{h}_{\mu\nu}\left(\boldsymbol{x},t\right)=\frac{4G}{c^{4}}\int_{\mathcal{V}}\frac{1}{\left|\boldsymbol{x}-\boldsymbol{x}^{\prime}\right|}T_{\mu\nu}\left(\boldsymbol{x}^{\prime},t-\frac{\left|\boldsymbol{x}-\boldsymbol{x}^{\prime}\right|}{c}\right)d^{3}x^{\prime}
\end{equation}
in that values on the domain boundary $\partial\mathcal{V}$ fix the propagating fluctuations.
Since Lorenz gauge in~\eqref{eq:Lorenz-gauge} is only a partial gauge fixing, it is usually convenient to remove the remaining degrees of freedom.
Under a small diffeomorphism~\eqref{eq:linEE_invariance}, the tensor $h_{\mu\nu}$ transforms as follows 
\begin{align}
	\bar{h}_{\mu\nu}\rightarrow\bar{h}_{\mu\nu}^{\prime} & =\bar{h}_{\mu\nu}-(\partial_{\nu}\xi_{\mu}+\partial_{\mu}\xi_{\nu}-\eta_{\mu\nu}\partial_{\rho}\xi^{\rho})\nonumber \\
	& \equiv\bar{h}_{\mu\nu}-\mathcal{D}_{\mu\nu\rho}\xi^{\rho}\label{eq:hmunu_manipulation}
\end{align}

and we can use the four independent arbitrary fields $\xi^{\mu}$ to rearrange the physical content of a \gw into the $h_{\mu\nu}$ components, by means of~\eqref{eq:hmunu_manipulation}. 
The most suitable for our purposes is the \ttg gauge. 
They are implicitly defined by a set of equations for the resulting $h^{TT}$ tensor
$h_{\mu\nu}^{TT}$
\begin{equation}
h_{0\mu}^{TT}=0\quad\left(h^{TT}\right)_{i}^{i}=0\quad\partial^{j}h_{ij}^{TT}=0
\end{equation}

However, this set of equation admit solution only in vacuum. If $\Box\bar{h}_{\mu\nu}\neq0$ (i.e. in the presence of matter or radiation), Lorenz gauge imposes
\begin{eqnarray}
	\partial^{\mu}\bar{h}_{\mu\nu}^{\prime}=0 & \Rightarrow & \Box\xi_{\mu}=0\\
	& \Rightarrow & \Box\mathcal{D}_{\mu\nu\rho}\xi^{\rho}=0
\end{eqnarray}
So we cannot set to zero any further component of $\bar{h}_{\mu\nu}$ without falling into contradiction
\begin{eqnarray}
	\Box h_{\mu\nu}=-\frac{16\pi G}{c^{4}}T_{\mu\nu} & \neq & 0=\Box\mathcal{D}_{\mu\nu\rho}\xi^{\rho}\\
	 h_{\mu\nu} & \neq & \Xi_{\mu\nu}
\end{eqnarray}
Being traceless the trace-removal is redundant (i.e. $\overline{h}_{\mu\nu}^{TT}=h_{\mu\nu}^{TT}$), and the linearized vacuum equation reads
\begin{equation}
	\square h_{ij}^{TT}=0\label{eq:linearEE_TTgauge}
\end{equation}
and ---being $h_{ij}^{TT}$ also symmetric--- the most general solution
can be cast in the form of \emph{tensor plane--waves} with wavevector
$k^{\mu}=\left(\frac{2\pi f}{c},\boldsymbol{k}\right)$
\begin{align}
	h_{ij}\left(x,k\right) & =\sum_{A=+,\times}h_{A}(k)\exp(ik^{\mu}x_{\mu})(e_{A})_{ij}\label{eq:vacuumsolution}
\end{align}
with $\boldsymbol{e}_{A=1,2}$ corresponding to a basis for \ttg
$3$--tensors $e_{ij}$ orthogonal to the propagation direction. This is again imposed by Lorentz gauge condition $k^{i}e_{ij}(k)=0$. 
In the particular case of a $3$--vector $\boldsymbol{k}$ in the $z$--direction, the
two required tensors could be chosen as
\begin{equation}
\boldsymbol{e}_{+}\equiv\boldsymbol{e}_{1}=\left(\begin{array}{ccc}
	1 & 0 & 0\\
	0 & -1 & 0\\
	0 & 0 & 0
\end{array}\right)\qquad\boldsymbol{e}_{\times}\equiv\boldsymbol{e}_{2}=\left(\begin{array}{ccc}
	0 & 1 & 0\\
	1 & 0 & 0\\
	0 & 0 & 0
\end{array}\right)
\end{equation}

\begin{figure}[h]
	\begin{centering}
		\includegraphics[width=.8\columnwidth]{./introduction/polarization.pdf}
		\par\end{centering}
	\caption[Polarization tensor basis of a propagating gravitational wave]{Polarization tensors \emph{plus} $\boldsymbol{e}_{+}\left(\hat{z}\right)$
		(left) and \emph{cross} $\boldsymbol{e}_{\times}(\hat{z})$ (right),
		depicted via the \emph{linear maps} $\dot{\boldsymbol{r}}=\boldsymbol{e}_{A}\boldsymbol{r}$
		along the $z-$axis.}
\end{figure}

Therefore a \gw propagating along a direction $\hat{\boldsymbol{n}}$ can be decomposed in a superposition of plane modes
\begin{equation}
		h_{ij}^{TT}\left(t,\boldsymbol{x}\right) =\sum_{A=+,\times}e^{A}_{ij}(\hat{\boldsymbol{n}}) \int_{-\infty}^{+\infty} df \tilde{h}_{A}(f)\exp(-2\pi\imath f(t-\hat{\boldsymbol{n}}\cdot\boldsymbol{x}/c))\label{eq:planewaves}
\end{equation} 

An important distinction is necessary here. If a detector is most sensitive to \acp{gw} with wavelength much bigger then the typical detector size, the retarded time is uniform over it and the term $|\boldsymbol{x}|f/c$ is negligible. Such an approximation holds for ground-based detectors ($|\boldsymbol{x}|\ll10^5\text{--}10^7\si{\meter}$), while it's not satisfied for space-based detectors across their whole sensitivity band ($|\boldsymbol{x}|\ll10^9\text{--}10^{12}\si{\meter}$). Proposed satellites configuration are expected to be as large as  $2.5\times10^8\si{\meter}$. 

When the long-wavelength approximation holds, it is possible to introduce strain scalar timeseries $h_+,h_\times$, defined by the inverse Fourier transforms
\begin{equation}
	h_A(t)=\int_{-\infty}^{+\infty} df \tilde{h}_A(f)\exp(-2\pi\imath ft)
\end{equation}
through which~\eqref{eq:vacuumsolution} becomes

\begin{equation}
	h_{ij}^{TT}(t) =\sum_{A=+,\times}e^{A}_{ij}(\hat{\boldsymbol{n}})h_A(t)
\end{equation} 