% Chapter 1

\chapter{Introduction to Black-hole Ringdown} % Main chapter title

\label{Chapter1} % For referencing the chapter elsewhere, use \ref{Chapter1} 

\section{Ringdown}

A high-level overview of black holes, ringdown, and the usefulness of the ringdown. 
This will motivate the three main chapters of the thesis. 
Specifically, this will aim to introduce
\begin{itemize}
	\item Binary black-hole mergers
	\item Perturbed black holes and quasinormal modes (the basic idea, and a brief history of results)
	\item The no-hair theorem and tests of it
\end{itemize}

% From paper 1
% ------------

The gravitational wave (GW) observatories LIGO \cite{LIGOScientific:2014pky} and Virgo \cite{VIRGO:2014yos} have now observed dozens of GW events \cite{LIGOScientific:2018mvr, LIGOScientific:2020ibl}, mostly from binary black hole (BBH) mergers. 
Particularly prominent in the GW signals of the higher-mass systems are the final few wave cycles, known as the \emph{ringdown}, emitted as the system settles into its final state: a Kerr black hole (BH). 
The ringdown signal contains a superposition of oscillatory modes, the frequency spectrum of which is characteristic of the remnant BH.

The characteristic oscillations of the remnant BH are called \emph{quasinormal modes} (QNMs), so-called because, unlike normal modes, they decay over time.
The QNM frequencies are complex, $\omega = 2\pi f - {i}/\tau$, with the real part $f$ giving the oscillation frequency and the reciprocal of the imaginary part $\tau$ giving the damping time. 
The QNM frequencies can be calculated within the framework of linearised gravity, treating the gravitational field in the vicinity of the remnant as a small (linear) perturbation of the Kerr metric \cite{Berti:2009kk}.
Therefore, the QNM description of the GW signal is only expected to be valid at sufficiently late times, when the nonlinearities from the merger have largely decayed away. 

The remnant Kerr BH has no hair; it is fully described by only a final mass, $M_f$, and a dimensionless final spin parameter, $\chi_f = |\vb*{\chi}_f|$. 
The same is true of the spectrum of QNM frequencies, $\omega_{\ell m n}(M_f, \chi_f)$, which are also functions of only the mass and spin. 
Individual QNMs are indexed by the triplet $(\ell, m, n)$ which are the polar ($\ell\geq2$), azimuthal ($-\ell \leq m \leq \ell$) and overtone ($n \geq 0$) numbers respectively. 
The spectrum is further complicated by the fact that QNMs occur in pairs. 
A complete description of the ringdown must include the \emph{mirror modes} $\omega'_{\ell m n}$ \cite{Berti:2009kk, Berti:2005ys, Dhani:2020nik, London:2014cma} with negative real frequency $f'_{\ell m n}$ along with the \emph{regular modes} $\omega_{\ell m n}$ with $f_{\ell m n}>0$
\footnote{We choose to classify QNMs as either \emph{regular} or \emph{mirror}. This is closely related to, but still distinct from, the prograde/retrograde classification of QNMs used in, for example, \cite{LIGOScientific:2020tif}.}.
A quantification of the mirror modes was treated in Appendix D of \cite{JimenezForteza:2020cve}; some of these estimates were later confirmed in \cite{Dhani:2020nik}.
The spectrum of mirror modes contains the same information as the regular modes (albeit with nontrivial relationships between them, see Eqs.~\ref{eq:sym_mirror_modes_conj}) which has sometimes led to them being neglected. 
Whether they can, in fact, be neglected will depend on the relative excitation amplitudes of the regular and mirror modes and their differing decay times. 
In general, the ringdown will contain a superposition of all these modes with different excitation amplitudes and phases (see Eq.~\ref{general_ringdown}). 
Usually, the GW strain is dominated by the $\ell=|m|=2$ modes. 
Furthermore, the overtones decay more quickly (i.e.\ $\tau$ decreases) with increasing $n$ so that at late times the signal will be dominated by the fundamental $n=0$ modes. 
Therefore, the most prominent QNM in the ringdown is expected to be the $(\ell, m, n)=(2,2,0)$ mode, and the observational challenge is usually to detect the presence of other, subdominant modes.

The study of QNMs has applications in both astro and fundamental physics. 
The highly constrained dependence of the QNM spectrum on only the remnant mass and spin means that, conversely, if a QNM frequency is measured, then the mass and spin of the final BH merger can be inferred. 
For high-mass systems, where only the ringdown signal is observable, this may be the only information available about the nature of the source \cite{Berti:2005ys, Baibhav:2020tma}. 
For lower-mass systems, measuring QNM frequencies allows us to estimate the remnant properties independently of the rest of the signal, and so consistency tests can be performed. 
For example, a test of the BH area theorem can be performed in this way \cite{Cabero:2017avf, Isi:2020tac}. 
A similar consistency test using full inspiral-merger-ringdown models and a sharp cut in the frequency (rather than time) domain was performed on GW150914 \cite{LIGOScientific:2016lio}. 
Each additional QNM that can be detected in the ringdown provides a separate estimate of the mass and spin of the remnant. 
Therefore, if multiple QNM frequencies can be identified, a ringdown-only consistency test on the expected Kerr-like nature of the remnant BH can be performed \cite{Dreyer:2003bv, Carullo:2019flw} (this is possible only if the $(\ell, m, n)$ of the modes are known). 
In these tests, deviations from the expected results may point to new physics beyond general relativity. 

QNMs also have practical uses in waveform modelling.
They are used in full inspiral-merger-ringdown BBH waveforms produced in both the phenomenological \cite{Pratten:2020ceb, Garcia-Quiros:2020qpx, Pratten:2020fqn} and effective-one-body approaches \cite{Buonanno:2006ui, Buonanno:2007pf, Pan:2011gk}.


\section{Scalar field on Schwarzschild background}

An example calculation of quasinormal modes to give some idea where they come from?

\section{Quasinormal modes from the geodesic correspondence}

We focus on the $\ell = m$ case, since these modes are associated with equatorial motion. 

First, we need the metric associated with a stationary and axisymmetric spacetime. 
The stationary and axisymmetric character requires that the metric coefficients be independent of $t$ and $\phi$, so that $g_{\mu \nu} = g_{\mu \nu}(r,\theta)$.
We also require that the spacetime is invariant to the simultaneous inversion of the time $t$ and the angle $\phi$ (i.e.\ to the transformation $t \rightarrow -t$, $\phi \rightarrow -\phi$). 
The physical meaning is that the spacetime we are considering is that associated with a rotating body. 
This invariance requires 
\begin{equation}
	g_{tr} = g_{t \theta} = g_{\phi r} = g_{\phi \theta} = 0.
\end{equation}
Then we have 
\begin{align}
	\dd s^2 &= g_{tt}\dd t^2 + 2g_{t \phi} \dd t \dd \phi + g_{\phi \phi}\dd \phi^2 \nonumber \\
	&+ \qty[ g_{rr}\dd r^2 + 2g_{r \theta} \dd r \dd \theta + g_{\theta \theta} \dd \theta^2 ].
\end{align}
It can be shown \cite{Chandrasekhar:1985kt} that the term in square brackets can be brought to the diagonal form $g_{r'r'}\dd r'^2 +  g_{\theta' \theta'} \dd \theta'^2$ by a change of coordinates $r'=r'(r,\theta)$ and $\theta'=\theta'(r,\theta)$.
Renaming our variables by removing the primes, this gives
\begin{equation}
	\dd s^2 = g_{tt}\dd t^2 + g_{rr}\dd r^2 + g_{\theta \theta}\dd \theta^2 + g_{\phi \phi}\dd \phi^2 + 2g_{t \phi}\dd t\dd \phi
\end{equation}

We can find geodesic curves $x^\mu(\lambda)$ by extremising the action $S=\int\mathrm{d}\lambda\,\mathcal{L}$ where the Lagrangian is given by
\begin{align}
	\mathcal{L} &= \frac{1}{2}g_{\mu \nu} \dot{x}^\mu \dot{x}^\nu \\
	&= \frac{1}{2}\qty(g_{tt}\dot{t}^2 + g_{rr}\dot{r}^2 + g_{\theta \theta}\dot{\theta}^2 + g_{\phi \phi}\dot{\phi}^2 + 2g_{t \phi}\dot{t}\dot{\phi}), \nonumber
\end{align}
and a dot denotes a derivative with respect to the affine parameter $\lambda$ along the curve. 

We could find the second order differential geodesic equations from the Euler-Lagrange (EL) equations, 
\begin{gather} \label{eq:ELeqns_1}
	\dv{\lambda}(\pdv{\mathcal{L}}{\dot{x}^\mu}) = \pdv{\mathcal{L}}{x^\mu}.
\end{gather}
However, first we recognise that the spacetime, and hence the action, are stationary; therefore the timelike component of the 4-momentum is a constant of the motion
\begin{equation}\label{eq:el_t} 
	\pdv{\mathcal{L}}{\dot{t}} = -E \;\implies\; g_{tt}\dot{t} + g_{t\phi}\dot{\phi} = -E.
\end{equation}
Similarly, from the axisymmetry of the spacetime we have another constant of motion $L$,
\begin{equation}\label{eq:el_phi}
	\pdv{\mathcal{L}}{\dot{\phi}} =L \;\implies\; g_{\phi \phi}\dot{\phi} + g_{t\phi}\dot{t} = L.
\end{equation}
From Eqs.~\ref{eq:el_t} and \ref{eq:el_phi} we can solve for the two components of the 4-velocity $\dot{t}$ and $\dot{\phi}$ to give
\begin{gather} \label{eq:tdot_L}
	\dot{t} = E \frac{g_{\phi \phi} + g_{t \phi}\hat{L}}{\qty(g_{t \phi})^2 - g_{t t}g_{\phi \phi}} \\
	\label{eq:phidot_L}
	\dot{\phi} = E \frac{g_{t \phi} + g_{t t}\hat{L}}{g_{t t}g_{\phi \phi} - \qty(g_{t \phi})^2}
\end{gather}
where $\hat{L} = L/E$.
We are also free to rescale our affine parameter $\lambda\rightarrow E\lambda$ to remove $E$ from the above expressions.
The azimuthal orbital frequency can also be calculated:
\begin{gather}
	\Omega_\phi = \frac{\mathrm{d}\phi}{\mathrm{d}t} =  \frac{\dot{\phi}}{\dot{t}} = -\frac{g_{t\phi}+g_{tt}\hat{L}}{g_{\phi\phi}+g_{t\phi}\hat{L}}
\end{gather}

In general, for the remaining two coordinates, $r$ and $\theta$, we can use the EL equations to find the second order differential geodesic equations.
However, in the special case of equatorial motion ($\theta=\pi/2\;\implies\;\dot{\theta}=0$) we can get a first order equation for $\dot{r}$ by considering the normalization of the four-velocity;
\begin{equation} \label{eq:rdot}
	g_{\mu\nu}\dot{x}^\mu\dot{x}^\mu = 0 \;\implies\; \dot{r}^2 = V_{\rm eff}(r;\hat{L}),
\end{equation}
where
\begin{align}\label{eq:Veff}
	V_{\rm eff}(r;\hat{L}) &= \frac{-g_{tt}\dot{t}^2 -g_{\phi\phi}\dot{\phi}^2-2g_{t\phi}\dot{t}\dot{\phi} }{g_{rr}} \nonumber \\
	&= \frac{g_{tt} \hat{L}^2+2 g_{t\phi } \hat{L}+g_{\phi\phi}}{g_{rr}(g_{t\phi}^2-g_{tt} g_{\phi \phi})},
\end{align}
and in the final line we have used Eqs.~\ref{eq:tdot_L} and \ref{eq:phidot_L} to eliminate $\dot{t}$ and $\dot{\phi}$.
In Eq.~\ref{eq:Veff} it is to be understood that the $g_{\mu\nu}$ metric coefficients are to be evaluated on the equatorial plane $\theta=\pi/2$ and so are only functions of $r$.

A \emph{light ring} is circular null geodesic orbit. 
The radius, $r_*$, and angular momentum, $\hat{L}_*$, of such an orbit must satisfy $V_{\rm eff} = V'_{\rm eff} = 0$, where a prime denotes a radial derivative with respect to $r$. The first condition yields
\begin{equation}
	\hat{L}_*(r) = \frac{-g_{t \phi} \pm \sqrt{g_{t \phi}^2 - g_{t t}g_{\phi \phi}}}{g_{t t}},
\end{equation}
while the second gives implicit formula for $r_*$:
\begin{gather}
	V'_{\rm eff}\big(r_*;\hat{L}_*(r_*)\big) = 0.
\end{gather}
In the case of the Kerr metric single root $r_*$.

Having found the equations of the light ring, now consider neighboring geodesics. 
First, consider polar motion. The Euler-Lagrange for $\theta$ (Eq.~\ref{eq:ELeqns_1} with $x^\mu=\theta$) is 
\begin{align}\label{eq:el_theta}
	g_{\theta \theta} \ddot{\theta} + & \qty(\pdv{g_{\theta \theta}}{\theta} \dot{\theta} + \pdv{g_{\theta \theta}}{r} \dot{r}) \dot{\theta}
	= \frac{1}{2} \bigg(\pdv{g_{t t}}{\theta} \dot{t}^2 + \nonumber \\ &\pdv{g_{r r}}{\theta} \dot{r}^2 + \pdv{g_{\theta \theta}}{\theta} \dot{\theta}^2 + \pdv{g_{\phi \phi}}{\theta} \dot{\phi}^2 + 2\pdv{g_{t \phi}}{\theta} \dot{t}\dot{\phi}\bigg).
\end{align}
Consider small perturbations in the $\theta$ direction \footnote{It is sufficient to consider $\theta$ and $r$ perturbations separately\ldots} about the light ring; i.e. set $r = r_*$, $\theta = \pi/2+\delta\theta(\lambda)$, and where $\dot{t}$ and $\dot{\phi}$ are given by Eqs.~\ref{eq:tdot_L} and \ref{eq:phidot_L} respectively and discard terms $\mathcal{O}(\delta\theta^2)$. 
This gives
\begin{align}\label{eq:el_theta_expanded}
	g_{\theta \theta}&(r_*,\pi/2) \ddot{\delta \theta} = \frac{1}{2} \bigg(\pdv[2]{g_{t t}(r_*,\pi/2)}{\theta} \dot{t}^2  
	+ \nonumber\\ & \pdv[2]{g_{\phi\phi}(r_*,\pi/2)}{\theta} \dot{\phi}^2 
	+2\pdv[2]{g_{t \phi}(r_*,\pi/2)}{\theta} \dot{t}\dot{\phi}\bigg)\delta \theta.
\end{align}
which describes simple harmonic motion, $\ddot{\delta\theta}=-\tilde{\Omega}^2_\theta\delta\theta$, where the constant $\tilde{\Omega}_\theta$ is the frequency of the oscillations with respect to the parameter $\lambda$.
The frequency of the oscillations with respect to coordinate time $t$ is given by
\begin{align}
	\Omega_\theta = \frac{\tilde{\Omega}_\theta}{\dot{t}} =\sqrt{-\frac{
			\pdv[2]{g_{t t}}{\theta} +2\pdv[2]{g_{t \phi}}{\theta} \Omega_\phi + \pdv[2]{g_{\phi\phi}}{\theta} \Omega_\phi^2
		}{2g_{\theta \theta}}} ,
\end{align}
where all quantities on the right hand side are to be evaluated at the light ring. 
In the case of the Kerr metric $\Omega_\theta^2>0$ and the light ring is stable in the polar direction.

Now consider motion in the radial direction ($\dot{r}\neq 0$).
Differentiating Eq.~\ref{eq:rdot} with respect to $\lambda$ gives
\begin{align}
	\ddot{r} = \frac{1}{2}V'_{\rm eff}(r).
\end{align}
Consider small perturbations $r = r_*+\delta r(\lambda)$ with $\theta = \pi/2$ and discarding $\mathcal{O}(\delta r^2)$ terms gives
\begin{align}
	\ddot{\delta r} = \frac{1}{2}V''_{\rm eff}(r_*)\delta r.
\end{align}
Looking for periodic solutions, the frequency of the radial oscillaitons (with respect to coordinate time) is given by
\begin{align}
	\Omega_{r} = \sqrt{-\frac{V''_{\rm eff}(r_*)}{2\dot{t}^2}}.
\end{align}
In the case of the Kerr metric the light ring orbit has $\Omega_r^2<0$ and is unstable in the radial direction; therefore $\Gamma = (-\Omega_r^2)^{-1/2}$ is the instability (Lyapunov) timescale.

\section{The Kerr Spectrum}

An explanation of the conventions used, and maybe also the general waveform (although it make make sense to have that right at the top). 

\begin{figure}[h]
	\centering
	\includegraphics[width=\columnwidth]{IntroductiontoBlackHoleRingdown/qnm_taxonomy.pdf}
	\caption[The Kerr quasinormal mode spectrum]{ 
		The Kerr quasinormal mode spectrum.}
	\label{fig:ch1:qnm_taxonomy}
\end{figure}