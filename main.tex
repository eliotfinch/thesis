%%%%%%%%%%%%%%%%%%%%%%%%%%%%%%%%%%%%%%%%%
% Masters/Doctoral Thesis 
% LaTeX Template
% Version 2.5 (27/8/17)
%
% This template was downloaded from:
% http://www.latextemplates.com/template/masters-doctoral-thesis
%
% Version 2.x major modifications by:
% Vel (vel@latextemplates.com)
%
% This template is based on a template by:
% Steve Gunn (http://users.ecs.soton.ac.uk/srg/softwaretools/document/templates/)
% Sunil Patel (http://www.sunilpatel.co.uk/thesis-template/)
%
% Template license:
% CC BY-NC-SA 3.0 (http://creativecommons.org/licenses/by-nc-sa/3.0/)
%
% Additional edits:
% Change biblatex options to use number (Vancouver) style referencing
% Remove font changes (mainly to make math symbols match figures)
% Using single line spacing for the frontmatter only, and double line spacing elsewhere
% New title page
% Adjust caption settings (found in the class)
% Changed geometry headheight from 4ex to 24pt to avoid a warning (found in the class)
% Re-organised order of opening sections
% Changed link colour (found in the class)
%
%%%%%%%%%%%%%%%%%%%%%%%%%%%%%%%%%%%%%%%%%

%----------------------------------------------------------------------------------------
%	PACKAGES AND OTHER DOCUMENT CONFIGURATIONS
%----------------------------------------------------------------------------------------

\documentclass[
12pt, % The default document font size, options: 10pt, 11pt, 12pt
%oneside, % Two side (alternating margins) for binding by default, uncomment to switch to one side
english, % ngerman for German
doublespacing, % Single line spacing, alternatives: onehalfspacing or doublespacing
%draft, % Uncomment to enable draft mode (no pictures, no links, overfull hboxes indicated)
%nolistspacing, % If the document is onehalfspacing or doublespacing, uncomment this to set spacing in lists to single
%liststotoc, % Uncomment to add the list of figures/tables/etc to the table of contents
%toctotoc, % Uncomment to add the main table of contents to the table of contents
%parskip, % Uncomment to add space between paragraphs
%nohyperref, % Uncomment to not load the hyperref package
headsepline, % Uncomment to get a line under the header
%chapterinoneline, % Uncomment to place the chapter title next to the number on one line
%consistentlayout, % Uncomment to change the layout of the declaration, abstract and acknowledgements pages to match the default layout
]{MastersDoctoralThesis} % The class file specifying the document structure

\usepackage[utf8]{inputenc} % Required for inputting international characters
\usepackage[T1]{fontenc} % Output font encoding for international characters

% \usepackage{mathpazo} % Use the Palatino font by default

\usepackage{microtype}

\usepackage{physics}
\usepackage{amssymb}
% \usepackage{aas_macros}
% \usepackage{aastex631}

% Link colours
\usepackage[dvipsnames]{xcolor}

% Side-by-side subfigures/captions
\usepackage[caption=false]{subfig}

\newcommand{\dvec}[1]{\vb*{#1}}
\newcommand{\pvec}[1]{\vec{#1}}
\newcommand\params{\ensuremath{\vec{\theta}}}

\newcommand\comment[1]{{\color{red} #1}}

\usepackage[sorting=none, citestyle=numeric-comp]{biblatex} % Use the bibtex backend with the authoryear citation style (which resembles APA)
\addbibresource{bibliography.bib} % The filename of the bibliography
\renewbibmacro{in:}{}

\usepackage[autostyle=true]{csquotes} % Required to generate language-dependent quotes in the bibliography

%----------------------------------------------------------------------------------------
%	MARGIN SETTINGS
%----------------------------------------------------------------------------------------

\geometry{
    paper=a4paper, % Change to letterpaper for US letter
    top=3cm,
    bottom=3cm,
    inner=3cm,
    outer=2cm,
    % inner=2.5cm, % Inner margin
    % outer=3.8cm, % Outer margin
    % bindingoffset=.5cm, % Binding offset
    % top=1.5cm, % Top margin
    % bottom=1.5cm, % Bottom margin
    %showframe, % Uncomment to show how the type block is set on the page
}

%----------------------------------------------------------------------------------------
%	THESIS INFORMATION
%----------------------------------------------------------------------------------------

\thesistitle{Black-hole Ringdown: Quasinormal Modes in Numerical-relativity Simulations and Gravitational-wave Observations} % Your thesis title, this is used in the title and abstract, print it elsewhere with \ttitle
\supervisor{Dr. Christopher J. Moore} % Your supervisor's name, this is used in the title page, print it elsewhere with \supname
\examiner{} % Your examiner's name, this is not currently used anywhere in the template, print it elsewhere with \examname
\degree{Doctor of Philosophy} % Your degree name, this is used in the title page and abstract, print it elsewhere with \degreename
\author{Eliot Finch} % Your name, this is used in the title page and abstract, print it elsewhere with \authorname

\keywords{} % Keywords for your thesis, this is not currently used anywhere in the template, print it elsewhere with \keywordnames
\university{University of Birmingham} % Your university's name and URL, this is used in the title page and abstract, print it elsewhere with \univname
\department{School of Physics \& Astronomy} % Your department's name and URL, this is used in the title page and abstract, print it elsewhere with \deptname
\group{Institute for Gravitational Wave Astronomy} % Your research group's name and URL, this is used in the title page, print it elsewhere with \groupname
\faculty{College of Engineering and Physical Sciences} % Your faculty's name and URL, this is used in the title page and abstract, print it elsewhere with \facname

\AtBeginDocument{
\hypersetup{pdftitle=\ttitle} % Set the PDF's title to your title
\hypersetup{pdfauthor=\authorname} % Set the PDF's author to your name
\hypersetup{pdfkeywords=\keywordnames} % Set the PDF's keywords to your keywords
}

\begin{document}

\frontmatter % Use roman page numbering style (i, ii, iii, iv...) for the pre-content pages

\pagestyle{plain} % Default to the plain heading style until the thesis style is called for the body content

% Use single spacing only for the frontmatter
\begin{singlespacing}

%----------------------------------------------------------------------------------------
%	TITLE PAGE
%----------------------------------------------------------------------------------------

\begin{titlepage}
\begin{center}

% \vspace*{.06\textheight}
\includegraphics[width=0.8\columnwidth]{Figures/full-colour-logo.pdf}
% \vspace{.03\textheight}

\HRule \\[0.2cm] % Horizontal line
{\scshape\huge Black-hole Ringdown:\par}\vspace{0.4cm} % Thesis title
Quasinormal Modes in Numerical-relativity Simulations and \\
Gravitational-wave Observations\vspace{0.1cm}
\HRule \\[1.cm] % Horizontal line

by\\[0.9cm]
{\scshape \large \authorname}\\[2.cm]

A thesis submitted to the University of Birmingham for the degree of \\[0.1cm]
{\textsc \degreename}\\[2cm]

\vfill

\begin{flushright} \normalsize
\groupname\\
\deptname\\
\facname\\
\univname\\[0.7cm]
April 2023

\end{flushright}
 
\end{center}
\end{titlepage}

%----------------------------------------------------------------------------------------
%	DECLARATION PAGE
%----------------------------------------------------------------------------------------

% \begin{declaration}
% \addchaptertocentry{\authorshipname} % Add the declaration to the table of contents
% \noindent I, \authorname, declare that this thesis titled, \enquote{\ttitle} and the work presented in it are my own. I confirm that:

% \begin{itemize} 
% \item This work was done wholly or mainly while in candidature for a research degree at this University.
% \item Where any part of this thesis has previously been submitted for a degree or any other qualification at this University or any other institution, this has been clearly stated.
% \item Where I have consulted the published work of others, this is always clearly attributed.
% \item Where I have quoted from the work of others, the source is always given. With the exception of such quotations, this thesis is entirely my own work.
% \item I have acknowledged all main sources of help.
% \item Where the thesis is based on work done by myself jointly with others, I have made clear exactly what was done by others and what I have contributed myself.\\
% \end{itemize}
 
% \noindent Signed:\\
% \rule[0.5em]{25em}{0.5pt} % This prints a line for the signature
 
% \noindent Date:\\
% \rule[0.5em]{25em}{0.5pt} % This prints a line to write the date
% \end{declaration}

% \cleardoublepage

%----------------------------------------------------------------------------------------
%	ABSTRACT PAGE
%----------------------------------------------------------------------------------------

\begin{abstract}
\addchaptertocentry{\abstractname} % Add the abstract to the table of contents
The study of black-hole ringdown, the endpoint of a binary black-hole merger, is a cornerstone of gravitational-wave astronomy. 
A black hole's characteristic quasinormal-mode spectrum offers the possibility of clean and powerful tests of general relativity, encoding the properties of the remnant black hole (and, in principle, the binary from which it was formed) in the superposition of exponentially-damped sinusoids which constitute the ringdown signal. 
The observation of gravitational-wave signals is now routine, and the burgeoning field of black-hole spectroscopy is already contributing to our knowledge of the gravitational-wave sky.

In this thesis we investigate the ringdown in two ways.
Firstly, we develop our ringdown models by performing fits directly to state-of-the-art numerical-relativity simulations.
Our analysis considers, for the first time, a large selection of black-hole binaries with misaligned spins.
In this clean and noise-free regime we can understand which quasinormal modes are important to include in our models, and it also informs us on the choice of ringdown start time.
These studies help to guide our expectations when we analyse real gravitational-wave data.
Secondly, we develop a novel frequency-domain ringdown analysis method for use with real data, applying it to both simulated signals and to real gravitational-wave data.
Our method solves many of the problems associated with conventional ringdown-only analyses.
Applying the method to GW150914, we quantify the significance with which a ringdown overtone can be found in the data.
\end{abstract}

%----------------------------------------------------------------------------------------
%	DEDICATION
%----------------------------------------------------------------------------------------

% \dedicatory{For/Dedicated to/To my\ldots} 

%----------------------------------------------------------------------------------------
%	ACKNOWLEDGEMENTS
%----------------------------------------------------------------------------------------

\begin{acknowledgements}
\addchaptertocentry{\acknowledgementname} % Add the acknowledgements to the table of contents

The acknowledgments and the people to thank go here, don't forget to include your project advisor\ldots
\end{acknowledgements}

\end{singlespacing}

%----------------------------------------------------------------------------------------
%	QUOTATION PAGE
%----------------------------------------------------------------------------------------

% \vspace*{0.2\textheight}

% \noindent\enquote{\itshape A quote.}\bigbreak

% \hfill A name

%----------------------------------------------------------------------------------------
%	LIST OF CONTENTS/FIGURES/TABLES PAGES
%----------------------------------------------------------------------------------------

\tableofcontents % Prints the main table of contents

\listoffigures % Prints the list of figures

% \listoftables % Prints the list of tables

%----------------------------------------------------------------------------------------
%	ABBREVIATIONS
%----------------------------------------------------------------------------------------

% \begin{abbreviations}{ll} % Include a list of abbreviations (a table of two columns)

% \textbf{LAH} & \textbf{L}ist \textbf{A}bbreviations \textbf{H}ere\\
% \textbf{WSF} & \textbf{W}hat (it) \textbf{S}tands \textbf{F}or\\

% \end{abbreviations}

%----------------------------------------------------------------------------------------
%	PHYSICAL CONSTANTS/OTHER DEFINITIONS
%----------------------------------------------------------------------------------------

% \begin{constants}{lr@{${}={}$}l} % The list of physical constants is a three column table

% % The \SI{}{} command is provided by the siunitx package, see its documentation for instructions on how to use it

% Speed of Light & $c_{0}$ & \SI{2.99792458e8}{\meter\per\second} (exact)\\
% %Constant Name & $Symbol$ & $Constant Value$ with units\\

% \end{constants}

%----------------------------------------------------------------------------------------
%	SYMBOLS
%----------------------------------------------------------------------------------------

% \begin{symbols}{lll} % Include a list of Symbols (a three column table)

% $a$ & distance & \si{\meter} \\
% $P$ & power & \si{\watt} (\si{\joule\per\second}) \\
% %Symbol & Name & Unit \\

% \addlinespace % Gap to separate the Roman symbols from the Greek

% $\omega$ & angular frequency & \si{\radian} \\

% \end{symbols}

%----------------------------------------------------------------------------------------
%	THESIS CONTENT - CONTRIBUTIONS
%----------------------------------------------------------------------------------------

\chapter*{Contributions Overview}

This thesis contains the following published works (Refs.~\cite{Finch:2021iip, Finch:2021qph, Finch:2022ynt}):
\begin{itemize}
\item[{[1]}] Eliot Finch and Christopher J. Moore. ``Modeling the ringdown from precessing black hole binaries''. Published in \href{https://journals.aps.org/prd/abstract/10.1103/PhysRevD.103.084048}{Phys. Rev. D 103, 084048 (2021)}.
\item[{[2]}] Eliot Finch and Christopher J. Moore. ``Frequency-domain analysis of black-hole
ringdowns''. Published in \href{https://journals.aps.org/prd/abstract/10.1103/PhysRevD.104.123034}{Phys. Rev. D 104, 123034 (2021)}.
\item[{[3]}] Eliot Finch and Christopher J. Moore. ``Searching for a ringdown overtone in
GW150914''. Published in \href{https://journals.aps.org/prd/abstract/10.1103/PhysRevD.106.043005}{Phys. Rev. D 106, 043005 (2022)}.
\end{itemize}
Supplementary material and data releases for the above works are available at Refs.~\cite{finch_eliot_2021_4538194, finch_eliot_2021_5569759, finch_eliot_2022_6949492}.

\section*{Chapter contributions and synopses}

Chapter~\ref{Chapter1} is an introductory chapter, containing standard results and calculations. 
I wrote this chapter with input from my supervisor, and I created all figures. 

\vspace{0.2cm}

\noindent In Chapter~\ref{Chapter1} we provide wider context to the study of black-hole ringdown.
This includes gravitational waves, their production via binary black-hole mergers, and the formation of a initially perturbed remnant black hole.
We discuss the basic morphology of the ringdown signal, and provide details on its constituent quasinormal modes. 
This includes some basic calculations to build intuition.
Finally, we provide motivation for the study of the ringdown in the form of tests of general relativity. 

\begin{center}
    \rule[.5ex]{.5\textwidth}{.5pt}
\end{center}

\noindent Chapter~\ref{Chapter2} is a reformatted version of Ref.~\cite{Finch:2021iip}.
The introduction has been re-worked to fit into the thesis format, and some minor edits have been made to the text to include new references and improve readability. 
The study was conceived with help from my supervisor and co-author, Christopher J.\ Moore.
I performed all analyses, wrote the code (available at Ref.~\cite{qnmfits}), and created all figures.
I wrote the majority of the text, with input from my supervisor.

\vspace{0.2cm}

\noindent The study of Chapter~\ref{Chapter2} was prompted by numerical work of Giesler et al., who found that the linear ringdown description could be extended to relatively early times if multiple quasinormal-mode overtones were included in the model.
We generalised this work (which considered only aligned-spin systems) to precessing systems, testing the validity of the model across a wider parameter space.
We find a much greater variation in the performance of ringdown fits than in the aligned-spin case. 
The inclusion of mirror modes and higher harmonics, along with overtones, improves the reliability of ringdown fits with an early start time (perhaps a sign that mode mixing in the ringdown is generally more important in precessing systems); however, there remain cases with poor performing fits.
This suggests that it is not possible to reliably model the ringdown from as early a time as claimed in the work of Giesler et al.

\begin{center}
    \rule[.5ex]{.5\textwidth}{.5pt}
\end{center}

\noindent Chapter~\ref{Chapter3} is a reformatted version of Ref.~\cite{Finch:2021qph}. 
The introduction has been re-worked to fit into the thesis format, and some minor edits have been made to the text to include new references and improve readability. 
The idea for the data-analysis method presented in this chapter was proposed by my supervisor, and jointly developed to its final form.
I performed all analyses, wrote the code (available at Ref.~\cite{fdringdown}), and created all figures. 
I wrote the majority of the text, with input from my supervisor.

\vspace{0.2cm}

\noindent In Chapter~\ref{Chapter3} we propose a novel frequency-domain ringdown analysis method, and demonstrate its use with simulated gravitational-wave signals.
Our approach avoids the issues of spectral leakage that would normally be expected (associated with the abrupt start of the ringdown) by modelling the inspiral and merger parts of the signal using a flexible sum of sine-Gaussian wavelets truncated at the onset of the ringdown; these effectively marginalise out the inspiral and merger.
Performing the analysis in the frequency domain allows us to use standard (and by now well-established) Bayesian inference pipelines for gravitational wave data as well as giving us the ability to readily search over the sky position and the ringdown start time, although we find that it is necessary to use an informative prior for the latter. 
We test our method by using it to analyse several simulated signals with varying signal-to-noise ratios injected into two- and three-detector networks. 
We find that our frequency-domain approach is generally able to place tighter constraints on the remnant black-hole mass and spin than a standard time-domain analysis.

\begin{center}
    \rule[.5ex]{.5\textwidth}{.5pt}
\end{center}

\noindent Chapter~\ref{Chapter4} is a reformatted version of Ref.~\cite{Finch:2022ynt}. 
The introduction has been re-worked to fit into the thesis format, and some minor edits have been made to the text to improve readability. 
I conceived the study, and performed all analyses with input from my supervisor.
With the exception of Figs.~\ref{fig:other_QNMs} and \ref{fig:skymap} (created by my supervisor) I created all figures, and wrote the majority of the text.

\vspace{0.2cm}

\noindent Prompted by disagreements in the literature, in Chapter~\ref{Chapter4} we reanalyse the GW150914 data searching for quasinormal mode overtone using our frequency-domain approach.
Our analysis has several advantages compared to other analyses; in particular, the source sky position and the ringdown start time are marginalised over (as opposed to simply being fixed) as part of a Bayesian ringdown analysis. 
We find tentative evidence for an overtone in GW150914, but at a lower significance than reported elsewhere. 
Our preferred analysis, marginalising over the uncertainty in the time of peak strain amplitude, gives a posterior on the overtone amplitude peaked away from zero at $\sim 1.8 \sigma$.

%----------------------------------------------------------------------------------------
%	THESIS CONTENT - CHAPTERS
%----------------------------------------------------------------------------------------

\mainmatter % Begin numeric (1,2,3...) page numbering

\pagestyle{thesis} % Return the page headers back to the "thesis" style

% Include the chapters of the thesis as separate files from the Chapters folder
% Uncomment the lines as you write the chapters

% Chapter 1

\chapter{Introduction to Black-hole Ringdown} % Main chapter title

\label{Chapter1} % For referencing the chapter elsewhere, use \ref{Chapter1} 

\section{Ringdown}

A high-level overview of black holes, ringdown, and the usefulness of the ringdown. 
This will motivate the three main chapters of the thesis. 
Specifically, this will aim to introduce
\begin{itemize}
	\item Binary black-hole mergers
	\item Perturbed black holes and quasinormal modes (the basic idea, and a brief history of results)
	\item The no-hair theorem and tests of it
\end{itemize}

% From paper 1
% ------------

The gravitational wave (GW) observatories LIGO \cite{LIGOScientific:2014pky} and Virgo \cite{VIRGO:2014yos} have now observed dozens of GW events \cite{LIGOScientific:2018mvr, LIGOScientific:2020ibl}, mostly from binary black hole (BBH) mergers. 
Particularly prominent in the GW signals of the higher-mass systems are the final few wave cycles, known as the \emph{ringdown}, emitted as the system settles into its final state: a Kerr black hole (BH). 
The ringdown signal contains a superposition of oscillatory modes, the frequency spectrum of which is characteristic of the remnant BH.

The characteristic oscillations of the remnant BH are called \emph{quasinormal modes} (QNMs), so-called because, unlike normal modes, they decay over time.
The QNM frequencies are complex, $\omega = 2\pi f - {i}/\tau$, with the real part $f$ giving the oscillation frequency and the reciprocal of the imaginary part $\tau$ giving the damping time. 
The QNM frequencies can be calculated within the framework of linearised gravity, treating the gravitational field in the vicinity of the remnant as a small (linear) perturbation of the Kerr metric \cite{Berti:2009kk}.
Therefore, the QNM description of the GW signal is only expected to be valid at sufficiently late times, when the nonlinearities from the merger have largely decayed away. 

The remnant Kerr BH has no hair; it is fully described by only a final mass, $M_f$, and a dimensionless final spin parameter, $\chi_f = |\vb*{\chi}_f|$. 
The same is true of the spectrum of QNM frequencies, $\omega_{\ell m n}(M_f, \chi_f)$, which are also functions of only the mass and spin. 
Individual QNMs are indexed by the triplet $(\ell, m, n)$ which are the polar ($\ell\geq2$), azimuthal ($-\ell \leq m \leq \ell$) and overtone ($n \geq 0$) numbers respectively. 
The spectrum is further complicated by the fact that QNMs occur in pairs. 
A complete description of the ringdown must include the \emph{mirror modes} $\omega'_{\ell m n}$ \cite{Berti:2009kk, Berti:2005ys, Dhani:2020nik, London:2014cma} with negative real frequency $f'_{\ell m n}$ along with the \emph{regular modes} $\omega_{\ell m n}$ with $f_{\ell m n}>0$
\footnote{We choose to classify QNMs as either \emph{regular} or \emph{mirror}. This is closely related to, but still distinct from, the prograde/retrograde classification of QNMs used in, for example, \cite{LIGOScientific:2020tif}.}.
A quantification of the mirror modes was treated in Appendix D of \cite{JimenezForteza:2020cve}; some of these estimates were later confirmed in \cite{Dhani:2020nik}.
The spectrum of mirror modes contains the same information as the regular modes (albeit with nontrivial relationships between them, see Eqs.~\ref{eq:sym_mirror_modes_conj}) which has sometimes led to them being neglected. 
Whether they can, in fact, be neglected will depend on the relative excitation amplitudes of the regular and mirror modes and their differing decay times. 
In general, the ringdown will contain a superposition of all these modes with different excitation amplitudes and phases (see Eq.~\ref{general_ringdown}). 
Usually, the GW strain is dominated by the $\ell=|m|=2$ modes. 
Furthermore, the overtones decay more quickly (i.e.\ $\tau$ decreases) with increasing $n$ so that at late times the signal will be dominated by the fundamental $n=0$ modes. 
Therefore, the most prominent QNM in the ringdown is expected to be the $(\ell, m, n)=(2,2,0)$ mode, and the observational challenge is usually to detect the presence of other, subdominant modes.

The study of QNMs has applications in both astro and fundamental physics. 
The highly constrained dependence of the QNM spectrum on only the remnant mass and spin means that, conversely, if a QNM frequency is measured, then the mass and spin of the final BH merger can be inferred. 
For high-mass systems, where only the ringdown signal is observable, this may be the only information available about the nature of the source \cite{Berti:2005ys, Baibhav:2020tma}. 
For lower-mass systems, measuring QNM frequencies allows us to estimate the remnant properties independently of the rest of the signal, and so consistency tests can be performed. 
For example, a test of the BH area theorem can be performed in this way \cite{Cabero:2017avf, Isi:2020tac}. 
A similar consistency test using full inspiral-merger-ringdown models and a sharp cut in the frequency (rather than time) domain was performed on GW150914 \cite{LIGOScientific:2016lio}. 
Each additional QNM that can be detected in the ringdown provides a separate estimate of the mass and spin of the remnant. 
Therefore, if multiple QNM frequencies can be identified, a ringdown-only consistency test on the expected Kerr-like nature of the remnant BH can be performed \cite{Dreyer:2003bv, Carullo:2019flw} (this is possible only if the $(\ell, m, n)$ of the modes are known). 
In these tests, deviations from the expected results may point to new physics beyond general relativity. 

QNMs also have practical uses in waveform modelling.
They are used in full inspiral-merger-ringdown BBH waveforms produced in both the phenomenological \cite{Pratten:2020ceb, Garcia-Quiros:2020qpx, Pratten:2020fqn} and effective-one-body approaches \cite{Buonanno:2006ui, Buonanno:2007pf, Pan:2011gk}.


\section{Scalar field on Schwarzschild background}

An example calculation of quasinormal modes to give some idea where they come from?

\section{Quasinormal modes from the geodesic correspondence}

We focus on the $\ell = m$ case, since these modes are associated with equatorial motion. 

First, we need the metric associated with a stationary and axisymmetric spacetime. 
The stationary and axisymmetric character requires that the metric coefficients be independent of $t$ and $\phi$, so that $g_{\mu \nu} = g_{\mu \nu}(r,\theta)$.
We also require that the spacetime is invariant to the simultaneous inversion of the time $t$ and the angle $\phi$ (i.e.\ to the transformation $t \rightarrow -t$, $\phi \rightarrow -\phi$). 
The physical meaning is that the spacetime we are considering is that associated with a rotating body. 
This invariance requires 
\begin{equation}
	g_{tr} = g_{t \theta} = g_{\phi r} = g_{\phi \theta} = 0.
\end{equation}
Then we have 
\begin{align}
	\dd s^2 &= g_{tt}\dd t^2 + 2g_{t \phi} \dd t \dd \phi + g_{\phi \phi}\dd \phi^2 \nonumber \\
	&+ \qty[ g_{rr}\dd r^2 + 2g_{r \theta} \dd r \dd \theta + g_{\theta \theta} \dd \theta^2 ].
\end{align}
It can be shown \cite{Chandrasekhar:1985kt} that the term in square brackets can be brought to the diagonal form $g_{r'r'}\dd r'^2 +  g_{\theta' \theta'} \dd \theta'^2$ by a change of coordinates $r'=r'(r,\theta)$ and $\theta'=\theta'(r,\theta)$.
Renaming our variables by removing the primes, this gives
\begin{equation}
	\dd s^2 = g_{tt}\dd t^2 + g_{rr}\dd r^2 + g_{\theta \theta}\dd \theta^2 + g_{\phi \phi}\dd \phi^2 + 2g_{t \phi}\dd t\dd \phi
\end{equation}

We can find geodesic curves $x^\mu(\lambda)$ by extremising the action $S=\int\mathrm{d}\lambda\,\mathcal{L}$ where the Lagrangian is given by
\begin{align}
	\mathcal{L} &= \frac{1}{2}g_{\mu \nu} \dot{x}^\mu \dot{x}^\nu \\
	&= \frac{1}{2}\qty(g_{tt}\dot{t}^2 + g_{rr}\dot{r}^2 + g_{\theta \theta}\dot{\theta}^2 + g_{\phi \phi}\dot{\phi}^2 + 2g_{t \phi}\dot{t}\dot{\phi}), \nonumber
\end{align}
and a dot denotes a derivative with respect to the affine parameter $\lambda$ along the curve. 

We could find the second order differential geodesic equations from the Euler-Lagrange (EL) equations, 
\begin{gather} \label{eq:ELeqns_1}
	\dv{\lambda}(\pdv{\mathcal{L}}{\dot{x}^\mu}) = \pdv{\mathcal{L}}{x^\mu}.
\end{gather}
However, first we recognise that the spacetime, and hence the action, are stationary; therefore the timelike component of the 4-momentum is a constant of the motion
\begin{equation}\label{eq:el_t} 
	\pdv{\mathcal{L}}{\dot{t}} = -E \;\implies\; g_{tt}\dot{t} + g_{t\phi}\dot{\phi} = -E.
\end{equation}
Similarly, from the axisymmetry of the spacetime we have another constant of motion $L$,
\begin{equation}\label{eq:el_phi}
	\pdv{\mathcal{L}}{\dot{\phi}} =L \;\implies\; g_{\phi \phi}\dot{\phi} + g_{t\phi}\dot{t} = L.
\end{equation}
From Eqs.~\ref{eq:el_t} and \ref{eq:el_phi} we can solve for the two components of the 4-velocity $\dot{t}$ and $\dot{\phi}$ to give
\begin{gather} \label{eq:tdot_L}
	\dot{t} = E \frac{g_{\phi \phi} + g_{t \phi}\hat{L}}{\qty(g_{t \phi})^2 - g_{t t}g_{\phi \phi}} \\
	\label{eq:phidot_L}
	\dot{\phi} = E \frac{g_{t \phi} + g_{t t}\hat{L}}{g_{t t}g_{\phi \phi} - \qty(g_{t \phi})^2}
\end{gather}
where $\hat{L} = L/E$.
We are also free to rescale our affine parameter $\lambda\rightarrow E\lambda$ to remove $E$ from the above expressions.
The azimuthal orbital frequency can also be calculated:
\begin{gather}
	\Omega_\phi = \frac{\mathrm{d}\phi}{\mathrm{d}t} =  \frac{\dot{\phi}}{\dot{t}} = -\frac{g_{t\phi}+g_{tt}\hat{L}}{g_{\phi\phi}+g_{t\phi}\hat{L}}
\end{gather}

In general, for the remaining two coordinates, $r$ and $\theta$, we can use the EL equations to find the second order differential geodesic equations.
However, in the special case of equatorial motion ($\theta=\pi/2\;\implies\;\dot{\theta}=0$) we can get a first order equation for $\dot{r}$ by considering the normalization of the four-velocity;
\begin{equation} \label{eq:rdot}
	g_{\mu\nu}\dot{x}^\mu\dot{x}^\mu = 0 \;\implies\; \dot{r}^2 = V_{\rm eff}(r;\hat{L}),
\end{equation}
where
\begin{align}\label{eq:Veff}
	V_{\rm eff}(r;\hat{L}) &= \frac{-g_{tt}\dot{t}^2 -g_{\phi\phi}\dot{\phi}^2-2g_{t\phi}\dot{t}\dot{\phi} }{g_{rr}} \nonumber \\
	&= \frac{g_{tt} \hat{L}^2+2 g_{t\phi } \hat{L}+g_{\phi\phi}}{g_{rr}(g_{t\phi}^2-g_{tt} g_{\phi \phi})},
\end{align}
and in the final line we have used Eqs.~\ref{eq:tdot_L} and \ref{eq:phidot_L} to eliminate $\dot{t}$ and $\dot{\phi}$.
In Eq.~\ref{eq:Veff} it is to be understood that the $g_{\mu\nu}$ metric coefficients are to be evaluated on the equatorial plane $\theta=\pi/2$ and so are only functions of $r$.

A \emph{light ring} is circular null geodesic orbit. 
The radius, $r_*$, and angular momentum, $\hat{L}_*$, of such an orbit must satisfy $V_{\rm eff} = V'_{\rm eff} = 0$, where a prime denotes a radial derivative with respect to $r$. The first condition yields
\begin{equation}
	\hat{L}_*(r) = \frac{-g_{t \phi} \pm \sqrt{g_{t \phi}^2 - g_{t t}g_{\phi \phi}}}{g_{t t}},
\end{equation}
while the second gives implicit formula for $r_*$:
\begin{gather}
	V'_{\rm eff}\big(r_*;\hat{L}_*(r_*)\big) = 0.
\end{gather}
In the case of the Kerr metric single root $r_*$.

Having found the equations of the light ring, now consider neighboring geodesics. 
First, consider polar motion. The Euler-Lagrange for $\theta$ (Eq.~\ref{eq:ELeqns_1} with $x^\mu=\theta$) is 
\begin{align}\label{eq:el_theta}
	g_{\theta \theta} \ddot{\theta} + & \qty(\pdv{g_{\theta \theta}}{\theta} \dot{\theta} + \pdv{g_{\theta \theta}}{r} \dot{r}) \dot{\theta}
	= \frac{1}{2} \bigg(\pdv{g_{t t}}{\theta} \dot{t}^2 + \nonumber \\ &\pdv{g_{r r}}{\theta} \dot{r}^2 + \pdv{g_{\theta \theta}}{\theta} \dot{\theta}^2 + \pdv{g_{\phi \phi}}{\theta} \dot{\phi}^2 + 2\pdv{g_{t \phi}}{\theta} \dot{t}\dot{\phi}\bigg).
\end{align}
Consider small perturbations in the $\theta$ direction \footnote{It is sufficient to consider $\theta$ and $r$ perturbations separately\ldots} about the light ring; i.e. set $r = r_*$, $\theta = \pi/2+\delta\theta(\lambda)$, and where $\dot{t}$ and $\dot{\phi}$ are given by Eqs.~\ref{eq:tdot_L} and \ref{eq:phidot_L} respectively and discard terms $\mathcal{O}(\delta\theta^2)$. 
This gives
\begin{align}\label{eq:el_theta_expanded}
	g_{\theta \theta}&(r_*,\pi/2) \ddot{\delta \theta} = \frac{1}{2} \bigg(\pdv[2]{g_{t t}(r_*,\pi/2)}{\theta} \dot{t}^2  
	+ \nonumber\\ & \pdv[2]{g_{\phi\phi}(r_*,\pi/2)}{\theta} \dot{\phi}^2 
	+2\pdv[2]{g_{t \phi}(r_*,\pi/2)}{\theta} \dot{t}\dot{\phi}\bigg)\delta \theta.
\end{align}
which describes simple harmonic motion, $\ddot{\delta\theta}=-\tilde{\Omega}^2_\theta\delta\theta$, where the constant $\tilde{\Omega}_\theta$ is the frequency of the oscillations with respect to the parameter $\lambda$.
The frequency of the oscillations with respect to coordinate time $t$ is given by
\begin{align}
	\Omega_\theta = \frac{\tilde{\Omega}_\theta}{\dot{t}} =\sqrt{-\frac{
			\pdv[2]{g_{t t}}{\theta} +2\pdv[2]{g_{t \phi}}{\theta} \Omega_\phi + \pdv[2]{g_{\phi\phi}}{\theta} \Omega_\phi^2
		}{2g_{\theta \theta}}} ,
\end{align}
where all quantities on the right hand side are to be evaluated at the light ring. 
In the case of the Kerr metric $\Omega_\theta^2>0$ and the light ring is stable in the polar direction.

Now consider motion in the radial direction ($\dot{r}\neq 0$).
Differentiating Eq.~\ref{eq:rdot} with respect to $\lambda$ gives
\begin{align}
	\ddot{r} = \frac{1}{2}V'_{\rm eff}(r).
\end{align}
Consider small perturbations $r = r_*+\delta r(\lambda)$ with $\theta = \pi/2$ and discarding $\mathcal{O}(\delta r^2)$ terms gives
\begin{align}
	\ddot{\delta r} = \frac{1}{2}V''_{\rm eff}(r_*)\delta r.
\end{align}
Looking for periodic solutions, the frequency of the radial oscillaitons (with respect to coordinate time) is given by
\begin{align}
	\Omega_{r} = \sqrt{-\frac{V''_{\rm eff}(r_*)}{2\dot{t}^2}}.
\end{align}
In the case of the Kerr metric the light ring orbit has $\Omega_r^2<0$ and is unstable in the radial direction; therefore $\Gamma = (-\Omega_r^2)^{-1/2}$ is the instability (Lyapunov) timescale.

\section{The Kerr Spectrum}

An explanation of the conventions used, and maybe also the general waveform (although it make make sense to have that right at the top). 

\begin{figure}[h]
	\centering
	\includegraphics[width=\columnwidth]{IntroductiontoBlackHoleRingdown/qnm_taxonomy.pdf}
	\caption[The Kerr quasinormal mode spectrum]{ 
		The Kerr quasinormal mode spectrum.}
	\label{fig:ch1:qnm_taxonomy}
\end{figure}
% Chapter 2

\chapter{Modelling the Ringdown from Precessing Black-hole Binaries}
\label{Chapter2}

\section{Introduction}
\label{ch2:sec:introduction}

In 2019 Giesler et al.~\cite{Giesler:2019uxc} demonstrated that using QNMs with $n > 0$ (that is, ringdown ``overtones'') could push the validity of the linear ringdown model to times as early as the peak of the GW strain. 
Their work involved fitting ringdown models to a selection of aligned-spin SXS simulations, and in this chapter we extend their work to misaligned-spin (i.e.\ precessing) systems.

The 2019 study sparked many other papers which involve fitting ringdown models with overtones to NR simulations~\cite{Bhagwat:2019dtm, Ota:2019bzl, JimenezForteza:2020cve, Cook:2020otn, Dhani:2020nik, Mourier:2020mwa, Dhani:2021vac, Forteza:2021wfq, MaganaZertuche:2021syq} (including the present work), but it should be noted that fears of over-fitting and the physical validity of the overtones were also present.
Even at the time of writing there is not a consensus on this point~\cite{Baibhav:2023clw, Nee:2023osy}, a central issue being that of the ringdown start time.

A prerequisite for any ringdown analysis is a suitable choice for the start time, $t_0$, of the ringdown. 
Starting too early risks obtaining biased measurements, because a GW signal contaminated with nonlinearities from the merger cannot be described by a model based solely on QNMs.
On the other hand, due to the exponential decay of the ringdown, starting too late leaves too little SNR to make useful measurements. 
In ringdown studies, typically the start time is given in reference to the maxima of some time-dependent quantity.
This could be the modulus of the $(2,2)$ mode of the strain or the $\Psi_4$ Weyl scalar, or the total GW luminosity; these quantities peak at times that typically differ by a few tens of $M$ (see Ref.~\cite{Berti:2007fi} for a discussion of some possible choices for the ringdown start time).
Often, to avoid concerns of fitting to the nonlinear merger, the ringdown start time is chosen to be $10M$ to $20M$ after the peak of these reference quantities. 
The work of Giesler et al.~\cite{Giesler:2019uxc}, which itself builds on previous studies of fitting ringdown models to NR simulations~\cite{Dorband:2006gg, Buonanno:2006ui, Berti:2007fi, Kamaretsos:2011um, Kamaretsos:2012bs, London:2014cma, Baibhav:2017jhs}, found that by including up to seven overtones the ringdown analysis can be started as early as the peak of the $(2,2)$ mode strain. 
This might be considered a surprising result; the signal peak is expected to occur when the remnant BH (to the extent that it yet even makes sense to consider it as such) is most highly distorted and linear perturbation theory is not expected to be valid. 
The failure of this intuition was investigated in Ref.~\cite{Okounkova:2020vwu} which suggests much of the nonlinearity is trapped behind a forming common apparent horizon and never makes it out to future null infinity in the form of GWs. 
Further support came from a study on the overtone excitation factors~\cite{Oshita:2021iyn}, which quantify the ease of excitation of the modes, and it was found that higher overtones are relatively easy to excite.
Even more surprising, Dhani~\cite{Dhani:2020nik} extended this approach via the inclusion of mirror modes along with overtones (thereby doubling the number of QNMs) and it was found that it was possible to start the ringdown analysis even earlier (up to $10M$ before the peak). 
Clearly it is not surprising that a model with so many free parameters is able to fit the GW signal well; the important point is that it is able to do so without obtaining biased values for the final mass and spin. 

The previous studies mentioned have only considered aligned-spin BBH systems, although we note that Ref.~\cite{Kamaretsos:2012bs} performed some limited analyses on precessing simulations.
We also note that some work on precessing systems has been done in the extreme mass ratio limit, see Refs.~\cite{Hughes:2019zmt,Lim:2019xrb,Lim:2022veo}.
It is well known that misalignment between the orbital angular momentum and the spins of the component BHs cause the orbit to precess during the inspiral phase of the evolution, leading to qualitatively different GW signals at early times (see, e.g., Ref.~\cite{Apostolatos:1994mx}). 
It is less clear what effect, if any, misaligned component spins would have on the late-time ringdown signal which is generally associated with the remnant BH. 
The primary aim of this chapter is to address this question by systematically extending the analyses of Refs.~\cite{Giesler:2019uxc, Dhani:2020nik} to a large number of precessing BBH simulations from the SXS catalog~\cite{Boyle:2019kee}. 
We find that for BBH systems with misaligned spins, and that exhibit precession during their inspiral phase, a model consisting only of overtones (with or without mirror modes) cannot be reliably applied from the peak amplitude of the $(2,2)$ strain. 
A more conservative ringdown start time corresponding to the peak of the total energy flux (i.e., the GW luminosity) improves reliability, but we still see significant variation in performance across different simulations. 
The introduction of a higher harmonic (QNMs with $\ell > 2$) to the overtone model helps to reduce this variation, hinting at the importance of mode mixing.

Previous studies have focused on using full NR simulations to test ringdown models. 
In this chapter we also briefly investigate the use of surrogates, which provide an opportunity to test models over a continuous parameter space. 
We find caution should be taken, particularly for surrogates of precessing systems, due to errors in the surrogate waveforms.

In Section~\ref{ch2:sec:model} we write down the most general form of the ringdown model which will be used throughout this chapter.
The method used to fit the ringdown model to the SXS simulations is then explained in Section~\ref{ch2:sec:fitting}.
In Section~\ref{aligned-spin-section} we reproduce some important results from Refs.~\cite{Giesler:2019uxc, Dhani:2020nik}, which are later compared with those for precessing systems in Section~\ref{misaligned-spin-section}. 
With precessing systems, it is necessary to perform a frame rotation to account for the fact that the spin of the remnant BH will not be aligned with the initial coordinate axes used to set up the simulation; the procedure for doing this is also discussed in Section~\ref{misaligned-spin-section}. 
In Section~\ref{surrogate-section} we comment on the use of NR surrogates to test ringdown models, and in Section~\ref{NR_error_appendix} we discuss the estimation of numerical errors present in the NR simulations.
Finally, concluding remarks are presented in Section~\ref{sec:discussion}. 
Throughout, we use units in which $G=c=1$.

\section{Model for the spherical modes}
\label{ch2:sec:model}

NR expands the GW strain in the basis of (spin-weighted) spherical harmonics
\begin{equation}\label{ch2:eq:spherical_expansion}
    h = \sum_{\ell = 2}^\infty \sum_{m = -\ell}^\ell h_{\ell m}(t) {}_{-2}Y_{\ell m}(\Omega),
\end{equation}
where $\Omega$ is used as shorthand for the angles $\theta$, $\phi$.
By convention, the NR frame is uniquely fixed by requiring that initially the two component BHs are located on the $x$-axis and the orbital angular momentum, $\vb*{L}$, points along the $z$-axis.
The $h_{\ell m}(t)$ coefficients are referred to as the spherical-harmonic modes of the GW signal.
The $\ell=\abs{m}=2$ modes are typically largest, while the remaining ``higher modes'' are generally subdominant.
The output of an NR simulation usually includes the first few modes (e.g.\ $\ell \leq 8$) with the asymptotic radial dependence scaled out.
The spherical-harmonic modes are defined with respect to a particular frame at infinity, chosen such that the centre-of-mass of the system is at rest at some initial time. 
Note, however, that this still leaves freedom to perform an overall rotation (as will become important when we discuss precessing systems).

At late times ($t \geq t_0$, where $t_0$ is to be determined), perturbation theory expands the GW strain in the basis of the (spin-weighted) spheroidal harmonics
\begin{equation}\label{ch2:eq:spheroidal_expansion}
    h = \sum_{\ell =2}^\infty \sum_{m = -\ell}^\ell \sum_{n = 0}^\infty \left[ C_{\ell m n} e^{-i \omega_{\ell m n} (t - t_0)} {}_{-2}S_{\ell m n}(\Omega) + C'_{\ell m n} e^{-i \omega'_{\ell m n} (t - t_0)} {}_{-2}S'_{\ell m n}(\Omega) \right].
\end{equation}
Here, $C_{\ell m n}$ are complex coefficients (containing an amplitude and a phase), $\omega_{\ell m n} = 2\pi f_{\ell m n} - i/\tau_{\ell m n}$ are the complex QNM frequencies (which are functions of the remnant BH mass $M_f$ and spin $\chi_f$), and ${}_{-2}S_{\ell m n}(\Omega) = {}_{-2}S_{\ell m}(\Omega; a\omega_{\ell m n})$ are the spheroidal harmonics.
The spheroidal harmonics are functions of the spheroidicity $\gamma = a\omega_{\ell m n}$, where $a = M_f \chi_f$ is the Kerr parameter.
The primes denote the mirror modes, which satisfy $\operatorname{Re}[\omega'_{\ell m n}] = 2\pi f'_{\ell m n} < 0$.
As discussed in Section~\ref{ch1:sec:bh_spectroscopy}, these mirror modes are related to the regular modes via $\omega'_{\ell m n} = -\omega^*_{\ell -m n}$.
The prime on the spheroidal harmonic enters in the spheroidicity: ${}_{-2}S'_{\ell m n}(\Omega) = {}_{-2}S_{\ell m}(\Omega; a\omega'_{\ell m n})$.

It is important to note that Eq.~\ref{ch2:eq:spheroidal_expansion} is valid in a frame in which the remnant BH is at rest, with its spin vector pointing along the positive $z$-direction (such a frame is unique up to an unimportant rotation about the $z$-axis).
This ringdown frame is only the same as the NR frame for aligned-spin BBH systems (it is possible that systems with large component spins in the negative $z$-direction will exhibit a ``spin flip'', where the final spin also points in the negative $z$-direction; in these cases the two frames will only differ by a sign).
For misaligned-spin systems the remnant spin can point in essentially any direction and the NR and ringdown frames are misaligned; in these instances, as explained in Section~\ref{misaligned-spin-section}, we need to rotate the frame of the NR simulation to bring it into the ringdown frame where we can then apply Eq.~\ref{ch2:eq:spheroidal_expansion}.
The ringdown frame will also be moving with respect to the NR frame as a result of the recoil, or kick, from the anisotropic emission of GWs near merger. 
The effects of the kick are neglected here; it is assumed that the NR and ringdown frames are related by a rotation.

Assuming that the remnant BH spin vector is aligned with the $z$-axis in the NR frame (that is, the required rotation has been applied to the NR spherical-harmonic modes), we can equate Eqs.~\ref{ch2:eq:spherical_expansion} and \ref{ch2:eq:spheroidal_expansion} to get
\begin{equation}
    \sum_{\ell m} h_{\ell m}(t) {}_{-2}Y_{\ell m}(\Omega) = \sum_{\ell m n} \left[ C_{\ell m n} e^{-i \omega_{\ell m n} t} {}_{-2}S_{\ell m n}(\Omega) + C'_{\ell m n} e^{-i \omega'_{\ell m n} t} {}_{-2}S'_{\ell m n}(\Omega) \right],
\end{equation}
where we have dropped the limits on the sums for clarity.
We can then extract $h_{\ell m}$ using spherical-harmonic orthogonality:
\begin{align}\label{eq:hlm_with_ints}
    h_{\ell' m'}(t) = \sum_{\ell m n} \bigg[ &C_{\ell m n} e^{-i \omega_{\ell m n} t} \left(\int_\Omega \dd{\Omega} ~ {}_{-2}S_{\ell m n}(\Omega) ~ {}_{-2}Y^*_{\ell' m'}(\Omega)\right) \nonumber \\
    + &C'_{\ell m n} e^{-i \omega'_{\ell m n} t} \left(\int_\Omega \dd{\Omega} ~ {}_{-2}S'_{\ell m n}(\Omega) ~ {}_{-2}Y^*_{\ell' m'}(\Omega)\right) \bigg].
\end{align}
Following the convention of Ref.~\cite{Stein:2019mop}, the first integral is the spherical-spheroidal mixing coefficient:
\begin{equation}\label{ch2:eq:mu}
    \int_\Omega \dd{\Omega} ~ {}_{-2}S_{\ell m n}(\Omega) ~ {}_{-2}Y^*_{\ell' m'}(\Omega) = \mu_{\ell' m' \ell n} \delta_{m' m}.
\end{equation}
To evaluate the second integral we first rewrite the primed spheroidal harmonic as 
\begin{align}\label{eq:spheroidal_transform}
    {}_{-2}S'_{\ell m n}(\Omega) &= {}_{-2}S_{\ell m}(\theta, \phi; a\omega'_{\ell m n}) \nonumber \\
    &= {}_{-2}S_{\ell m}(\theta, \phi; -a\omega^*_{\ell -m n}) \nonumber \\
    &= (-1)^\ell {}_{-2}S^*_{\ell -m}(\pi - \theta, \phi; a\omega_{\ell -m n})
\end{align}
where the last line follows from Eqs.~48b and 48c of Ref.~\cite{Cook:2014cta}.
Next, we use the symmetries of the spherical harmonics to write
\begin{equation}\label{eq:spherical_transform}
    {}_{-2}Y^*_{\ell m}(\theta, \phi) = (-1)^{\ell} ~ {}_{-2}Y_{\ell -m}(\pi - \theta, \phi).
\end{equation}
Using Eqs.~\ref{eq:spheroidal_transform} and \ref{eq:spherical_transform} we can rewrite the second integral of Eq.~\ref{eq:hlm_with_ints} as
\begin{align}
    \int_\Omega \dd{\Omega}& ~ {}_{-2}S'_{\ell m n}(\Omega) ~ {}_{-2}Y^*_{\ell' m'}(\Omega) = \nonumber \\ 
    &= \int_\Omega \dd{\Omega} ~ (-1)^\ell {}_{-2}S^*_{\ell -m}(\pi - \theta, \phi; a\omega_{\ell -m n}) ~ (-1)^{\ell'} ~ {}_{-2}Y_{\ell' -m'}(\pi - \theta, \phi) \nonumber \\
    &= (-1)^{l+l'} \int_\Omega \dd{\Omega} ~ {}_{-2}S^*_{\ell -m}(\pi - \theta, \phi; a\omega_{\ell -m n}) ~ {}_{-2}Y_{\ell' -m'}(\pi - \theta, \phi) \nonumber \\
    &= (-1)^{l+l'} \mu^*_{\ell' -m' \ell n} \delta_{m' m}.
\end{align}
Substituting for both of the integrals of Eq.~\ref{eq:hlm_with_ints} we get
\begin{align}\label{ch2:eq:hlm_model}
    h_{\ell' m'}(t) &= \sum_{\ell m n} \left[ C_{\ell m n} e^{-i \omega_{\ell m n} t} \mu_{\ell' m' \ell n} \delta_{m' m} + C'_{\ell m n} e^{-i \omega'_{\ell m n} t} (-1)^{l+l'} \mu^*_{\ell' -m' \ell n} \delta_{m' m} \right] \nonumber \\
    &= \sum_{\ell n} \left[ C_{\ell m' n} e^{-i \omega_{\ell m' n} t} \mu_{\ell' m' \ell n} + C'_{\ell m' n} e^{-i \omega'_{\ell m' n} t} (-1)^{l+l'} \mu^*_{\ell' -m' \ell n} \right] \nonumber \\
    &= \sum_{\ell n} \left[ C_{\ell m' n} e^{-i \omega_{\ell m' n} t} \mu_{\ell' m' \ell n} + C'_{\ell m' n} e^{-i \omega'_{\ell m' n} t} \mu'_{\ell' m' \ell n} \right],
\end{align}
where $\mu'_{\ell' m' \ell n} = (-1)^{l+l'} \mu^*_{\ell' -m' \ell n}$.
Eq.~\ref{ch2:eq:hlm_model} tells us how a given spherical-harmonic mode (as provided by NR simulations) can be expressed in terms of QNMs. 
It reveals that each spherical-harmonic mode has contributions from every QNM of the same $m$, weighted by the spherical-spheroidal mixing coefficients; this is an effect known as mode mixing~\cite{Berti:2014fga}.

\section{Fitting implementation}
\label{ch2:sec:fitting}

Given some spherical-harmonic modes $h_{\ell m}$, we can turn Eq.~\ref{ch2:eq:hlm_model} into a least-squares fitting problem to find the best-fit complex coefficients $C_{\ell m n}$ and $C'_{\ell m n}$ (this assumes the complex frequencies and mixing coefficients are also given). 
First, we write Eq.~\ref{ch2:eq:hlm_model} as a matrix equation. 
Note that in the above we have separated the regular and mirror modes to show explicitly how to deal with the mirror modes (i.e. how to obtain their frequencies and mixing coefficients from the regular frequencies and mixing coefficients). 
We will now drop this distinction for clarity.

\subsection{Single-mode fit}

For simplicity, first consider the case when we want to model a single spherical mode (for example, the $h_{22}$ mode). We write
\begin{equation}\label{eq:hlm_matrix_single}
    \vb*{h}_{\ell m} = \vb*{a}_{\ell m} \vdot \vb*{C},
\end{equation}
where $\vb*{h}_{\ell m} = \qty(h_{\ell m}(t_0),\ h_{\ell m}(t_1),\ \ldots,\ h_{\ell m}(t_{K-1}))$ is the waveform data discretely sampled at a total of $K$ times labelled by $t_k$. 
In general $t_0$ may not exist on the default array of simulation times, so care must be taken (for example, we can interpolate the simulation data and evaluate on a new grid of times, or we could use the first value after $t_0$, or the closest value to $t_0$). 

The matrix $\vb*{a}_{\ell m}$ is where the choice of QNM content in our model enters. 
It has the form
\begin{equation}
    \vb*{a}_{\ell m} = 
    \begin{pmatrix}
    e^{-i \omega_0 (t_0 - t_0)} \mu_{\ell m,0} & e^{-i \omega_1 (t_0 - t_0)} \mu_{\ell m,1} & \cdots & e^{-i \omega_{J-1} (t_0 - t_0)} \mu_{\ell m,J-1} \\ 
    e^{-i \omega_0 (t_1 - t_0)} \mu_{\ell m,0} & e^{-i \omega_1 (t_1 - t_0)} \mu_{\ell m,1} & \cdots & e^{-i \omega_{J-1} (t_1 - t_0)} \mu_{\ell m,J-1} \\ 
    \vdots & \vdots & \ddots & \vdots \\
    e^{-i \omega_0 (t_{K-1} - t_0)} \mu_{\ell m,0} & e^{-i \omega_1 (t_{K-1} - t_0)} \mu_{\ell m,1} & \cdots & e^{-i \omega_{J-1} (t_{K-1} - t_0)} \mu_{\ell m,J-1}
    \end{pmatrix},
\end{equation}
where we have suppressed the three QNM labels and instead labelled each QNM by a single number. 
There are a total of $J$ QNMs included in the model.
So, $\vb*{a}_{\ell m}$ is a matrix of shape $(K, J)$.
To reiterate, the QNM content in our model can consist of any regular or mirror modes, as long as the correct frequencies and mixing coefficients are used in the above matrix (we have just used $\omega_j$ and $\mu_{\ell m, j}$ as generic terms). 

The vector $\vb*{C}$ contains our complex amplitudes, of which there are a total of $J$ (one for each QNM included in the model). This has the form
\begin{equation}
    \vb*{C} = \left(C_0,\ C_1,\ \ldots,\ C_{J-1}\right)^T.
\end{equation}
When we perform the matrix multiplication of Eq.~\ref{eq:hlm_matrix_single} we are multiplying a matrix of shape $(K,J)$ by a vector of length $J$, so we are left with a vector of length $K$.
Written in this form, we can easily apply least-squares solvers to invert the equation for the vector of coefficients. 
This minimises the Euclidean 2-norm $\norm{\vb*{h}_{\ell m} - \vb*{a}_{\ell m} \vdot \vb*{C}}$ (i.e.\ the sum of the squares of the fit residuals).
In this chapter we will only be dealing with single-mode fits as described here.
However, for completeness, below we describe how multimode fits can be implemented (available in the code developed for this work~\cite{qnmfits}).
We also note that the multimode-fit formalism is used in Fig.~\ref{fig:amp_ratio} to predict QNM amplitudes.

\subsection{Multimode fit}

Due to mode mixing, a given QNM contributes to all spherical-harmonic modes with the same $m$. 
This means we can perform a fit to multiple spherical-harmonic modes with a single set of shared QNM amplitudes $\vb*{C}$ (see, e.g.\ Fig.~\ref{fig:amp_ratio}).
We will approach this by joining $\vb*{h}_{\ell m}$ vectors together for each $(\ell, m)$ we want to include in the fit (to effectively form a single time series), and similarly by ``stacking'' $\vb*{a}_{\ell m}$ matrices on top of each other.

The single-mode matrix equation, Eq.~\ref{eq:hlm_matrix_single}, becomes
\begin{equation}\label{eq:hlm_matrix_multi}
    \vb*{h} = \vb*{a} \vdot \vb*{C},
\end{equation}
where
\begin{equation}
    \vb*{h} = 
    \begin{bmatrix}
    \vb*{h}_0 & \vb*{h}_1 & \cdots & \vb*{h}_{I-1}
    \end{bmatrix}
\end{equation}
and $I$ is the number of spherical-harmonic modes to include in the fit (we have suppressed the two spherical-harmonic indices for clarity). So, $\vb*{h}$ is a vector of length $I \times K$.
Similarly
\begin{equation}
    \vb*{a} = 
    \begin{bmatrix}
    \vb*{a}_0 \\ \vb*{a}_1 \\ \vdots \\ \vb*{a}_{I-1}
    \end{bmatrix},
\end{equation}
which has shape $(I \times K, J)$. 
We see that when we multiply this new coefficient matrix by the vector $\vb*{C}$ (length $J$) in Eq.~\ref{eq:hlm_matrix_multi} we recover a vector of correct length $I \times K$.
The quantity we're minimising is now
\begin{equation}
    \norm{\vb*{h} - \vb*{a} \vdot \vb*{C}} = \sqrt{ \sum_{\ell m} \norm{\vb*{h}_{\ell m} - \vb*{a}_{\ell m} \vdot \vb*{C}}^2 }
\end{equation}
which gives equal ``weight'' to each $(\ell,m)$ mode.
%, and can be shown to be equivalent as averaging the least-squares fit over the sky.


\section{Aligned-spin systems}\label{aligned-spin-section}

Following Giesler et al.~\cite{Giesler:2019uxc}, the spherical-harmonic modes of the ringdown signal can be modelled by writing each as a sum of $N$ overtones:
\begin{equation}\label{GieslerRD}
    h_{\ell m}^N(t) = \sum_{n=0}^N C_{\ell m n} e^{-i\omega_{\ell m n}(t-t_0)}, \quad \textrm{for} \quad t \geq t_0.
\end{equation}
This \emph{overtone} model is a restriction of the sum in Eq.~\ref{ch2:eq:hlm_model}, where overlaps between different harmonic $\ell$ indices (mode mixing) as well as mirror modes are neglected. 
As in Ref.~\cite{Giesler:2019uxc}, we model each spherical-harmonic mode individually as a sum of QNMs.
In Ref.~\cite{Giesler:2019uxc}, the efficacy of this model for $\ell=m=2$ was demonstrated by performing least squares fits to the $h_{22}$ mode for a selection of aligned-spin SXS simulations. The authors note that this was also verified for other values of $(\ell,m)$.

The overtone model in Eq.~\ref{GieslerRD} contains $2(N+1)$ free parameters in the complex amplitudes, $C_{\ell m n}$, plus the two parameters, $M_f$ and $\chi_f$, that determine the $\omega_{\ell m n}$ frequencies.
All of these parameters depend on the properties of the progenitor binary, but we do not study these dependencies here.

Our fitting algorithm finds the amplitudes $C_{\ell m n}$ that minimise the sum-of-the-squares of the fit residuals.
We find it convenient to treat the remnant property parameters $M_f$ and $\chi_f$ differently from the excitation amplitudes. 
If we also want to minimise over the remnant properties (as opposed to fixing them to the true values given by NR) then first a discrete 2-dimensional numerical grid of values for $M_f$ and $\chi_f$ is constructed.
At each point on this grid, we consider varying only the complex amplitudes $C_{\ell m n}$. 
Eq.~\ref{eq:hlm_matrix_single} turns this minimisation problem into a linear algebra problem that can be efficiently solved with, for example, \texttt{numpy.linalg.lstsq}~\cite{Harris:2020xlr}.
Finally, the point of the grid with the lowest overall value for the sum-of-the-squares of the fit residuals is chosen.

Once the least-squares fit to the data has been obtained, the quality of the fit is quantified via the mismatch and the error on the remnant parameters.
The mismatch between signals $h_1$ and $h_2$ is defined as
\begin{equation}\label{mismatch}
    \mathcal{M} = 1 - \frac{\Re[\braket{h_1}{h_2}]}{\sqrt{\braket{h_1}\braket{h_2}}},
\end{equation}
where we use the following complex inner product~\cite{Nollert:1998ys}
\begin{equation} \label{eq:inner_prod}
    \braket{h_1}{h_2} = \int_{t_0}^T h_1(t) h^*_2(t) ~ \dd t.
\end{equation}
We integrate from the ringdown start time, $t_0$, to an upper limit $T$ chosen such that the whole ringdown is captured (we use $T = t_0 + 100M$).
When fitting models with very small mismatches, the finite accuracy of the NR simulations must be considered; this is discussed in Section~\ref{NR_error_appendix}.
As noted in Ref.~\cite{Giesler:2019uxc}, a small mismatch is not sufficient by itself to justify the model.
The overtone model contains more parameters as $N$ is increased, and it is necessary to check for over-fitting.
To address this, we check to see if the remnant BH properties are correctly recovered by the model. 
The combined error on the remnant mass and spin is quantified by~\cite{Giesler:2019uxc}
\begin{equation} \label{eq:epsilon}
    \epsilon = \sqrt{ \left( \frac{\delta M_f}{M} \right)^2 + \left( \delta\chi_f \right)^2 },
\end{equation}
where $\delta M_f = M_{\mathrm{best fit}} - M_f$, and $\delta \chi_f = \chi_{\mathrm{best fit}} - \chi_f$. 
The best-fit values are those which minimise the mismatch, while the true values are taken from the metadata for the SXS simulation.
A ringdown model can be said to perform well if it yields small values for both $\mathcal{M}$ and $\epsilon$.

\begin{figure}[t]
    \centering
    \includegraphics[width=0.6\columnwidth]{Figures/ModellingTheRingdownFromPrecessingBlackHoleBinaries/305_mismatch_vs_t0_updated.pdf}
    \caption[Mismatch of the overtone model fitted to SXS:BBH:0305]{ 
    Mismatch as a function of ringdown start time for the overtone model (Eq.~\ref{GieslerRD}) when fitting to the $h_{22}$ mode of the NR simulation SXS:BBH:0305. 
    When using only the fundamental $\ell = m = 2$, $n = 0$ QNM the start time that gives the lowest mismatch with the NR data is well after the merger (the rising mismatch at late times is a numerical artefact). 
    However, reproducing the results from Ref.~\cite{Giesler:2019uxc}, we find that by including $N=7$ overtones the GW signal can be fitted using QNMs starting from as early as the peak strain. 
    We also show (in light grey) the mismatch curves obtained when including up to 20 overtones.
    The dashed grey curve shows the estimate of the error in the underlying NR simulation and is described in Section~\ref{NR_error_appendix}.
    }
    \label{305_mismatch_vs_t0}
\end{figure}

\begin{figure}[t]
    \centering
    \includegraphics[width=0.6\columnwidth]{ModellingTheRingdownFromPrecessingBlackHoleBinaries/305_epsilon_grid.pdf}
    \caption[Recovery of SXS:BBH:0305 remnant properties using the overtone model]{ 
    Recovery of the SXS:BBH:0305 remnant properties when fitting the overtone model (Eq.~\ref{GieslerRD}) to the $h_{22}$ mode from the time of its peak strain.
    The heat map shows the mismatch for the fit with $N=7$ overtones, which shows a pronounced minimum close ($\epsilon=3.4\times 10^{-4}$) to the true remnant parameters (indicated by the horizontal and vertical lines).
    The sequence of crosses shows the locations of the minima for fits performed with different values of $N$, all using the same start time (the cross colours correspond to the colours used in Fig.~\ref{305_mismatch_vs_t0}; crosses for $N=5$ and 6 are omitted to avoid crowding the plot, but they converge towards the true remnant parameters). 
    If we choose a different ringdown start time for each $N$ corresponding to the mismatch minima in Fig.~\ref{305_mismatch_vs_t0}, we do see a reduction in $\epsilon$ for the lower $N$ models, however the $N=7$ model with $t_0=t_\mathrm{peak}^{h_{22}}$ remains the best performing model.
    } 
    \label{305_epsilon_grid}
\end{figure}

Following Ref.~\cite{Giesler:2019uxc}, we now apply these ideas to the simulation SXS:BBH:0305~\cite{Lovelace:2016uwp}.
This simulation has source parameters consistent with GW150914 and was originally chosen to demonstrate the success of the overtone model. 
Fig.~\ref{305_mismatch_vs_t0} shows the mismatch values obtained with the overtone model when using the true values of $M_f$ and $\chi_f$.
With $N=7$ (that is, eight QNMs = the fundamental mode + seven overtones) the $h_{22}$ mode can be fitted all the way back to the time of its peak amplitude, $t_{\mathrm{peak}}^{h_{22}}$, while still achieving the smallest possible mismatch.
Using a smaller number of overtones requires a later choice for the start time to achieve the smallest possible mismatch.
A larger number of overtones can also be used (shown with the light grey lines in the figure), and low-mismatch fits as early as $\sim 10M$ before the time of peak amplitude can be achieved.
We show mismatch curves with up to 20 overtones, by which point the inclusion of extra modes does not significantly help.
Ref.~\cite{Giesler:2019uxc} refer to the values of the QNM amplitudes to justify stopping at $N=7$; when performing fits at the time of peak strain the $n=4$ mode has the largest amplitude, with the amplitude of higher overtones decaying rapidly. 
We perform a brief study of the overtone amplitudes in SXS:BBH:0305 (see Fig.~\ref{305_even_more_overtones}), and we also refer the reader to Ref.~\cite{Forteza:2021wfq} for a more in-depth study and for the $(2,2,8)$ QNM frequency data used here).

In addition to giving a small ($\sim 10^{-6}$) mismatch, the $N=7$ overtone model, with $t_0 = t_{\mathrm{peak}}^{h_{22}}$, also achieves this minimum mismatch with the correct values for the remnant properties; this is shown by the heat map in Fig.~\ref{305_epsilon_grid} where the values of $M_f$ and $\chi_f$ are now allowed to vary. 
We find, for the $N=7$ model, a remnant error $\epsilon = 3.4 \cross 10^{-4}$. 
Importantly, this is larger than the NR error on the remnant properties (which is estimated to be $\epsilon_{\mathrm{NR}} = 2.1 \cross 10^{-5}$, see Section~\ref{NR_error_appendix} for details). This confirms that this is really the true scale of the bias in the inferred remnant parameters when using the overtone model, and not just the numerical noise floor in the NR simulation.
Again, using a smaller number of overtones and starting the ringdown as early as $t_{\mathrm{peak}}^{h_{22}}$ gives inferior results with the minimum in the mismatch being biased away from the true parameters.
The results in Figs.~\ref{305_mismatch_vs_t0} and \ref{305_epsilon_grid} show that the overtone model performs well for SXS:BBH:0305 (i.e.\ yields small $\mathcal{M}$ and $\epsilon$) even when starting the ringdown as early as the peak in the strain.

In the top panel of Fig.~\ref{305_even_more_overtones} we show the values of $\epsilon$ obtained with different numbers of overtones (up to $N=20$) and for three different ringdown start times (at the time of the $h_{22}$ amplitude peak, and $5M$ before/after the peak time). 
We see that, when $t_0 = h_\mathrm{peak}^{h_{22}}$ (black line), $N=7$ overtones does the best job at recovering the remnant properties. 
As expected, lower numbers of overtones perform worse (this is already shown in Fig.~\ref{305_epsilon_grid} by the coloured markers).
Interestingly, larger numbers of overtones also perform worse.
We speculate that this is due to over-fitting; as shown in Fig.~\ref{305_mismatch_vs_t0}, using more than seven overtones at this start time does not improve the mismatch significantly, and so the extra free parameters in the model are not necessary.
We have found that using more QNMs than needed in the fits can lead to unstable behaviour (for example, in terms of the QNM amplitudes, which start to vary significantly as more overtones are added). 
Starting at a later time, $5M$ after the peak (red line), we see a similar behaviour but with $N=4$ overtones giving the lowest $\epsilon$ (specifically, the first minimum in the $\epsilon$ vs $N$ curve occurs at $N=4$; the curve eventually reaches lower values of $\epsilon$ at high $N$, but this could be a non-physical result of over-fitting).
This is consistent with what we see in Fig.~\ref{305_mismatch_vs_t0}, where the mismatch curve levels-out with four overtones when starting $\sim 5M$ after the peak (and the above argument regarding over-fitting with additional overtones still holds).
Starting at the earlier time of $5M$ before the peak (blue line), we find that $N=13$ overtones recovers a value of $\epsilon$ which is comparable to the result when seven overtones are fitted from the peak amplitude. 

\begin{figure}[t]
    \centering
    \includegraphics[width=0.9\columnwidth]{Figures/ModellingTheRingdownFromPrecessingBlackHoleBinaries/305_even_more_overtones.pdf}
    \caption[Remnant-property errors and mode amplitudes for different numbers of overtones fitted to SXS:BBH:0305]{ 
    \emph{Top:} The remnant mass-spin error ($\epsilon$) from an overtone model fit to SXS:BBH:0305, for different numbers of overtones in the model ($N$) and for three different ringdown start times (line colours). For each start time there is a choice of $N$ which gives a minimum in the curve (indicated with a black circle).
    \emph{Bottom row:} The absolute value of each best-fit QNM amplitude from a fit at the minimum of the curve from the top panel (indicated by the connecting lines). We report the amplitudes rescaled to what they would be at the time $h_\mathrm{peak}^{h_{22}}$. For the left and right panels, where the fit is performed $5M$ after and before that time respectively, the unscaled amplitudes are shown in light grey.
    }
    \label{305_even_more_overtones}
\end{figure}

This hints at the interesting possibility of using even more overtones and starting the fit at even earlier times, but we note that care should be taken.
Firstly, as shown in Fig.~\ref{305_mismatch_vs_t0}, as we use more overtones we are reaching mismatches further below the estimated waveform accuracy (dashed grey line) and so we are at risk of over-fitting.
And secondly, we see some instability in the overtone amplitudes (which can be an indicator of whether these modes are physical or not).
We demonstrate this with the bottom three panels of Fig.~\ref{305_even_more_overtones}, which show the QNM amplitudes obtained with the lowest-$\epsilon$ fit at each of the three start times.
Being exponentially decaying modes, the value of the amplitude obtained in the fit depends strongly on the value of $t_0$ used.
However, we know the expected decay time $\tau_{\ell m n}$ of each mode, and so to make a fair comparison of mode amplitudes we rescale them to a reference time (we perform the same procedure in Chapter~\ref{Chapter4} for the overtone amplitude, see Fig.~\ref{fig:amp_at_tref}).
We choose to show the amplitudes rescaled to their value at the time of peak $h_{22}$ strain, such that the amplitudes in the middle panel are unchanged.
For the other two panels, in light-grey we show the unscaled amplitude values obtained from the least-squares fit.
If the overtones were physical, we would expect the recovered amplitude to be stable with different choices of start time.
Indeed, there is good agreement for the amplitudes up to and including the $n=3$ mode.
But, this agreement is less clear for the higher overtones.
For example, the recovered amplitude of the $(2,2,7)$ mode is a factor of $\sim 9$ larger when performing the fit $5M$ before the peak (right panel) vs at the time of the peak (middle panel).
See Ref.~\cite{Forteza:2021wfq} for a more in-depth study of going beyond the $n=7$ mode, where a selection of different NR simulations were also considered.
Their conclusions are broadly in agreement with what is shown here; i.e., a larger number of overtones can be shown to fit the data well and recover the remnant properties, but the mode amplitudes show some instability.
For simplicity, in the rest of this chapter we limit ourselves to seven overtones, which also aids comparison with previous work.

\begin{figure}[t]
    \centering
    \includegraphics[width=\columnwidth]{ModellingTheRingdownFromPrecessingBlackHoleBinaries/aligned_spin_epsilon_M_hist.pdf}
    \caption[Remnant-property errors and mismatches for the overtone model fitted to aligned-spin SXS simulations]{\emph{Left:} histograms of the mass-spin remnant error $\epsilon$ from an overtone-model fit to 85 aligned-spin SXS simulations for several different overtone numbers $N$. 
    \emph{Right:} histograms of the mismatch from a fit with the true remnant mass and spin parameters, with the same overtone models and SXS simulations as in the left histogram.
    The solid histograms show results from fits performed starting at the peak of the $h_{22}$ mode with $N$ overtones of the fundamental $\ell = m = 2$ mode.
    The red dashed line shows results from a $N=7$ model that also includes mirror modes (see Section~\ref{subsec:mirror_modes}) and was fitted with a ringdown starting $5M$ before the peak of the strain.}
    \label{aligned_spin_epsilon_hist}
\end{figure}

In order to see how robust the conclusions drawn from SXS:BBH:0305 are in general, the calculations of $\epsilon$ and $\mathcal{M}$ were repeated for a wider selection of SXS simulations. Following Ref.~\cite{Giesler:2019uxc}, we consider only aligned-spin simulations with initial spin magnitudes $|\vb*{\chi}_{1,2}| = \chi_{1,2} < 0.8$, and mass ratios $q < 8$. We also require that the $z$-component of $\vb*{\chi}_f$ is greater than zero, which eliminates the ``spin flip'' systems. The simulations were chosen in the ID range SXS:BBH:1412 to SXS:BBH:1513, as these cover a range of initial spin magnitudes and mass ratios.
After applying these cuts, this left 85 spin-aligned SXS simulations in our test set.
For each simulation, fits were performed using the overtone model with $N=0$, 3, and 7 and with a start time of $t_0 = t_{\mathrm{peak}}^{h_{22}}$. 
The results are shown in Fig.~\ref{aligned_spin_epsilon_hist}. 
We see distributions similar to those in Fig.~3 of Ref.~\cite{Giesler:2019uxc}. 
The inclusion of additional overtones systematically shifts the entirety of both the $\epsilon$ and $\mathcal{M}$ histograms to smaller values.
We note, as it will become important later, that the worst cases in these histograms improve, along with the median values.
This demonstrates that, when using the overtone model on systems with aligned spins, the ringdown reliably starts as early as the peak in the $h_{22}$ mode of the strain. 


\subsection{Mirror modes} \label{subsec:mirror_modes}

For a given $\ell$, $m$ and $n$, the equations governing QNM frequencies allow two solutions: one, $\omega_{\ell m n} = 2\pi f_{\ell m n} - i/\tau_{\ell m n}$, with a positive real part; and another, $\omega'_{\ell m n} = 2\pi f'_{\ell m n} - i/ \tau'_{\ell m n}$, with negative real part~\cite{Dhani:2020nik, Berti:2005ys}.
The frequencies of the mirror modes $\omega'_{\ell m n}$ are related to the regular modes $\omega_{\ell m n}$ by Eq.~\ref{ch1:eq:mirror}.

A new ringdown model which explicitly includes the mirror modes can be written as
\begin{equation}
    h_{\ell m}^{N,\, {\rm mirror}}(t) = \sum_{n=0}^N \qty[ C_{\ell m n} e^{-i \omega_{\ell m n}(t-t_0)} + C'_{\ell m n} e^{-i \omega'_{\ell m n}(t-t_0)} ]\quad \textrm{for} \quad t \geq t_0.
\end{equation}
This \emph{mirror-mode} model is an extension of the overtone model in Eq.~\ref{GieslerRD}; if $C'_{\ell m n} = 0$ the mirror modes aren't excited and we recover the previous overtone model. 
This model has twice as many free parameters as the overtone model; $4(N+1)$ in the complex amplitudes, plus the two remnant parameters $M_f,\; \chi_f$.
The mirror-mode model is still a restriction of the full sum in Eq.~\ref{ch2:eq:hlm_model} as overlaps between modes with different $\ell$ indices (i.e.\ mode mixing) are still not included.
Substituting for $\omega'_{\ell m n}$ using the conjugate symmetry property in Eq.~\ref{ch1:eq:mirror}, we can rewrite the mirror-mode model in the form
\begin{equation} \label{ch2:eq:mirror_model}
   h_{\ell m}^{N,\, {\rm mirror}}(t) = \sum_{n=0}^N \qty[ C_{\ell m n} e^{-i \omega_{\ell m n}(t-t_0)} + C'_{\ell m n} e^{i \omega^*_{\ell -m n}(t-t_0)} ]\quad \textrm{for} \quad t \geq t_0.
\end{equation}
This is how the model was implemented in practice.

As was shown by Dhani~\cite{Dhani:2020nik}, the inclusion of mirror modes can improve the ringdown modelling of aligned-spin systems. In particular, the ringdown can be considered to start even earlier in the waveform, whilst still recovering the correct remnant properties. We confirm this here by repeating the above analysis for the same set of spin-aligned SXS simulations, but now using the mirror-mode model in Eq.~\ref{ch2:eq:mirror_model} with $N=7$ and an earlier choice for the ringdown start time, $t_0 = t_{\mathrm{peak}}^{h_{22}} - 5M$.
Although Ref.~\cite{Dhani:2020nik} demonstrated the mirror-mode model starting $10M$ before the peak in the $h_{22}$ strain, we adopt a more conservative choice of $5M$.
The results are shown in Fig.~\ref{aligned_spin_epsilon_hist} using a dashed line. 
The addition of mirror modes gives a small improvement in the mismatch, but this is to be expected with the increased number of parameters.
However, the $\epsilon$ histogram shows that the overall performance of the mirror-mode model is comparable to that of the $N=7$ overtone model, despite the use of an earlier start time.


\section{Misaligned-spin systems}\label{misaligned-spin-section}

The analyses in Section~\ref{aligned-spin-section}, and analyses in previous studies,
% ~\cite{Dorband:2006gg, Buonanno:2006ui, Berti:2007fi, Kamaretsos:2011um, Kamaretsos:2012bs, London:2014cma, Baibhav:2017jhs, Giesler:2019uxc, Bhagwat:2019dtm, Ota:2019bzl, JimenezForteza:2020cve, Cook:2020otn, Dhani:2020nik, Mourier:2020mwa, Forteza:2021wfq, MaganaZertuche:2021syq},
was limited to BBH systems with component spins that are aligned with the orbital angular momentum, $\vb*{L}$.
This is a potentially serious limitation as misaligned spins are expected to be a generic feature of astrophysical BBHs.
Misaligned spins generally lead to precession of the orbital plane during the inspiral phase of the evolution and a richer phenomenology in the GW signals~\cite{Apostolatos:1994mx}.
There is strong evidence for precession in the GW events observed so far when looking at the population level~\cite{LIGOScientific:2020kqk, LIGOScientific:2021psn}, and tentative evidence in individual events. 
This includes high-mass event GW190521~\cite{LIGOScientific:2020iuh, LIGOScientific:2020ufj}, the asymmetric-mass event GW190412~\cite{LIGOScientific:2020stg}, and recently there have been claims of precession in the GWTC-3 event GW200129\_065458~\cite{Hannam:2021pit} (but we note that there are potential data-quality issues associated with this event~\cite{Payne:2022spz}).
In this section we investigate the effect of precession on the modelling of the ringdown by repeating analyses like those in Section~\ref{aligned-spin-section}, but now on precessing NR simulations.

As already mentioned, the remnant BH will not have a spin vector aligned with the $z$-axis in the NR frame if it has undergone precession. 
The ringdown models we have been using are only valid in the frame with the remnant BH spin vector aligned with the $z$-axis.
Therefore, before applying the ringdown models to precessing NR data, we need to rotate the NR data into the suitable frame. 

The direction from which a GW source is viewed affects the observed signal 
(e.g.\ you see circularly/linearly polarised GWs with a larger/smaller amplitude when viewing parallel/perpendicular to $\vb*{L}$).
These differences in the GW signals also manifest themselves at the level of individual spherical-harmonic modes as amplitude modulations. 
The frame in which the expansion (Eq.~\ref{ch2:eq:spherical_expansion}) is performed affects the values of the spherical-harmonic modes. 
The idea is to re-expand the NR data in the ``ringdown frame'' as follows:
\begin{equation}\label{hprimedecomp}
    h'(t,\Omega') = \sum_{\ell = 2}^\infty \sum_{m = -\ell}^\ell h'_{\ell m}(t) {}_{-2}Y_{\ell m}(\Omega'),
\end{equation}
where the prime on $\Omega$ indicates we are using new coordinates where the remnant spin is aligned with the $z$-axis.
The coefficients in this expansion, $h'_{\ell m}$, are what we now fit our ringdown models to. 
In particular, we will focus on modelling the $\ell = m = 2$ spherical harmonic mode in the ringdown frame, $h'_{22}$.

For aligned-spin systems, the $h_{2\pm2}$ are usually the dominant modes in the sum in Eq.~\ref{ch2:eq:spherical_expansion}. This is related to the fact that the GW signal amplitude is largest when viewed along the direction of the orbital angular momentum: $\vb*{L}$ or $-\vb*{L}$. For misaligned-spin systems undergoing precession, other modes become important. This in turn is related to the constantly changing direction of the orbital angular momentum, $\vb*{L}(t)$.
Changing into the non-inertial, coprecessing frame in which $\vb*{L}$ always points along the $z$-direction has been found to account for most precessional effects and makes the precessing waveform remarkably similar to a non-precessing one.
This transformation into the coprecessing frame has been successfully used to help model the full inspiral-merger-ringdown waveforms for precessing systems~\cite{Schmidt:2010it, Schmidt:2012rh} in the context of phenomenological~\cite{Hannam:2013oca, Khan:2018fmp, Pratten:2020ceb}, effective-one-body~\cite{Pan:2013rra, Ossokine:2020kjp} and NR surrogate~\cite{Blackman:2017dfb, Blackman:2017pcm, Varma:2019csw} modelling.
There is an analogy with the approach taken here for the modelling of the ringdown. In order to simplify the task, we choose to work in a frame adapted to final spin angular momentum of the remnant, $\vb*{\chi}_f$.
Although, in our case, the rotation required to get into this frame is not time dependent and our chosen frame is therefore inertial.

To obtain an expression for $h'_{\ell m}$ in terms of the $h_{\ell m}$ provided by NR, we invert Eq.~\ref{hprimedecomp} to obtain
\begin{equation}
    h'_{\ell m}(t) = \int_{\Omega'} h'(t,\Omega') ~ {}_{-2}Y_{\ell m}^*(\Omega') ~ \dd{\Omega'}.
\end{equation}
For spin-weighted fields there is a subtlety that $h'(t,\Omega') \neq h(t, \Omega)$, but instead
\begin{equation}
    h'(t,\Omega') = h(t,\Omega) ~ e^{-is\gamma}
\end{equation}
where $s$ is the spin weight (s = $-2$ in our case), and $\gamma$ is some angle (for example, a rotation about the third degree of freedom we have when going into a new coordinate system). 
Substituting into our expression for $h'_{\ell m}$ we get
\begin{equation}
    h'_{\ell m}(t) = \int_{\Omega'} h(t,\Omega) ~ e^{2i\gamma} ~ {}_{-2}Y_{\ell m}^*(\Omega') ~ \dd{\Omega'}.
\end{equation}
Under a rotation, a spin-weighted spherical harmonic transforms into a linear combination of spin-weighted spherical harmonics of the same $\ell$ but different $m$~\cite{Boyle:2013nka}:
\begin{equation}
    {}_{-2}Y_{\ell m}(\Omega') = \sum_{m' = -\ell}^{\ell} \qty[ D^{\ell}_{m' m} (\mathbf{R}) ]^* {}_{-2}Y_{\ell m'}(\Omega) ~ e^{2i\gamma}
\end{equation}
where $D^{\ell}_{m' m} (\mathbf{R})$ is the Wigner $D$ matrix for a rotation $\mathbf{R}$ of the basis. 
Substituting into the expression for $h'_{\ell m}$ we get
\begin{align}\label{Yrotation_wignerD}
    h'_{\ell m}(t) &= \int_{\Omega'} h(t,\Omega) ~ e^{2i\gamma} ~ \qty[ ~ \sum_{m' = -\ell}^{\ell} \qty[ D^{\ell}_{m' m} (\mathbf{R}) ]^* {}_{-2}Y_{\ell m'}(\Omega) ~ e^{2i\gamma} ~ ]^* ~ \dd{\Omega'} \nonumber \\
    &= \sum_{m' = -\ell}^{\ell} D^{\ell}_{m' m} (\mathbf{R}) ~ \int_{\Omega'} h(t,\Omega) ~ {}_{-2}Y^*_{\ell m'}(\Omega) ~ \dd{\Omega'} \nonumber \\
    &= \sum_{m' = -\ell}^{\ell} D^{\ell}_{m' m} (\mathbf{R}) ~ h_{\ell m'}(t),
\end{align}
where the $\gamma$ terms have cancelled (note that it doesn't matter the final integral is over the primed coordinates, as we're integrating over the full sphere).

This tells us how we can express the more natural $h'_{\ell m}$ modes (with coordinates suited to the remnant BH) in terms of the SXS $h_{\ell m}$ modes. Each $h'_{\ell m}$ mode is a superposition of $h_{\ell m}$ modes with the same $\ell$ but different $m$.
The rotation $\mathbf{R}$ can be obtained from the direction of the remnant BH spin vector (which is provided as metadata for all SXS simulations). Specifically, $\mathbf{R}$ is any rotation that maps the $z$-axis onto the final spin vector.

We now apply the overtone model to the ringdown of an example precessing simulation SXS:BBH:1856~\cite{Varma:2019csw}. 
This simulation (at the reference time) has a mass ratio of $q=2.78$ and dimensionless spins $\vb*{\chi}_1=(0.18, -0.54, -0.45)$ and $\vb*{\chi}_2=(-0.12, -0.31, -0.031)$ on the heavier and lighter components respectively. 
This simulation was chosen because it exhibits strong precession effects visible as amplitude modulations in $h_{22}(t)$. The final spin vector is $\vb*{\chi}_f=(-0.03,-0.19,0.42)$ and the rotated mode $h'_{22}(t)$ was computed using Eq.~\ref{Yrotation_wignerD}.

\begin{figure}[t]
    \centering
    \includegraphics[width=0.6\columnwidth]{ModellingTheRingdownFromPrecessingBlackHoleBinaries/tEdot-t22_hist.pdf}
    \caption[Differences in the times of peak strain amplitude and peak gravitational-wave luminosity]{  
    Histogram of the differences between the two possible start times considered in Section~\ref{misaligned-spin-section}: the peak of the (rotated) strain mode $h'_{22}$, and the peak of the GW energy flux.
    The normalised distribution of the differences between these times is shown both for the 85 aligned-spin systems used in Section~\ref{aligned-spin-section} and for the 252 precessing simulations considered in Section~\ref{misaligned-spin-section}. 
    The peak of the flux almost always occurs later than the peak of the strain, making this a more conservative choice for the ringdown start time. 
    We note that there is a much greater variation amongst the population with misaligned spins.
    }
    \label{tEdot-t22}
\end{figure}

The overtone model in Eq.~\ref{GieslerRD} was fitted to the rotated $\ell=m=2$ mode of the strain, $h'_{22}(t)$, in the same way as was done for the aligned-spin systems in Section~\ref{aligned-spin-section}.
There is some ambiguity in how to choose the ringdown start time $t_0$ in a way that gives as fair a comparison as possible with the non-precessing case. 
We cannot use the peak of the $h_{22}(t)$ strain, as was done in Section~\ref{aligned-spin-section}, as this mode suffers from precession induced amplitude modulations. 
One option would be to use instead the peak of the rotated strain mode $h'_{22}(t)$.
However, we find that using the peak of the GW energy flux, $\dot{E}$, (which can be computed from the modes in either frame, see Eq.~3.8 in Ref.~\cite{Ruiz:2007yx}) gives more consistent results between simulations. For example, some precessing configurations show a peak in the (rotated) strain relatively early in the signal, leading to poorer fits.
The use of the peak in the energy flux is also a conservative choice in the sense that $t_{\mathrm{peak}}^{\dot{E}} > t_{\mathrm{peak}}^{h_{22}}$ in almost all cases (see Fig.~\ref{tEdot-t22}).

\begin{figure}[t]
    \centering
    \includegraphics[width=0.6\columnwidth]{ModellingTheRingdownFromPrecessingBlackHoleBinaries/1856_mismatch_vs_t0_with_error_edit.pdf}
    \caption[Mismatch for the overtone model fitted to SXS:BBH:1856]{
    Mismatch as a function of ringdown start time for the overtone model (Eq.~\ref{GieslerRD}) when fitting to the rotated $h'_{22}$ mode of the NR simulation SXS:BBH:1856.
    When using $N=7$ overtones, the lowest mismatch is achieved starting slightly ($\sim 10M$) before the peak in the GW energy flux.
    However, the minimum mismatch is $\sim 100$ times larger than that obtained for the example spin-aligned system SXS:BBH:0305 in Fig.~\ref{305_mismatch_vs_t0}. 
    The dashed grey curve shows the estimate of the error in the underlying NR simulation and is described in Section~\ref{NR_error_appendix}.
    }
    \label{1856_mismatch_vs_t0}
\end{figure}

Fig.~\ref{1856_mismatch_vs_t0} shows how the mismatch varies for SXS:BBH:1856 as a function of ringdown start time, for different values of $N$ in the overtone model Eq.~\ref{GieslerRD}.
With each additional overtone, the minimum mismatch is reached at an earlier time (the same behaviour as was seen in Fig.~\ref{305_mismatch_vs_t0}).
However, the values of the minimum mismatch are a factor of $\sim 100$ larger than those obtained in the aligned-spin case. 

The $N=7$ model achieves a minimum mismatch $\sim 10M$ before the time of peak GW energy flux. This is fairly typical behaviour among the misaligned-spin SXS simulations considered.
However, we note there is a much greater variety of possible behaviours for misaligned-spin systems than for the aligned-spin population. 
The greater variation amongst the misaligned-spin population has already been hinted at in Fig.~\ref{tEdot-t22}, where the spread of start times is greater than in the aligned-spin cases.

\begin{figure}[t]
    \centering
    \includegraphics[width=0.6\columnwidth]{ModellingTheRingdownFromPrecessingBlackHoleBinaries/1856_epsilon_grid_alt.pdf}
    \caption[Recovery of SXS:BBH:1856 remnant properties using the overtone model]{
    Recovery of the SXS:BBH:1856 remnant properties when fitting the overtone model (Eq.~\ref{GieslerRD}) to the rotated $h'_{22}$ mode from the time of its peak energy flux.
    The heat map shows the mismatch for the fit with $N=7$, while the crosses show the locations of the minima in the mismatch for fits performed with different values of $N$.
    The mismatch shows a much broader and less deep minimum than that seen for the spin-aligned system SXS:BBH:0305 in Fig.~\ref{305_epsilon_grid}.
    The minimum in the mismatch is also biased away from the true remnant parameters with $\epsilon=0.025$ for the $N=7$ fit.
    The sequence of crosses for fits with different values of $N$ also do not show the same convergent trend towards the true remnant parameters that was observed for SXS:BBH:0305 in Fig.~\ref{305_epsilon_grid}.
    }
    \label{1856_epsilon_grid}
\end{figure}

\begin{figure}[t]
    \centering
    \includegraphics[width=\columnwidth]{ModellingTheRingdownFromPrecessingBlackHoleBinaries/misaligned_spin_epsilon_M_hist.pdf}
    \caption[Remnant-property errors and mismatches for the overtone model fitted to misaligned-spin SXS simulations]{
    \emph{Left:} histograms of the mass-spin remnant error $\epsilon$ from an overtone model fit to the rotated $h'_{22}$ modes of 252 misaligned-spin SXS simulations for several different overtone numbers $N$. 
    \emph{Right:} histograms of the mismatch from a fit with the true remnant mass and spin parameters, with the same overtone models and SXS simulations as in the left histogram.
    The solid histograms show results from fits performed starting at the peak of the energy flux with $N$ overtones of the fundamental $\ell = m = 2$ mode.
    The red dashed line shows results from a $N=7$ model that also includes mirror modes and was fitted with a ringdown starting $5M$ before the peak in the energy flux.
    These histograms should be compared with those in Fig.~\ref{aligned_spin_epsilon_hist}; we note that the effect of precession is to (i) significantly broaden the histograms (i.e.\ the quality of the fit is much more varied) and (ii) to significantly degrade the quality of the fit for some systems.
    }
    \label{misaligned_spin_epsilon_hist}
\end{figure}

The heat map of Fig.~\ref{1856_epsilon_grid} shows the mismatch as a function of the remnant BH properties, for the $N=7$ model.
The coloured crosses indicate the mismatch minimum for different values of $N$. 
Comparing with Fig.~\ref{305_epsilon_grid}, we see the mismatch minimum is less pronounced than the aligned-spin case, which is probably contributing to the larger value of $\epsilon$ (for $N=7$ we find $\epsilon = 0.025$, which is much larger than the estimated numerical error $\epsilon_{\mathrm{NR}} = 8.6 \cross 10^{-5}$). 
In addition, the convergent behaviour with increasing $N$ is not present. For $N \geq 1$, all mismatch minima appear randomly distributed around the true remnant properties.
If we reproduce this figure with a earlier start time of $t_0 = t_{\mathrm{peak}}^{\dot{E}} - 10M$ (motivated by the time of minimum mismatch for $N=7$ in Fig.~\ref{1856_mismatch_vs_t0}), the heat map remains unchanged, and the value of $\epsilon$ recovered for $N=7$ is not significantly improved ($\epsilon = 0.013$). The earlier start time does cause the value of $\epsilon$ for $N \leq 3$ to increase significantly, which may be expected as we are now using a start time before those models reach a mismatch minimum. 

Following Section~\ref{aligned-spin-section}, we now extend this analysis to a wider selection of SXS simulations to investigate the robustness (or lack thereof) of this behaviour. We consider only misaligned-spin simulations, chosen such that the angle between the initial spins, $\chi_{\theta}$, satisfies $\pi/16 < \chi_{\theta} < 15\pi/16$. We again require initial spin magnitudes $\chi_{1,2} < 0.8$ and mass ratios $q<8$. The 252 simulations were chosen in the ID range SXS:BBH:1643 to SXS:BBH:1899, as these cover a range of mass ratios and initial spin configurations.

The results are shown in Fig.~\ref{misaligned_spin_epsilon_hist}. 
When compared to the $N=0$ model, the addition of three overtones reduces the remnant error and mismatch. 
However, the inclusion of additional overtones does not change the $\epsilon$ histogram, and produces only a minor reduction in the mismatch.
Comparing the $N=7$ histogram for $\epsilon$ to that found in Fig.~\ref{aligned_spin_epsilon_hist}, we see that, on average, $\epsilon$ increases by a factor of $\sim 10$ and, in the worst cases, by a factor of $\sim 20$ (however, the overtone model does still perform similarly well for a small fraction of simulations). 
The histograms for $\epsilon$ reflect the behaviour of Fig.~\ref{1856_epsilon_grid}, where models with $N \geq 1$ don't show systematic improvements. 
It would be interesting to investigate whether the binary parameters correlate with $\epsilon$, and if certain binary configurations are responsible for the largest remnant errors. We have performed preliminary studies which reveal no clear correlations of $\epsilon$ with either the amount of precession (quantified via $\chi_p$~\cite{Schmidt:2014iyl}) or the recoil velocity. We defer a more detailed study of this question to future work.

It was checked if using an earlier start time of $t_{\mathrm{peak}}^{\dot{E}} - 10M$ changed the recovered distribution on $\epsilon$. 
This choice was motivated by the location of the mismatch minimum typically seen for misaligned-spin simulations (e.g.\ see Fig.~\ref{1856_mismatch_vs_t0}). 
It was found the $N=7$ model results did not significantly change. 
However, the $N=3$ and $N=0$ models performed worse.
Finally, we also note that all of the histograms are wider than those in Fig.~\ref{aligned_spin_epsilon_hist}. 
This may be due to mirror modes and/or higher harmonics having a more important role for precessing systems (see below). 


\subsection{Mirror modes} \label{subsec:misaligned_mirror_modes}

We repeat the population analysis with the $N=7$ mirror-mode model, again shifting the ringdown start time back by $5M$ to make a clear comparison to Fig.~\ref{aligned_spin_epsilon_hist}. 
The results are shown by the red dashed lines in Fig.~\ref{misaligned_spin_epsilon_hist}. The histogram for $\epsilon$ doesn't reach values as high as the overtone model (with worst-case values of $\epsilon \sim 0.04$ compared to the overtone model's $\sim 0.2$), but otherwise has a broadly similar distribution.
However, there is a significant improvement on the recovered mismatch values. This is expected because of the large number of parameters. And, as discussed, this alone isn't enough to say the model is successful.

Inspecting individual simulations, we see that the inclusion of mirror modes can make the mismatch minima in the mass-spin plane more pronounced (advantageous, as it reduces uncertainty on $\epsilon$). For example, Figs.~\ref{1856_mirror_mode_mismatch_vs_t0} and \ref{1856_mirror_mode_epsilon_grid} show how mirror mode fits perform for SXS:BBH:1856. 
We see significantly smaller mismatches, and a stronger mismatch peak around the true remnant properties. However, on average this does not translate to smaller values of $\epsilon$ for the $N=7$ model (as can be seen from the red dashed histogram in Fig.~\ref{misaligned_spin_epsilon_hist}). For SXS:BBH:1856, the $N=7$ model gives $\epsilon = 0.014$, which is not a significant improvement.

\begin{figure}[t]
    \centering
    \includegraphics[width=0.6\columnwidth]{ModellingTheRingdownFromPrecessingBlackHoleBinaries/1856_mismatch_vs_t0_mirror_modes_with_error_edit.pdf}
    \caption[Mismatch for the mirror-mode model fitted to SXS:BBH:1856]{
    Mismatch as a function of ringdown start time for the mirror-mode model (Eq.~\ref{ch2:eq:mirror_model}) when fitting to the rotated $h'_{22}$ mode of the NR simulation SXS:BBH:1856.
    Comparing with Fig.~\ref{1856_mismatch_vs_t0}, the locations of the mismatch minima are roughly unchanged in time, but the inclusion of mirror modes reduces the mismatch to values similar to those in Fig.~\ref{305_mismatch_vs_t0}. The dashed grey curve shows the estimate of the error in the underlying NR simulation and is described in Section~\ref{NR_error_appendix}.
    }
    \label{1856_mirror_mode_mismatch_vs_t0}
\end{figure}

\begin{figure}[t]
    \centering
    \includegraphics[width=0.6\columnwidth]{ModellingTheRingdownFromPrecessingBlackHoleBinaries/1856_epsilon_grid_mirror_modes_m5.pdf}
    \caption[Recovery of SXS:BBH:1856 remnant properties using the mirror-mode model]{ 
    Recovery of the SXS:BBH:1856 remnant properties when fitting the mirror-mode model (Eq.~\ref{ch2:eq:mirror_model}) to the rotated $h'_{22}$ mode from $5M$ before the time of its peak energy flux.
    The heat map shows the mismatch for the fit with $N=7$, while the crosses show the locations of the minima in the mismatch for fits performed with different values of $N$ ($N=0$ lies outside the figure, and is not included for clarity). 
    Comparing with Fig.~\ref{1856_epsilon_grid}, the inclusion of mirror modes sharpens the mismatch peak and achieves smaller mismatch values. 
    However, when averaged across the population of precessing simulations, the mirror-mode model doesn't give smaller values for the remnant error (see dashed curve in Fig.~\ref{misaligned_spin_epsilon_hist}). 
    Here, $\epsilon = 0.014$ for the $N=7$ model.
    }
    \label{1856_mirror_mode_epsilon_grid}
\end{figure}

To investigate whether the choice of ringdown start time could be contributing to the wider histograms seen in Fig.~\ref{misaligned_spin_epsilon_hist}, the behaviour of the mismatch heat maps (e.g.\ Figs.~\ref{305_epsilon_grid}, \ref{1856_epsilon_grid}, \ref{1856_mirror_mode_epsilon_grid}) with varying start time was explored for selected SXS simulations. 
Animations of ringdown fits with varying start time can be found at Ref.~\cite{finch_eliot_2021_4538194}.
For the aligned-spin simulation SXS:BBH:0305, we see that the location of the mismatch minimum in the mass-spin plane settles on the true remnant properties for a sufficiently late choice of the start time ($t_0 \geq t_{\mathrm{peak}}^{h_{22}}$ for the $N=7$ overtone model). 
In addition, the mismatch minimum stays centred on the true remnant properties until numerical noise takes over.
For earlier choices of the start time, the $N=7$ overtone model gives biased values for the final mass and spin, see Ref.~\cite{finch_eliot_2021_4538194}.
Applying the $N=7$ overtone model to the misaligned-spin simulation SXS:BBH:1856, we see that the location of the mismatch minimum moves around the mass-spin plane as start time is varied. Even at late times, it never settles on the true remnant properties.
The inclusion of mirror modes, as seen in Fig.~\ref{1856_mirror_mode_epsilon_grid}, narrows the mismatch minimum. The movement of the mismatch minimum around the mass-spin plane is reduced as well, however it still doesn't settle on the location of the true remnant properties.
This behaviour may explain some of the observed widening of the histograms, and perhaps hints something is missing from the ringdown model.


\subsection{Higher harmonics}\label{kitchen-sink}

As demonstrated by Fig.~\ref{misaligned_spin_epsilon_hist}, %(and also Fig.~\ref{misaligned_spin_epsilon_hist_m10} in appendix \ref{appendix_a})
the overtone and mirror-mode models considered so far achieve median values for the remnant error $\epsilon \sim 0.01$, a factor of 10 or more higher than the aligned-spin fits of Fig.~\ref{aligned_spin_epsilon_hist}. In addition, the spread of $\epsilon$ values recovered is significantly larger, leading to values of $\epsilon$ up to $\sim 0.1$. 
These models perform significantly worse in some cases for precessing systems than aligned-spin systems.

We now investigate whether the inclusion of higher harmonics (that is, QNMs with $\ell > 2$) can improve the fits to $h'_{22}(t)$.
These higher harmonics were neglected by both the overtone (Eq.~\ref{GieslerRD}) and mirror-mode (Eq.~\ref{ch2:eq:mirror_model}) models.
However, mode mixing occurs as a consequence of the different angular basis functions used in the waveform decompositions in Eqs.~\ref{ch2:eq:spherical_expansion} and \ref{ch2:eq:spheroidal_expansion} and the fact that these basis functions are not mutually orthogonal~\cite{Berti:2014fga}.
The amount of mode mixing between the spherical mode ${}_{-2}Y_{\ell m}$ and the spheroidal mode ${}_{-2}S_{\ell m n}$ is determined by the remnant spin $\chi_f$ and the QNM frequency. This can be quantified by how much these functions fail to be orthogonal; i.e.\ by the spherical-spheroidal mixing coefficients (Eq.~\ref{ch2:eq:mu}).
A translational offset between the NR and ringdown frames (e.g.\ due to a kick) can also lead to mixing between $m$-modes~\cite{Boyle:2015nqa}; this effect is neglected here.
To include the contribution from higher harmonics, we define a new ringdown model for the spherical-harmonic modes which now allows for a sum over different $\ell$:
\begin{equation}\label{full_ringdown}
    h_{\ell m}^{N,\,L,\, {\rm mirror}}(t) = \sum_{n=0}^N \sum_{l=2}^{L} \qty[ C_{l m n} e^{-i \omega_{l m n}(t-t_0)} + C'_{l m n} e^{i \omega^*_{l m n}(t-t_0)} ]\quad \textrm{for} \quad t \geq t_0.
\end{equation}
This \emph{harmonic} model contains all of the allowed QNMs in Eq.~\ref{ch2:eq:spheroidal_expansion}, including the mirror modes and the overtones.
This comes at the expense of a large number of free parameters; there are $4(N+1)(L-\ell+1)$ in the complex amplitudes, plus the two remnant parameters $M_f,\; \chi_f$ that determine the complex QNM frequencies.

\begin{figure}[t]
    \centering
    \includegraphics[width=\columnwidth]{ModellingTheRingdownFromPrecessingBlackHoleBinaries/misaligned_spin_epsilon_M_hist_harmonics.pdf}
    \caption[Remnant-property errors and mismatches for the harmonic model fitted to misaligned-spin SXS simulations]{
    \emph{Left:} histograms of the mass-spin remnant error $\epsilon$ from harmonic model fits (Eq.~\ref{full_ringdown}) to the same 252 misaligned-spin SXS simulations used in Fig.~\ref{misaligned_spin_epsilon_hist}. 
    Shown (in dashed lines) are the $L=3$ and $L=4$ models with $N=7$ overtones and mirror modes. 
    Shown in green is the overtone model with $N=7$ and $L=2$ (no mirror modes); this is the same as the green histogram in Fig.~\ref{misaligned_spin_epsilon_hist} and is included here to aid comparison. 
    We also show for comparison the $L=3$ model without mirror modes (grey histogram).
    \emph{Right:} histograms of the mismatch from a fit with the true remnant mass and spin parameters, with the same models and SXS simulations as in the left histogram. 
    The harmonic model with mirror modes, which includes many free parameters, achieves small mismatches but without significant improvement in the remnant error. 
    We note that the inclusion of $L=4$ does not bring any additional improvements over $L=3$.
    }
    \label{misaligned_spin_epsilon_hist_harmonics}
\end{figure}

Multiple variations of this harmonic model were trialled (varying $N$, $L$, and the inclusion of mirror modes) on the same population of 252 misaligned-spin SXS simulations.
Fig.~\ref{misaligned_spin_epsilon_hist_harmonics} shows the chosen subset of results.
All results shown include seven overtones, and include $L=2$ (16 free parameters in the complex amplitudes), $L=3$ without and with mirror modes (32 and 64 free parameters respectively), and $L=4$ with mirror modes (96 free parameters).
As before, we fit to the rotated $h'_{22}(t)$ spherical harmonic mode.
To make a clear comparison with the previous models, we again use a ringdown start time corresponding to the peak of the GW energy flux.

The inclusion of higher harmonics with the mirror modes drastically improves the mismatch.
A small mismatch is not surprising for a model with so many free parameters, and in some of these cases we are likely pushing beyond the limits of accuracy of the NR simulations. See Section~\ref{NR_error_appendix} for a discussion of the numerical errors.
There is a modest reduction in $\epsilon$ for some systems, and in particular we see less systems with $\epsilon > 0.01$ (at least for $L=3$). This hints at the importance of higher harmonics in some precessing systems. Despite this, we still see worst-case values of $\epsilon \sim 0.04$.
% We also note that 

\section{Surrogates}\label{surrogate-section}

NR simulations are computationally expensive, and although the number of simulations available in public catalogs is growing they are still limited in their parameter space coverage. 
NR surrogate models~\cite{Blackman:2015pia, Blackman:2017pcm, Varma:2019csw, Varma:2018mmi} would appear to be an attractive alternative.
These models use reduced-order and surrogate modelling techniques to extend the results of a set of NR simulations smoothly across parameter space. 
The use of surrogates could, in principle, allow us to extend the results of this chapter to include many more systems as well as allowing us to study how the excitations of the various QNMs vary during a smooth exploration of parameter space.
However, care must be taken as the surrogate modelling necessarily introduces an additional source of error into the waveforms, on top of the errors originally in the NR waveforms themselves.
 
When attempting to fit QNM ringdown models with overtones to NRSur7dq4~\cite{Varma:2019csw} waveforms, it was found that incorrect values for $M_f$ and $\chi_f$ were being recovered (particularly at high mass ratios). This being the case even for aligned-spin or non-spinning systems. Although the NRSur7dq4 waveforms do not provide the remnant properties, these can be obtained via NRSur7dq4Remnant~\cite{Varma:2019csw} (it was found the problem did not lie with the values returned by NRSur7dq4Remnant but rather with the waveform surrogate).

\begin{figure}[t]
    \centering
    \includegraphics[width=0.6\columnwidth]{ModellingTheRingdownFromPrecessingBlackHoleBinaries/surrogate_epsilon_and_mass_ratio.pdf}
    \caption[Comparison of remnant-property errors from two surrogate models and a selection of SXS simulations]{
    Comparison of the remnant error $\epsilon$ from two surrogate models and a selection of SXS simulations. All are zero initial spin. The fits were performed on the $h_{22}$ mode with the $N=7$ overtone model, Eq.~\ref{GieslerRD}, starting from the time of peak strain. The labels on each cross correspond to the SXS ID. The dashed line indicates where we are outside the training range of NRSur7dq4.
    }
    \label{surrogate_epsilon_vs_q}
\end{figure} 

To investigate the performance of NRSur7dq4 ringdown waveforms, a series of simulations with zero initial spin with increasing mass ratio $q$ from 1 to 6 were used. 
The $N=7$ overtone model (Eq.~\ref{GieslerRD}) was fitted to the $h_{22}(t)$ mode of each starting from the peak strain (as in Section~\ref{aligned-spin-section})
and the remnant error $\epsilon$ (Eq.~\ref{eq:epsilon}) was calculated for each.
The results are shown in Fig.~\ref{surrogate_epsilon_vs_q}, along with the results for similar fits performed directly on 11 zero-spin SXS simulations at discrete values of the mass ratio. 
The fits to the NRSur7dq4 surrogate produce values for $\epsilon$ that are 1-2 orders of magnitude higher than for the equivalent SXS simulations. 
Also shown are the results from a similar analysis with the more restrictive aligned-spin surrogate NRHybSur3dq8~\cite{Varma:2018mmi, Varma:2018aht}; this was found to be in close agreement with the SXS simulations.

Residuals and mismatches can also be computed between surrogate and NR waveforms (taking care to align the waveforms in both time and phase).
For SXS:BBH:0168, the $q=3$, zero-spin simulation used in Fig.~\ref{surrogate_epsilon_vs_q}, we find $\sim 2\%$ residuals in the ringdown when comparing to the NRSur7dq4 surrogate with the same parameters. 
This leads to a mismatch between the surrogate and SXS:BBH:0168 of $3.7 \times 10^{-4}$, when integrating over the ringdown. For comparison, we have a $\sim 10^{-6}$ mismatch between the ringdown model Eq.~\eqref{GieslerRD} and the SXS simulation. The relatively high mismatch between the NRSur7dq4 and SXS waveforms translates to the relatively high values of $\epsilon$ seen in Fig.~\ref{surrogate_epsilon_vs_q}. 

It seems that the high-dimensional precessing surrogate NRsur7dq4 is not yet sufficiently accurate in the ringdown for the purposes of QNM overtone studies that, by virtue of their large number of free parameters, fit the ringdown with very small mismatches. 
By contrast, the lower-dimensional aligned-spin surrogate NRHybSur3dq8 does appear to be sufficiently accurate for such studies.


\section{Numerical relativity errors}\label{NR_error_appendix}

It is important to remember the finite accuracy of the NR simulations used in ringdown studies.
This is particularly true when using models with many QNMs which, by their very nature, use a large number of free parameters and regularly achieve very small ($\sim 10^{-6}$) mismatches.
If care is not taken, we risk fitting our models to the numerical noise. 
In this section we describe the numerical checks performed on the two individual simulations used in this chapter: SXS:BBH:0305, and SXS:BBH:1856. %, and the three simulations shown in Fig.~\ref{misaligned_spin_variation}.
In each case the numerical errors were estimated by comparing results obtained using data from the two highest resolutions (levels) available in the SXS catalog. 

First, we quantify the numerical error in the mismatch.
This was done by calculating the mismatch between the two highest NR resolutions from a time $t_0$ to a time $T = t_0 + 100M$, for a range of $t_0$. For each start time, we optimally align the two waveforms in time (taking the absolute value in the mismatch automatically optimises the mismatch over phase). The alignment in time can be done by matching the time of peak strain, for example, or by numerically rolling the waveform to find the optimal time shift for each mismatch calculation.
The results are shown by the grey dashed lines in the mismatch vs start time plots in Figs.~\ref{305_mismatch_vs_t0}, \ref{1856_mismatch_vs_t0} (duplicated in Fig.~\ref{1856_mirror_mode_mismatch_vs_t0}). % and the 3 panels of Fig.~\ref{misaligned_spin_variation}.
Generally, we see numerical error estimates at or below the model mismatches, particularly at late times, indicating that we are not fitting to the numerical noise.
The main exception is Fig.~\ref{1856_mirror_mode_mismatch_vs_t0} where the mirror-mode model is applied to a precessing system. This is expected; precessing NR simulations, and those with high mass ratios are generally expected to have larger numerical errors. Additionally, the mirror mode and harmonic models have the highest numbers of free parameters making them more likely to reach the accuracy of the NR simulation. 

Second, we investigate the numerical error on the remnant mass and spin.
We quantify the numerical error with $\epsilon_{\mathrm{NR}}$, the Euclidean distance (Eq.~\ref{eq:epsilon}) between the remnant properties reported in the two highest resolution levels of the NR simulation.
The $\epsilon_{\mathrm{NR}}$ values are reported in the previous sections. % and in the table in appendix \ref{appendix_a}.
In all cases $\epsilon_{\mathrm{NR}} < \epsilon$. 
This supports the conclusions in this chapter and indicates they are likely to be robust against numerical noise in the underlying NR simulations used.


\section{Conclusions} \label{sec:discussion}

This chapter has made a first systematic attempt at using QNMs to model the ringdown of BHs formed from BBHs with misaligned component spins in the inspiral.
Previously, for aligned-spin systems, it has been found that the ringdown can be modelled with low mismatch and low remnant errors using a model that includes overtones of the fundamental QNM~\cite{Giesler:2019uxc}. 
For seven overtones, the ringdown can be reliably modelled from the peak of the $h_{22}(t)$ strain for a range of SXS simulations.
Additionally, the inclusion of mirror modes can allow the ringdown to be modelled from even earlier times~\cite{Dhani:2020nik}.
In this chapter, which generalised these studies to precessing systems, we find that while QNM models can reliably achieve small mismatches, in the worst cases the remnant errors are more than a factor of 10 higher.
This is the case even when choosing to start the ringdown at the more conservative (i.e.\ later) peak in GW energy flux. 
The inclusion of higher harmonics reduces the remnant error in some cases, perhaps a sign that mode mixing in the ringdown is generally more important in precessing systems. However, in other cases, a bias remains in the recovered remnant properties.
We conclude that it is not possible to reliably model the ringdown from the peak in the flux, or indeed from the peak in the strain. 

We end by sounding a brief note of caution to any who attempt to construct a QNM model starting at or before the peak flux or strain. 
While such a model will work in some cases, it risks biased results in others. 
This risk is subtle because QNM models can give small mismatches even when they fail to adequately describe the remnant.


% \section{Overtone Model Fits to a Variety of Precessing NR Simulations}\label{appendix_a}

% \begin{figure}[h]
%     \centering
%     \includegraphics[width=\textwidth]{ModellingTheRingdownFromPrecessingBlackHoleBinaries/appendix_plot_with_error_edit.pdf}
%     \caption[Selection of results for modelling the ringdown of misaligned-spin SXS simulations using the overtone model]{ 
%     A selection of results for modelling the ringdown of precessing NR simulations from the SXS catalog \cite{Boyle:2019kee, Mroue:2013xna,sxs_catalog} using the overtone model in Eq.~\ref{GieslerRD}.
%     These plots show the results for the three systems described in the table that have been chosen to illustrate the wider range of behaviours that occur for precessing systems, from good at the top to bad at the bottom.
%     The left-hand column of plots also shows the difficulty in identifying a general start time for the ringdown as mismatch is minimised for a range of different times and sometimes there isn't even a clear first minimum.
%     }
% 	\label{misaligned_spin_variation}
% \end{figure}

% \begin{footnotesize}
% \begin{center}
% \begin{tabular}{ c|c|c|c|c|c } 
% %\hline
% $\;$SXS:BBH ID $\;$ & $\;$Figure row$\;$ & $\;$Remnant error $\epsilon$$\;$ ($\epsilon_{\mathrm{NR}}$) &  $\;$Mass ratio $q$$\;$ & Component spins $\vb*{\chi}_1$, $\vb*{\chi}_2$  & $\;$Remnant spin $\vb*{\chi}_f$$\;$ \\
% \hline
% 1677 & top & $8.1 \cross 10^{-4}$ ($1.8 \cross 10^{-4}$) & 2.64 & $(-0.06,\,0,\,0.27)$, $(-0.49,\,-0.55,\,0.06)$ & $(-0.05,\,0,\,0.68)$ \\ 
% %\hline
% 1768 & middle & $2.6 \cross 10^{-2}$ ($8.0 \cross 10^{-4}$) & 3.49 & $(0.65,\,0.03,\,0.01)$, $(-0.3,\,0.05,\,0.47)$ & $(0.31,\,-0.02,\,0.56)$ \\ 
% %\hline
% 1789 & bottom & $1.6 \cross 10^{-1}$ ($4.8 \cross 10^{-4}$) & 3.72 & $(0.46,\,0.08,\,-0.52)$, $(-0.43,\,-0.28,\,-0.17)$ & $(0.14,\,0.01,\,0.31)$ \\ 
% %\hline
% \end{tabular}
% \end{center}
% \end{footnotesize}


% \section{Overtone Model Fits to a Population of Precessing NR Systems Starting Before the Peak Flux}\label{misaligned_spin_fits_appendix}

% The analysis on the population of misaligned-spin simulations performed in section \ref{misaligned-spin-section} (results plotted in Fig.~\ref{misaligned_spin_epsilon_hist}) is repeated here using an earlier start time for the ringdown: $t_0=t^{\dot{E}}_{\rm peak}-10M$.
% This was done to check whether a poor choice of start time was responsible for some of the poor fits obtained using the overtone model in Eq.~\ref{GieslerRD}.
% The new results are plotted in Fig.~\ref{misaligned_spin_epsilon_hist_m10}.
% We find that the $N=7$ model results do not significantly change with the new start time.
% The $N=3$ and $N=0$ model results do change and generally give a worse fit with the earlier start time, as might be expected. This analysis shows that the overtone model (with or without mirror modes) cannot be reliably applied to precessing systems at early times. 

% \begin{figure*}[h]
%     \centering
%     \includegraphics[width=\columnwidth]{ModellingTheRingdownFromPrecessingBlackHoleBinaries/misaligned_spin_epsilon_M_hist_m10.pdf}
%     \caption[Remnant error and mismatches for fits to misaligned-spin SXS simulations using the overtone model starting from $10M$ before the peak of the $h_{22}$ strain]{
%     Left: histograms of the mass-spin remnant error $\epsilon$ from an overtone model fit to the rotated $h'_{22}$ mode of 252 misaligned-spin SXS simulations for several different overtone numbers $N$. 
%     Right: histograms of the mismatch from a fit with the true remnant mass and spin parameters, with the same overtone models and SXS simulations as in the left histogram.
%     %
%     These results are similar to those in Fig.~\ref{misaligned_spin_epsilon_hist} in the main text, but use a start time that is earlier by $10M$.
%     %
%     The solid histograms show results from fits performed starting $10M$ before the peak of the energy flux with $N$ overtones of the fundamental $\ell = m = 2$ mode.
%     The red dashed line shows results from a $N=7$ model that also includes mirror modes and was fitted with a ringdown starting $15M$ before the peak in the energy flux.
%     }
%     \label{misaligned_spin_epsilon_hist_m10}
% \end{figure*} 
 
% Chapter 3

\chapter{Frequency-Domain Analysis of Black-Hole Ringdowns}

\label{Chapter3}

\section{Introduction}\label{ch3:sec:introduction}

The LIGO \cite{LIGOScientific:2014pky} and Virgo \cite{VIRGO:2014yos} observatories now routinely observe gravitational-wave (GW) signals from the inspiral, merger and ringdown of compact binaries \cite{LIGOScientific:2018mvr, LIGOScientific:2020ibl}.
Most of these GW signals come from binary black holes (BHs), and those with the highest masses [say, with detector frame total masses in the range $\sim (50$ -- $500)\,M_\odot$] typically exhibit loud ringdown signals that are sometimes visible in the whitened strain data.

The ringdown is associated with the system settling down into its final, stationary state.
Within general relativity (GR), the remnant is generally assumed to be a Kerr BH which is fully described by only a mass and spin (i.e.\ the \emph{no-hair} theorem).
As the merger and ringdown proceeds, the GW amplitude decreases and the final stages of this process can be well-described as linear perturbations of a remnant Kerr BH.
Perturbation theory identifies a discrete spectrum of complex (i.e.\ damped) frequencies $\omega_{\ell m n}$ which are prominent in the ringdown \cite{Berti:2009kk}.
These oscillations, known as quasinormal modes (QNMs) occur in pairs (``regular'' and ``mirror'' modes) and are indexed by integers $\ell \geq 2$ and $-\ell \leq m \leq \ell$ (spherical harmonic indices) and $n \geq 0$ (overtone index).
Hereafter, we use the term ringdown to mean the part of the signal that can be described by a superposition of QNMs.

Recent theoretical studies using catalogs of numerical relativity binary BH simulations suggest the ringdown typically starts early in the merger process, i.e.\ at or even slightly before the time of peak strain amplitude. 
This is only possible if the ringdown modeling includes overtones ($n \geq 1$) \cite{Giesler:2019uxc, JimenezForteza:2020cve, Forteza:2021wfq}, and possibly a combination of mirror modes and/or higher harmonics ($\ell\geq 3$) \cite{Cook:2020otn, Dhani:2020nik, Finch:2021iip}.
This early start time is good for the prospects of observing QNMs because it means the signal amplitude is still large when the ringdown starts and consequently the signal-to-noise ratio (SNR) in the ringdown is large.
Ref.~\cite{LIGOScientific:2020tif} was able to identify a QNM in 17 of the binary BHs observed so far, and furthermore found evidence for an overtone in two cases (namely GW150914 \cite{LIGOScientific:2016aoc} and GW190521\_074359).
We note that Ref.~\cite{LIGOScientific:2020tif} found no strong evidence for harmonics or overtones in the heaviest source, GW190521 \cite{LIGOScientific:2020iuh}.

Once QNMs have been correctly identified in an observed signal, they provide an exciting opportunity for testing GR, the Kerr metric hypothesis, and the no-hair theorem. 
The idea of using QNM frequencies for such tests, sometimes referred to as \emph{BH spectroscopy}, predates the detection of GWs \cite{Dreyer:2003bv, Berti:2005ys, Berti:2007zu, Berti:2016lat}.
Therefore, experimental QNM tests of GR started immediately with the first GW observation; Ref.~\cite{LIGOScientific:2016lio} found that the data following the peak of GW150914 was consistent with the least-damped QNM of the expected remnant.
Subsequently, several groups reanalyzed the ringdown of GW150914 with the aim of identifying additional QNMs for use in spectroscopic tests \cite{Carullo:2019flw, Isi:2019aib, Brito:2018rfr}. 
With the second GW catalog, similar analyses are now routinely performed on all suitable events \cite{LIGOScientific:2020tif}.
Besides BH spectroscopy, other types of QNM test are possible; for example, Ref.~\cite{Isi:2020tac} used the QNM frequencies of GW150914 to measure the horizon area of the remnant and thereby test Hawking's area theorem \cite{Hawking:1971tu}.
The applications mentioned so far use only the QNM frequencies; however, the excitation amplitudes and phases of the QNMs also carry useful information about the progenitor binary (see, for example, Refs.~\cite{Hughes:2019zmt, Berti:2006wq}). 

Unfortunately, QNMs are difficult to work with both from a data analysis and a theoretical perspective. 
The start time is uncertain; even with clean, noise-free numerical simulations an unambiguous determination of the ringdown start time, $t_0$, is impossible (see, for example, Ref.~\cite{Thrane:2017lqn}). 
The start of the ringdown is also abrupt in the time domain, which makes it non-local in the frequency domain and is the source of spectral leakage problems. 
Theoretically, the QNM content is uncertain; \emph{a priori} it is not known which modes should be included in the analysis.
The mode excitations depend on the initial conditions of the system in a non-trivial way, and which QNMs are detectable is also a function of the chosen ringdown start time and the SNR.
Also, QNMs do not form a complete basis \cite{Berti:2009kk}; the late-time signal contains additional components that decay more slowly, known as \emph{tails}.
Finally, it is also known that very similar compact objects can nevertheless have completely different QNM spectra \cite{Nollert:1996rf}.

Despite the difficulties, detecting and characterizing QNMs is a key goal in GW astronomy. 
The most natural approach to deal with the abrupt ringdown start is to work in the time domain. 
This differs from other GW data analysis which is almost universally performed in the frequency domain. 
However, by working in the time domain, the data can be cut precisely at a chosen $t_0$ and an analysis performed on only the ringdown part of the signal, $t \geq t_0$, without any spectral leakage.
As the particular segment of data to be analyzed is chosen before the analysis begins, the ringdown start time (and consequently the source sky location) usually must be fixed in these analyses. 
Although, see Ref.~\cite{Carullo:2019flw} where a posterior on ringdown start time is obtained for GW150914 and Ref.~\cite{Isi:2021iql} where there is a discussion about how it would be possible, in principle, to vary the sky location.
Another drawback of working in the time domain is that the noise covariance matrix is no longer diagonal, increasing computational cost of the likelihood. 
The covariance matrix is constructed with the autocovariance function, which characterizes the noise in the time domain.
And, as discussed in Ref.~\cite{Isi:2021iql}, care must be taken when estimating the autocovariance to avoid corrupting the ringdown data. Therefore, there are additional subtleties in a time-domain analysis compared to a frequency-domain analysis.

Recently, in Ref.~\cite{Capano:2021etf}, an alternative approach to ringdown analysis was presented and applied to GW190521 where a higher harmonic was identified, apparently in contradiction with the results of Ref.~\cite{LIGOScientific:2020tif} (but there are important differences in the analyses). 
Although this alternative approach is expressed in the frequency domain, it uses a modified expression for the likelihood (involving \emph{in-painting} the data before the start of the ringdown in such a way as to remove the contribution to the likelihood) and it has been shown to be equivalent to the standard time-domain approach \cite{Isi:2021iql}.
That these two formally equivalent analyses \cite{LIGOScientific:2020tif, Capano:2021etf} can come to different conclusions regarding the QNM content of GW190521 highlights some of the difficulties that come with this type of analysis, where important choices (that can affect the result) for the ringdown start time have to be made and care must be taken with the noise estimation.

In this paper, we present a new approach to performing ringdown analyses in the frequency domain. 
We employ a flexible sum of truncated wavelets to model the inspiral-merger signal (inspired by \texttt{BayesWave} \cite{Cornish:2014kda, Cornish:2020dwh}) and QNMs to model the ringdown.
The general idea behind our approach is illustrated in Fig.~\ref{fig:demo}.
By working in the frequency domain, we can use the standard, and now very mature, GW data analysis pipelines (our analysis pipeline is built on the public \texttt{Bilby} package \cite{Ashton:2018jfp}).
Also, working in the frequency domain makes it trivial to search and marginalize over the sky position the ringdown start time, $t_0$. 
We hope this approach can complement existing time-domain analyses.

The details of our method are described in Sec.~\ref{sec:methods}, where we compare and contrast the time- and frequency-domain likelihood functions before introducing our frequency-domain approach.
In Sec.~\ref{sec:injection_study} we present the results of a series of analyses on simulated GW signals where we test the performance of our approach and compare it with time-domain methods.
We present our conclusions in Sec.~\ref{ch3:sec:discussion}.
A complete set of posterior samples from this work are made available at Ref.~\cite{finch_eliot_2021_5569759}.


\section{Methods}\label{sec:methods}

This section describes the details of the proposed frequency-domain approach to the analysis of BH ringdowns. 
Sec.~\ref{subsec:data_analysis} describes the GW likelihood function and its implementation in both the time and frequency domains.
Sec.~\ref{subsec:motivation} motivates our frequency-domain approach by describing an extreme limit in which it becomes equivalent to the standard, time-domain approach.
Finally, Sec.~\ref{subsec:model} describes the combination of truncated wavelets and QNMs that comprise our waveform model.


\subsection{Time- and Frequency-Domain Likelihoods}\label{subsec:data_analysis}

\begin{figure*}[t]
	\centering
	\includegraphics[width=0.8\columnwidth]{FrequencyDomainAnalysisofBlackHoleRingdowns/demo.pdf}
	\caption[Illustration of the frequency-domain analysis method]{ 
		This figure is intended to illustrate the idea behind our approach for analyzing the ringdown in the frequency domain.
		The gray line in the bottom panel shows the whitened and band-passed Livingston time-series data around GW190521; this loud GW signal comes from a high-mass binary BH merger and exhibits a clear ringdown.
		For the purpose of illustration, we use the simplest version of our model where a single, truncated sine-Gaussian wavelet is used to model the inspiral-merger part of the signal and the fundamental $\ell=m=2,\ n=0$ QNM is used to model the ringdown.
		The ringdown start time, $t_0$, is allowed to vary as part of a Bayesian analysis which also searches over different values of wavelet parameters, the source sky position, the QNM amplitude and phase, and over the remnant BH mass and spin. Full details of this analysis will be presented elsewhere.
		The top panel shows the maximum likelihood waveform broken down into its wavelet ($h^{\rm IM}$, orange) and ringdown ($h^{\rm R}$, purple) parts as well as into plus (solid) and cross (dashed) polarizations. The discontinuity in our model can be clearly seen in the colored lines. However, when these polarizations are combined and projected [see Eq.~(\ref{eq:projection_antenna})] onto the interferometer (black line) the result is nearly continuous; we emphasize that this continuity has not been imposed by the model but is rather ``learnt'' from the data. When the projected maximum likelihood waveform is whitened according to the detector noise curve it becomes completely continuous, this is plotted as the central blue line in the bottom panel.
		In the bottom panel we also plot the uncertainty (90\% credible region; blue shaded band) on the recovered signal.
	}
	\label{fig:demo}
\end{figure*}

Most GW data analysis is done in the frequency domain because, with the usual assumptions of stationary zero-mean Gaussian noise, the instrumental noise is fully described by the (one-sided) noise power spectral density (PSD), $S_n(f)$.
In the literature, the most commonly encountered expression for the log-likelihood in one interferometer is the integral
\begin{align} \label{eq:logL_FD_continuous}
	\log\mathcal{L}(d|\pvec{\theta}) = -2\int_{0}^{\infty}\!\mathrm{d}f\;\frac{|\tilde{d}(f)-\tilde{h}(f;\pvec{\theta})|^2}{S_n(f)} + \mathrm{norm},
\end{align}
where $d(t)$ is the observed data, and $h(t;\pvec{\theta})$ is the signal model (projected onto the interferometer) described by parameters $\pvec{\theta}$.
The normalization constant in the likelihood is unimportant for our purposes and will be dropped in all following equations.
A tilde denotes the Fourier transform of a time series.
Because the noise is uncorrelated between two well separated interferometers, the log-likelihood for a network is obtained by summing the independent contributions from each instrument.

In practice, it is necessary to work with discretely sampled time series; $d_j = d(t_j)$, where $t_j=j\delta t$ for $j=0,1,\ldots, J-1$, and where $1/\delta t$ is the sampling frequency.
The discrete Fourier transform $\tilde{d}_k = \tilde{d}(f_k)$ is sampled at (positive) frequencies $f_k=k/(J\delta t)$ for $k=0,1,\ldots,K-1$, where $K=\left \lfloor (J+2)/2\right \rfloor$.
The log-likelihood in terms of the discretely sampled frequency series is given by the following sum,
\begin{align} \label{eq:logL_FD_discrete}
	\log\mathcal{L}(d|\pvec{\theta}) = \frac{-2}{J\delta t}\sum_{k}\frac{|\tilde{d}_k-\tilde{h}_k(\pvec{\theta})|^2}{S_n(f_k)},
\end{align}
which can be compared to Eq.~(\ref{eq:logL_FD_continuous}).
The noise PSD is usually estimated from off-source data using a Welch periodogram \cite{1161901}.
The frequency-domain expression for the log-likelihood involves a single sum; the noise covariance matrix is diagonal in the frequency domain.
Although, for finite duration time series there can be small correlations between frequency bins \cite{Talbot:2021igi}.

The log-likelihood can also be expressed in the time domain via
\begin{align} \label{eq:logL_TD_discrete}
	\log\mathcal{L}(d|\pvec{\theta}) = -\frac{1}{2}\sum_{jj'}\qty[d_j-h_j(\pvec{\theta})] C^{-1}_{jj'} \qty[d_{j'}-h_{j'}(\pvec{\theta})],
\end{align}
where $C_{jj'}$ is the noise covariance matrix.
The time-domain expression for the log-likelihood involves a double sum over a dense covariance matrix which is computationally more costly to evaluate than the frequency-domain expression [$\mathcal{O}(J^2)$ as opposed to $\mathcal{O}(J)$].

Because the noise is assumed to be stationary, $C_{jj'}$ has the Toeplitz structure
\begin{align} \label{eq:Toeplitz}
	C_{jj'} = \rho_{|j-j'|},
\end{align} 
where $\rho_j$ is the noise autocovariance. 
This can also be estimated from off-source data using the following two-point expectation:
\begin{align} \label{eq:autocovariance}
	\rho_j = \frac{1}{\mathcal{J}}\sum_{j'=0}^{\mathcal{J}-1}n_{j'}n_{(j'+j)}.
\end{align}
Here, $\mathcal{J}$ is the length of some off-source data segment, which is usually chosen to be longer than the analysis data (i.e.\ $\mathcal{J} \gg J$) \cite{Isi:2021iql}.
It is also necessary to treat the ``edges'' of the data segment (i.e.\ where $j + j' > \mathcal{J}-1$) either by zero-padding or imposing periodicity: $\rho_j = \rho_{\mathcal{J}-j}$.
Although these different treatments result in an autocovariance that differs for large $j$, if $\mathcal{J}$ is sufficiently large then the autocovariance will be consistent for $j < J$ (which is what enters the calculation of the likelihood).

The two expressions for the log-likelihood are equivalent.
The noise autocovariance (which appears in the time-domain log-likelihood) is related to the PSD (in the frequency-domain log-likelihood) via a discrete Fourier transform (Wiener-Khinchin theorem), when imposing the circularity condition \cite{Isi:2021iql}:
\begin{align} \label{eq:WKtheorem}
	\frac{1}{2}S_n(f_k) = \delta t \sum_{j}\rho_j \exp\left(\frac{-2\pi ijk}{J}\right).
\end{align}
We use the inverse of Eq.~(\ref{eq:WKtheorem}) to estimate the autocovariance, which comes with the requirement that the off-source segment length used in the PSD estimate is much longer than the analysis length \cite{Isi:2021iql}.

The time-domain expression for the log-likelihood has hitherto been considered more suitable for ringdown analyses.
This is because in the time-domain expression no Fourier transform of the data or model is required, and so no periodicity has to be ensured.
The abrupt start of ringdown models means they do not satisfy this periodicity condition, which leads to spectral leakage upon Fourier transforming. 
When performing Fourier transforms of the GW data, periodicity is ensured by applying window functions to taper the data.
This makes it difficult to isolate the ringdown region of a GW signal in some data; a sharp cut at the ringdown start time would introduce a discontinuity, whereas a smooth window would either suppress the ringdown signal or risk contamination of the ringdown by including unwanted parts of the inspiral-merger (see Fig.~7 in \cite{Isi:2021iql} for an illustration of this). These problems are naturally avoided in the time domain. 
In the following sections we describe how these problems can also be overcome in a frequency-domain analysis.


\subsection{Marginalizing Over the Inspiral-Merger}\label{subsec:motivation}

We now motivate our approach to analyzing the ringdown by first discussing a special case in which it becomes formally equivalent to the standard time-domain approach. 

Consider first the case of a single interferometer.
The observed data is a discretely sampled time series:
\begin{align}
	\ldots,\ d_{-2},\ d_{-1},\ d_{0},\ d_{1},\ d_{2},\ \ldots
\end{align}
We assume that the ringdown has been identified as starting at the time of $d_0$.
The GW signal has a large amplitude around $d_0$, but decays to zero at early and late times.

The standard approach to analyzing the ringdown is to cut the data at the start time where the signal amplitude is large and take only the data after that time (i.e.\ $d_{0},\ d_{1},\ d_{2},\ \ldots$) and to model this using a superposition of QNMs described by parameters $\pvec{\theta}$ (e.g.\ the remnant mass, spin, amplitudes and phases for each QNM).
The likelihood is written as
\begin{align} \label{eq:motivation_A}
	\mathcal{L}(d_{0}, d_{1}, d_{2}, \ldots|\pvec{\theta}),
\end{align}
and, because we have cut the signal where the amplitude is large, this must be expressed in the time domain [Eq.~(\ref{eq:logL_TD_discrete})] to avoid problems of spectral leakage.

We could instead extend our analysis segment backwards by starting at $d_{-1}$, and then analyze this longer data stream with a new model that treats the value of the signal at $d_{-1}$ as being a completely free parameter. 
The model is otherwise unchanged at later times and is now described by parameters $(\hat{d}_{-1}, \pvec{\theta})$.
Note that the new model for $d_{-1}$ is entirely unphysical and is also discontinuous in the sense that there is no requirement that the model takes similar values at times $-1$ and 0.
The likelihood for this new model
\begin{align} \label{eq:motivation_B}
	\mathcal{L}(d_{-1}, d_{0}, d_{1}, d_{2}, \ldots|\hat{d}_{-1}, \pvec{\theta}),
\end{align} 
is also given by Eq.~(\ref{eq:logL_TD_discrete}), only with a slightly larger covariance matrix.
If we marginalize this with respect to the ``inspiral-merger parameter'' $\hat{d}_{-1}$ 
(adopting a flat, improper prior on $\hat{d}_{-1}$ ranging between $\pm\infty$)
then we recover the original likelihood in Eq.~(\ref{eq:motivation_A}), i.e.\
\begin{align} \label{eq:motivation_C}
	\mathcal{L}(d_{0}, d_{1}, d_{2}, &\ldots|\pvec{\theta}) = \\ &\int_{-\infty}^{\infty} \mathrm{d}\hat{d}_{-1}\; \mathcal{L}(d_{-1}, d_{0}, d_{1}, d_{2}, \ldots|\hat{d}_{-1}, \pvec{\theta}). \nonumber
\end{align}
This follows from well-known properties of the Gaussian distribution.

We can of course include more early-time data in a similar way. 
If we treat the value of the GW signal at each of the times $d_{-2},\ d_{-3},\ \ldots$ as free parameters and marginalize over all of them then we recover the original likelihood in Eq.~(\ref{eq:motivation_A}).
The point of doing this is that it allows us to start the analysis at early times when the amplitude of the GW signal in the data is small.
This means we can apply windowing, aka tapering, operations to the data without fear of suppressing the signal, and we can therefore transform the likelihood into the frequency domain without encountering spectral leakage problems.

The extension of this argument to multiple interferometers is straightforward. 
The values of the model at each early time in each interferometer must all be treated independently as free parameters and marginalized over in the manner of Eq.~(\ref{eq:motivation_C}).

The point we wish to emphasize is that an analysis of only the ringdown data (usually done in the time domain to avoid problems of spectral leakage) is equivalent to an analysis of all the data (which can be done in the frequency domain using standard GW data analysis techniques) if the inspiral-merger part of the signal is suitably marginalized out. 
This requires the use of an unphysical and discontinuous model for the inspiral-merger signal.
For the equivalence to be exact, the inspiral-merger model should include an extremely large number of free parameters (one for each time stamp in each interferometer), but we will argue below that in practice it is sufficient to use a smaller number of parameters provided a sufficiently flexible model is used.

This discussion motivates the model we describe in Sec.~\ref{subsec:model}. Once the likelihood is expressed in the frequency domain, several extensions of the analysis (such as treating the ringdown start time as a parameter of the model; see Sec.~\ref{subsec:t0}) become natural.


\subsection{Model}\label{subsec:model}

As it is clearly impractical to model the data at each time stamp as a free parameter, we propose to use a continuous (but very flexible) inspiral-merger model instead. 
We choose a sum of sine-Gaussian wavelets, which are then truncated at the ringdown start time and attached to a ringdown QNM model. 
With this method, we aim to model the full inspiral-merger-ringdown signal.

The ringdown model is zero for early times, and after a start time $t_0$ takes the form
\begin{align}
	h^\mathrm{R}(t) &= h_+^\mathrm{R}(t) - ih_\times^\mathrm{R}(t) \nonumber \\
	&= \sum_{\ell m n} A_{\ell m n} e^{-i[\omega_{\ell m n}(t-t_0) - \phi_{\ell m n}]}, \quad t \geq t_0,
\end{align}
where the complex QNM frequencies $\omega_{\ell m n} = 2\pi f_{\ell m n} - i/\tau_{\ell m n}$ are functions of the remnant BH mass $M_f$ and dimensionless spin magnitude $\chi_f$. Here, $f_{\ell m n}$ is the oscillation frequency, and $\tau_{\ell m n}$ is the damping time.
Each QNM is further described by an amplitude, $A_{\ell m n}$, and phase parameter, $\phi_{\ell m n}$. 
It is possible to analytically take the Fourier transform of this expression and thereby write the ringdown model in the frequency domain as
\begin{align}
	\Tilde{h}^{\mathrm{R}}(f) &= \int_{-\infty}^\infty \dd{t} \qty[h_+^\mathrm{R}(t) - ih_\times^\mathrm{R}(t)] e^{-2\pi i f t} \nonumber \\
	&= \sum_{\ell m n} \frac{A_{\ell m n} e^{-i[2\pi ft_0 - \phi_{\ell m n}]}}{i(\omega_{\ell m n} + 2\pi f)}.
\end{align}

We model the inspiral-merger part of the signal as a truncated sum of $W$ wavelets.
The inspiral-merger model is zero for late times, but before the start time $t_0$ takes the form
\begin{align}
	h^\mathrm{IM}(t) &=  h_+^\mathrm{IM}(t) - ih_\times^\mathrm{IM}(t) \nonumber \\
	&= \sum_{w=1}^{W} \mathcal{A}_w \exp \Bigg[-2\pi i \nu_w(t-\eta_w) \\
	&\hspace{2.46cm} - \qty(\frac{t-\eta_w}{\tau_w})^2 + i\varphi_w \Bigg], \quad t < t_0. \nonumber
\end{align}
The limit on time is equivalent to a model which is multiplied by a Heaviside step function, $H(t_0 - t)$. Each wavelet is described by five parameters: $\mathcal{A}_w$ and $\varphi_w$ are the wavelet amplitudes and phases, $\tau_w$ are the wavelet widths, $\nu_w$ are the wavelet frequencies, and $\eta_w$ are the wavelet central times. Again, it is possible to analytically take the Fourier transform of this expression and thereby write the inspiral-merger model in the frequency domain as
\begin{align}
	&\Tilde{h}^\mathrm{IM}(f) = \int_{-\infty}^\infty \dd{t} \qty[h_+^\mathrm{IM}(t) - ih_\times^\mathrm{IM}(t)] e^{-2\pi i f t} \\
	&\quad= \sum_{w=1}^{W} \mathcal{A}_w \exp[-2\pi i\nu_w \eta_w -\pi^2(f+\nu_w)^2\tau_w^2 +i\varphi] \nonumber\\
	&\quad\quad\times \frac{\sqrt{\pi}}{2}\tau_w \left(1 + \mathrm{erf} \left[ \frac{t_0-\eta_w}{\tau_w} + \pi i(f+\eta_w)\tau_w \right] \right).\nonumber
\end{align}

The full IMR model is simply given by
\begin{equation}
	h(t) = h^\mathrm{IM}(t) + h^\mathrm{R}(t).
\end{equation}
We refer the reader to Fig.~\ref{fig:demo}, which provides an illustration of this model with a single QNM and a single wavelet ($W=1$).

If $N$ QNMs are used, then the ringdown part of the model is described by $2N+2$ parameters.
The inspiral-merger part of the model is described by $5W$ parameters (these numbers do not include $t_0$).
Additionally, there are two sky position angles ($\alpha$ and $\delta$) and a polarization angle ($\psi$) which enter the detector response described below.
Finally, the ringdown start time ($t_0$) can also be treated as a model parameter in our frequency-domain approach.
For a typical choice of these parameters, our IMR model is discontinuous at $t_0$.
The inspiral-merger part of the model contains no information of the physics of the source and no attempt is made to enforce continuity between the two parts; this helps to decouple the ringdown inference from the inspiral-merger parts of the data and thereby ensure that we are really performing a ringdown analysis. 

To complete the description of our model, the detector response is given by projecting the waveform polarizations onto each interferometer (IFO) with the appropriate antenna patterns, $F^\mathrm{IFO}_{+,\times}$.
For a given sky location and GW polarization angle the detector response for each ${\mathrm{IFO}\in \{\mathrm{H}, \mathrm{L}, \mathrm{V} \}}$ is given by
\begin{align} \label{eq:projection_antenna}
	h^\mathrm{IFO}(t) = F^\mathrm{IFO}_+(\alpha, \delta, \psi) ~ &h_+(t + \Delta t_\mathrm{IFO}) \nonumber \\
	+ F^\mathrm{IFO}_\times(\alpha, \delta, \psi) ~ &h_\times(t + \Delta t_\mathrm{IFO}).
\end{align}
Here, $\Delta t_\mathrm{IFO}(\alpha, \delta)$ accounts for the different signal arrival times at the detectors and is also a function of sky location.
By definition, $h_+(t) = \Re\{ h(t) \}$, and $h_\times(t) = -\Im \{ h(t) \}$.
Note, the frequency-domain waveforms presented here are Fourier transforms of the complex polarization sum $h_+(t) - ih_\times (t)$. This means the separation into the plus and cross polarizations is not as simple as for the time-domain waveforms. Instead, the property that $\Tilde{h}^*_{+,\times}(-f) = \Tilde{h}_{+,\times}(f)$ for a real time-series implies that
\begin{gather}
	\Tilde{h}_+(f) = \frac{\Tilde{h}(f) + \Tilde{h}^*(-f)}{2}, \\ 
	\Tilde{h}_\times(f) = -\frac{\Tilde{h}(f) - \Tilde{h}^*(-f)}{2i}.
\end{gather}


\section{Injection study}\label{sec:injection_study}

We use the numerical relativity surrogate NRHybSur3dq8 \cite{Varma:2018mmi} to simulate the full inspiral, merger and ringdown signal from GW190521-like and GW150914-like sources. 
We use these two sources to test our frequency-domain approach on the analysis of the ringdown and compare the results with a standard, time-domain analysis.
Results from the GW190521-like analyses are shown here while the results from the GW150914-like analyses are shown in Appendix~\ref{app:GW150914}.

For all the GW190521-like injections, the surrogate was initialized with a total mass of $271\,M_\odot$ (all masses are given in the detector frame) and a mass ratio of $1.27$. 
For simplicity, all of the component spins were set to zero and the inclination angle was also chosen to be zero (i.e.\ the source was injected ``face-on'').
The simulated sky location and GW polarization angle were taken to be the maximum likelihood values from the NRSur7dq4 analysis in Refs.~\cite{LIGOScientific:2020ibl, gwtc2datarelease} ($\alpha = 0.164$, $\delta = -1.14$, $\psi = 2.38$).
The distance to the binary was chosen so that it gives a particular value of the optimal SNR in Livingston; this was usually chosen to be 15 (corresponding to a distance of $4016\,\mathrm{Mpc}$) so that it would likely be possible to detect multiple QNMs (in particular overtones), but several smaller values are also considered in Sec.~\ref{subsec:overtones}.
We perform zero-noise injections (i.e. analyzing simulated data with a noise realization of zero) into a three-interferometer H-L-V LIGO-Virgo network (except in Sec.~\ref{subsec:t0} where a two-interferometer H-L injection is performed for comparison) and use the average PSDs from the first three months of O3 (available at Ref.~\cite{o3psd}).

\begin{figure}[t]
	\centering
	\includegraphics[width=0.8\columnwidth]{FrequencyDomainAnalysisofBlackHoleRingdowns/mass_spin_corner_fixed_sky.pdf}
	\caption[Posteriors on the remnant mass and spin for a GW190521-like injection analyzed with a fixed sky position, GW polarization angle and ringdown start time]{ 
		Posteriors on the (detector frame) remnant mass and dimensionless spin for the GW190521-like injection using a single QNM and analyzed with a fixed sky position, GW polarization angle and ringdown start time.
		The main panel shows the $90\%$ confidence contour while the side plots show the one-dimensional marginalized posteriors.
		The solid blue line shows the results of a time-domain (TD) analysis.
		The other dashed and dotted lines show the results of frequency-domain analyses using different numbers of wavelets, $W$.
		The vertical and horizontal lines indicate the true values.
	}
	\label{fig:mass_spin_corner_fixed_sky}
\end{figure}

Although an important advantage of our approach is that it allows for easy marginalization over the source sky position and ringdown start time, we first apply it to the case where these are fixed. 
This allows us to compare our results more directly to those from a time-domain analysis.
The sky position and polarization angles $\alpha$, $\delta$ and $\psi$ are fixed to their injected values and the ringdown start time $t_0$ is fixed to be $12.7\,\mathrm{ms}$ ($\sim 10\,M_f$ in geometric units) after the peak of the strain (this choice follows the analysis of the real GW190521 signal in Ref.~\cite{LIGOScientific:2020iuh}).

\begin{figure*}
	\centering
	%
	\begin{minipage}{0.49\linewidth}
		\includegraphics[width=0.8\columnwidth]{FrequencyDomainAnalysisofBlackHoleRingdowns/mass_spin_corner.pdf} 
	\end{minipage}
	%
	\hfill
	%
	\begin{minipage}{0.49\linewidth}
		\vspace{1.2cm}
		\includegraphics[width=0.8\columnwidth]{FrequencyDomainAnalysisofBlackHoleRingdowns/t0_prior_posterior.pdf}
	\end{minipage}
	%
	\caption[Posteriors on the remnant mass and spin for a GW190521-like injection analyzed while marginalizing over the sky position, polarization angle and ringdown start time]{ 
		\emph{Left}: 
		Similar to Fig.~\ref{fig:mass_spin_corner_fixed_sky}, posteriors on the remnant mass and dimensionless spin for the GW190521-like injection using a single QNM while marginalizing over the sky position, polarization angle and ringdown start time.
		%
		Also shown in the gray shaded region is the result of a frequency-domain analysis that does not include any wavelets (i.e.\ $W=0$); as expected, since this model has an abrupt discontinuity at $t_0$, this analysis yields severely biased estimates of the remnant mass and spin. This $W=0$ analysis is included here to highlight the important role the wavelets play in our frequency-domain approach.
		%
		\emph{Right}: The prior and posterior distributions on the ringdown start time in the Hanford frame for the same frequency-domain analyses. The prior (solid blue line) is a Gaussian centered $12.7\,\mathrm{ms}$ after the time of peak strain in Hanford, with a standard deviation of $1\,\mathrm{ms}$. The prior has been chosen to be informative; the posterior distributions do not differ significantly from the prior.
		It is necessary to use such an informative prior because we find that the ringdown start time cannot be reliably inferred solely from the data (see discussion in Sec.~\ref{subsec:t0}).
		We observe a slight preference for an early start time when using a small number of wavelets; we speculate that this is due to the wavelet model being less flexible than the maximally flexible model described in Sec.~\ref{subsec:motivation}.
	}
	\label{fig:mass_spin_corner_zero_spin}
	%
\end{figure*}

First, for reference, an analysis using the time-domain expression for the log-likelihood was performed on this injection searching for the fundamental QNM (i.e.\ $\ell = m = 2$ and $n=0$).
Only the ringdown data $t\geq t_0$ was analyzed; the time series in each interferometer was shifted to geocenter time using the injected sky position, then cut to include $0.1\,\mathrm{s}$ of data from the start of the ringdown.
For each interferometer, the PSD was converted into an autocovariance function using the inverse of the transformation in Eq.~(\ref{eq:WKtheorem}) and this was used to construct the covariance matrix with Eq.~(\ref{eq:Toeplitz}).
We sample over the remnant mass ($M_f$, using a flat prior between $100\,M_\odot$ and $400\,M_\odot$), the dimensionless remnant spin ($\chi_f$, using a flat prior between $0$ and $0.99$), the QNM phase ($\phi_{220}$, using a flat, periodic prior between $0$ and $2\pi$), and the QNM amplitude ($A_{220}$, using a flat prior between 0 and $5 \times 10^{-21}$).
As modes beyond the fundamental will likely have amplitude posteriors consistent with zero, we use a flat prior (as opposed to a log-uniform prior) on the amplitudes; we have checked the choice of prior has minimal influence on the results.
We emphasize that $t_0$ is fixed in this analysis; i.e.\ using a delta-function prior.
The resultant posterior on the remnant parameters is shown in Fig.~\ref{fig:mass_spin_corner_fixed_sky}.
The posterior is consistent with the true remnant properties indicated by the vertical and horizontal lines. 
The true values were obtained with the NRSur3dq8Remnant model \cite{Varma:2018aht, Varma:2019csw, vijay_varma_2018_1435832} which, provided with the injection parameters, can estimate the remnant properties.

Second, the corresponding frequency-domain analyses using $W=1$, 2 and 3 truncated wavelets were also performed on the same injection but now using $4\,\mathrm{s}$ of data centered on the signal.
The same ringdown parameters and priors as before were used. 
In addition, we now sample over the wavelet amplitudes ($\mathcal{A}_w$, using a flat prior between 0 and $5 \times 10^{-21}$), phases ($\varphi_w$, using a flat, periodic prior between $0$ and $2\pi$), widths ($\tau_w$, with flat priors between $5\,\mathrm{ms}$ and $100\,\mathrm{ms}$, or in geometric units between $\sim 4\,M_f$ and $\sim 80\,M_f$) and frequencies ($\nu_w$, with flat priors between $30\,\mathrm{Hz}$ and $100\,\mathrm{Hz}$, or in geometric units between $\sim 0.04\,M_f^{-1}$ and $\sim 0.13\,M_f^{-1}$). 
The label-switching ambiguity among the wavelets was removed by enforcing the ordering $\mathcal{A}_w\leq\mathcal{A}_{w+1}$ via the \emph{hypertriangulation} transformation described in Ref.~\cite{Buscicchio:2019rir}.
We also sample over the wavelet central times ($\eta_w$) using a Gaussian prior with a width of $50\,\mathrm{ms}$ ($\sim 40\,M_f$) centered on the ringdown start time; this choice was empirically found to be sufficiently flexible, whilst also encouraging the wavelets to accurately model the signal near the peak.
Recall that, for the moment, we are fixing the parameters $\alpha$, $\delta$, $\psi$ and $t_0$.
The resultant posteriors on the remnant parameters are shown in Fig.~\ref{fig:mass_spin_corner_fixed_sky}.

From Fig.~\ref{fig:mass_spin_corner_fixed_sky} we see that our frequency-domain approach gives posteriors on the remnant properties that are consistent with both the true values and the time-domain analysis.
We see slight variations in the results of the frequency-domain analyses depending on the number of wavelets used.
This is a high-mass injection with a short inspiral-merger in-band, so it would be expected that a small number of wavelets would be sufficient.
In all cases our frequency-domain approach yields slightly more precise measurements of the remnant properties than the time-domain approach.
We speculate that this is because of some coupling between the ringdown and inspiral-merger parts of the model, which leads to information from the early-time data informing our measurement of the remnant properties.
Indeed, the inspiral-merger model with finite $W$ is only an approximation to the maximally flexible model described in Sec.~\ref{subsec:motivation}.

It is also possible to visually check the performance of the frequency-domain approach by plotting the whitened waveform reconstructions.
These reconstructions, which are the relevant time series that enter the frequency-domain log-likelihood, were found to be in excellent agreement with the injected data. 
Examples of such reconstructions are shown for the GW150914-like injection in Appendix~\ref{app:GW150914}.

\begin{figure*}
	\centering
	%
	\begin{minipage}{0.49\linewidth}
		\includegraphics[width=0.8\columnwidth]{FrequencyDomainAnalysisofBlackHoleRingdowns/mass_spin_corner_widths.pdf}
	\end{minipage}
	%
	\hfill
	%
	\begin{minipage}{0.49\linewidth}
		\vspace{1.2cm}
		\includegraphics[width=0.8\columnwidth]{FrequencyDomainAnalysisofBlackHoleRingdowns/t0_prior_posterior_widths.pdf} 
	\end{minipage}
	%
	\caption[Similar to Fig.~\ref{fig:mass_spin_corner_zero_spin}, posteriors on the remnant mass and spin for the GW190521-like injection using a single wavelet ($W=1$) and a single QNM using different priors on $t_0$]{ 
		\emph{Left}: Similar to Fig.~\ref{fig:mass_spin_corner_zero_spin}, posteriors on the remnant mass and spin for the GW190521-like injection using a single wavelet ($W=1$) and a single QNM using different priors on $t_0$.
		The markers indicate the maximum likelihood values.
		The dashed orange curve is identical to that in Fig.~\ref{fig:mass_spin_corner_zero_spin}.
		\emph{Right}: The corresponding posteriors on the ringdown start time. For each, the prior is a Gaussian centered on the vertical line, with widths given in the legend.
		It can be seen that a wider prior causes earlier ringdown start times to be favored (this is the case even when additional wavelets are included). As a result of the earlier start time, a bias appears in the recovered remnant parameters and higher values of both $M_f$ and $\chi_f$ are favored.
	}
	\label{fig:start_time_prior}
\end{figure*}

We now turn to the case where the source sky position, polarization angle and the ringdown start time are treated as free parameters in the frequency-domain analysis.
We use a uniform prior over the sphere of the sky and a flat, periodic prior on $\psi$ between $0$ and $\pi$.
As the sky position is now allowed to vary in the analysis, the time delay between the different interferometers and the geocenter is not constant. 
Therefore, for the ringdown start time, we choose to place a Gaussian prior on the start time in one of the detectors where the ringdown is clearly visible (we choose Hanford). 
The Gaussian prior was centered on the fixed value used in the time-domain analysis and has a relatively narrow width of $1\,\mathrm{ms}$ ($\sim 0.8\,M_f$).
This choice of prior is quite restrictive (i.e.\ assuming good prior knowledge of $t_0$) and the reasons for this are discussed further in Sec.~\ref{subsec:t0}; although, we note here that this is still more flexible than the delta-function prior used above.
%See Sec.~\ref{subsec:t0} for further discussion of this prior choice.
The resultant posteriors on the remnant parameters are shown in left panel of Fig.~\ref{fig:mass_spin_corner_zero_spin}.
The results in the left panel of Fig.~\ref{fig:mass_spin_corner_zero_spin} show that the performance of our frequency-domain approach is not significantly degraded when the searching over $\alpha$, $\delta$, $\psi$ and $t_0$.
Also shown in the right panel of Fig.~\ref{fig:mass_spin_corner_zero_spin} are the posteriors on the ringdown start time, $t_0$ which are discussed in more detail in the next section.

Stochastic sampling was performed using the \texttt{dynesty} \cite{Speagle:2019ivv} implementation of the nested sampling algorithm \cite{doi:10.1063/1.1835238, Skilling:2006gxv}.
For the sampling method, we used random walks with fixed proposals.
Typically, the minimum number of steps used in the random walk was 2000, and the number of live points was 4000.
We note that our frequency-domain ringdown analysis has many more parameters than a time-domain analysis due to the $5W$ parameters used in the wavelet sum.
Posteriors on these inspiral-merger parameters are not presented here, but we note that as the number of wavelets is increased strong degeneracies develop among these parameters.
This is expected, and is desirable in this context, as the wavelet part of the model is designed to be extremely flexible. 
These degeneracies in the inspiral-merger part of the model are not a problem for our present purpose as they do not inhibit our ability to measure the QNMs or the remnant properties. 
All posterior samples, including those for the wavelet parameters, are available at Ref.~\cite{finch_eliot_2021_5569759}.


\subsection{Determining the ringdown start time}\label{subsec:t0}

The ringdown start time, $t_0$, appears as a model parameter in our frequency-domain approach.
This raises two interesting questions: what prior should be placed on $t_0$, and how well can $t_0$ be measured from the data?
The possibility of determining $t_0$ from the data is particularly enticing because the ringdown start time is theoretically uncertain and its choice is crucial for any ringdown analysis.

\begin{figure*}
	\centering
	\includegraphics[width=0.8\columnwidth]{FrequencyDomainAnalysisofBlackHoleRingdowns/t0_geocent_posterior.pdf}
	\caption[Posteriors on the ringdown start time]{ 
		\emph{Main panel:} posterior on the ringdown start time in the geocentric frame. The orange line corresponds to the $W=1$ model applied to the three-detector network injection (this is the same analysis presented in Figs.~\ref{fig:mass_spin_corner_zero_spin} and \ref{fig:start_time_prior}, which was plotted with a dashed orange line). 
		The dashed blue line corresponds to a similar analysis on a two-detector network injection (with Virgo removed). This is to motivate the choice to parameterize $t_0$ in the frame of a detector; in the geocentric frame, a multimodal structure appears as a result of different possible sky locations. This makes it harder to place a sensible prior.
		\emph{Left inset plot:} the sky location posterior on the southern hemisphere (orthographic projection). This contains the injected source location (indicated by the star) which is correctly recovered with sky area $\sim 77\ \mathrm{deg}^2$ (90\% confidence) for the three-detector network, and $\sim 1800\ \mathrm{deg}^2$ for the two-detector network.
		\emph{Right inset plot:} the northern hemisphere of the sky contains a secondary mode when using the two-detector network, which correlates with $t_0^\mathrm{geo}$. Both modes of the sky posterior are elongated along the circle of constant time delay between the two detectors.
	}
	\label{fig:t0_geocent_posterior}
\end{figure*}

Unfortunately, we find that when using wide priors on $t_0$, early ringdown start times are generally favored and this leads to a bias in the recovered remnant mass and spin.
This can be seen in the results in Fig.~\ref{fig:start_time_prior}, where the $W=1$ analysis previously shown in Fig.~\ref{fig:mass_spin_corner_zero_spin} is repeated with increased values of the $t_0$ prior width.
The posterior on $t_0$ is also affected by the number of wavelets used;
as can be seen from the right panel of Fig.~\ref{fig:mass_spin_corner_zero_spin}, larger values of $W$ tend to favor later ringdown start times. 
We have repeated the analyses in Fig.~\ref{fig:start_time_prior} with larger values of $W$ to see if this counteracts the preference for an early start time (and hence removes biases in the remnant parameters), however, this was found not to be the case.
These calculations show that the posterior obtained on the parameter $t_0$ in our approach depends on the prior and on the number of wavelets used in the (unphysical) model of the inspiral-merger signal.
Therefore, it does not seem to be possible to reliably measure the ringdown start time from the data alone.
It is for this reason that a narrow, informative, $t_0$ prior must be used in the analyses described in the previous section.
Although not desirable, this is still an improvement over the fixed $t_0$ routinely used in most time-domain analyses.

Finally, we discuss the choice that was made in the previous section to place the prior on $t_0$ in the frame of one of the interferometers.
Because the sky position is allowed to vary, using the geocenter time is inappropriate due to coupling with the sky position.
The orange curve in Fig.~\ref{fig:t0_geocent_posterior} shows the posterior on $t_0$ transformed into the geocenter frame from the $W=1$ frequency-domain analysis using the narrow $1\,\mathrm{ms}$ prior on the ringdown start time. 
Also shown in the blue-dashed line is a posterior from an identical injection into a two-interferometer H-L network.
Due to the multimodal sky posterior, the posterior on $t_0$ in the geocenter frame can also be multimodal (this is present in the three-detector analysis to a smaller extent but is most clear in the two-detector analysis).
This makes choosing a suitable prior for the ringdown start time more difficult in the geocenter frame.
It is for this reason that for the analyses described above, the prior was specified in the frame of one of the detectors.
The results in Fig.~\ref{fig:t0_geocent_posterior} also show that our frequency-domain approach yields a posterior on the source sky position as a by-product of the ringdown analysis. However, it should be stressed that this is \emph{not} a ringdown-only result; the entire IMR model, including the unphysical wavelet part, is contributing to this sky localization.


\subsection{Detecting additional QNMs}\label{subsec:overtones}

A key goal in the analysis of BH ringdowns is the detection of additional QNMs beyond the fundamental $\ell=m=2$, $n=0$ mode.
This has already been achieved; see, for example, Ref.~\cite{Isi:2019aib} where the $\ell=m=2$, $n=1$ overtone was identified in GW150914 using a time-domain analysis.
In this section we show, using our GW190521-like injection, that our frequency-domain approach is also able to identify additional QNMs.

\begin{figure*}
	\centering
	\includegraphics[width=0.8\columnwidth]{FrequencyDomainAnalysisofBlackHoleRingdowns/overtone_corner.pdf}
	\caption[Posteriors on the QNM amplitudes and remnant mass and spin for one- and two-mode analyses of a GW190521-like injection]{ 
		Posteriors on the QNM amplitudes and remnant mass and spin for one- and two-mode analyses (1QNM and 2QNM respectively) of the GW190521-like injection, performed in the frequency domain.
		The results in blue are for the recovery using two QNMs [the fundamental $(\ell,m,n)=(2,2,0)$ and its first overtone $(2,2,1)$] which are both detected with non-zero amplitudes using a $t_0$ prior centered on the time of the peak strain and with a width of $1\,\mathrm{ms}$.
		%
		Also shown in orange for comparison are the results using one QNM [the fundamental $(2,2,0)$ only] with a prior centered $12.7\,\mathrm{ms}$ after the peak, again with a width of $1\,\mathrm{ms}$.
		%
		The vertical and horizontal solid gray lines indicate the true values of the remnant mass and spin and the diagonal dashed gray line indicates $A_{220}=A_{221}$.
		%
		The difference in the $A_{220}$ amplitude between the two analyses can be explained by the different ringdown start times and the decay of the $\ell = m = 2$, $n = 0$ QNM.
		Over a time $\sim 12.7$ ms, we expect the $A_{220}$ amplitude to decay by a factor $\sim \exp[-12.7\,\mathrm{ms}/\tau_{220}] \approx 0.5$.
		This is shown in the shaded gray posterior in the top-left panel where the results of the 2QNM analysis are used to predict the value of the amplitude at the later start time used by the 1QNM analysis.
	}
	\label{fig:overtone_corner}
\end{figure*}

As a first step towards testing our model we search for the $n=1$ overtone of the fundamental QNM.
It would also be possible to search for higher harmonics (e.g.\ modes with $\ell\geq 3$); however, the results of previous investigations on numerical relativity simulations (see, e.g.\ \cite{Giesler:2019uxc, Ota:2019bzl, Dhani:2020nik, Finch:2021iip}) suggest that overtones are generally more prominent than harmonics in the ringdown and are therefore a natural first target for any search.

We reanalyze the GW190521-like injection in the frequency domain using the $W=1$ inspiral-merger model but this time including an overtone in the ringdown (the choice to use a single wavelet is motivated by the previous results; it is sufficient to model the inspiral-merger for this high-mass injection, and we see no significant improvements with additional wavelets).
When using overtones, it is appropriate to start the ringdown analysis at an earlier time. 
For the frequency-domain analyses a Gaussian prior with a standard deviation of $1\,\mathrm{ms}$ centered at the time of the peak strain was used (this is $12.7\,\mathrm{ms}$ earlier than was used above).
The results of this ``2QNM'' analysis are shown in Fig.~\ref{fig:overtone_corner}, along with the fundamental only ``1QNM'' analysis for comparison.
Posteriors are plotted for the QNM amplitudes and the remnant mass and spin parameters. 
We see that the overtone can be confidently detected with non-zero amplitude.
The 2QNM analysis yields more precise measurements of the remnant mass and spin due to a combination of the earlier ringdown start time (which gives a larger ringdown SNR) and the improved ringdown model.

We now turn our attention to the resolvability of this additional QNM as a function of the injected SNR and compare the sensitivities of the time- and frequency-domain approaches.
The GW190521-like source was re-injected at a series of lower SNRs: 12, 9, and 6 (in the Livingston detector).
A $W=1$ frequency-domain analysis was re-performed on this sequence of injections, along with a time-domain analysis for comparison. 
Following the use of an earlier ringdown start time in the frequency-domain analysis, for the time-domain analysis the ringdown start time was fixed to the peak of the strain.
The posteriors on the amplitude $A_{221}$ of the overtone are shown in Fig.~\ref{fig:A221_posterior}.
As the SNR is decreased, the overtone becomes increasingly difficult to detect and the posteriors become consistent with $A_{221}=0$.
This is the case for both the time- and frequency-domain analyses which give similar results.
This suggests the time- and frequency-domain approaches are equally sensitive to additional QNMs.

\begin{figure}[t]
	\centering
	\includegraphics[width=0.8\columnwidth]{FrequencyDomainAnalysisofBlackHoleRingdowns/A221_posterior.pdf}
	\caption[Overtone amplitude posteriors from a $W=1$ frequency-domain analysis]{ 
		Overtone amplitude posteriors from a $W=1$ frequency-domain (FD) analysis, where the fundamental $(\ell,m,n) = (2,2,0)$ QNM and its first overtone $(2,2,1)$ are included in the ringdown model. For comparison, the overtone amplitudes from a time-domain (TD) analysis are shown with the dashed lines. The injected SNR in Livingston is controlled by changing the injection luminosity distance: $D_L = \{4016.3,\ 5020.4,\ 6693.9,\ 10040.8\}$ Mpc for $\mathrm{SNR} = \{15,\ 12,\ 9,\ 6\}$ respectively. The maximum likelihood values scale as $D_L^{-1}$.
	}
	\label{fig:A221_posterior}
\end{figure}

\begin{table}[h]
	\centering
	\begin{tabular}{c|cccc}
		SNR & 15   & 12   & 9   & 6   \\ \hline
		TD & $~ 1.1 ~$ & $0.3$ & $-0.3$ & $-0.7$ \\
		FD & $~ 1.1 ~$ & $~ 0.4 ~$ & $~ -0.4 ~ $ & $~ -0.6 ~$   
	\end{tabular}
	\caption[The log-Bayes' factors in favor of an overtone for the series of GW190521-like injections at different SNRs, for both time-domain and $W=1$ frequency-domain analyses]{ 
		The log-Bayes' factors $\log_{10}\mathcal{B}^{2\mathrm{QNM}}_{1\mathrm{QNM}}$ in favor of an overtone for the series of GW190521-like injections at different SNRs, for both time-domain (TD) and $W=1$ frequency-domain (FD) analyses. 
		The uncertainties on these Bayes' factors are all $\pm (0.1$ -- $0.2)$, with errors on the evidences estimated from within a single nested sampling run.
	}
	\label{tab:bayes_factors}
\end{table}

Further evidence supporting this conclusion comes from the odds ratios (aka Bayes' factors) in favor of the overtone.
The Bayes' factors $\mathcal{B}^{2\mathrm{QNM}}_{1\mathrm{QNM}}$ (computed with equal prior odds) in favor of the second QNM were computed from both the time- and frequency-domain analyses. 
In order to do this, we perform an additional set of analyses on the series of injections used in Fig.~\ref{fig:A221_posterior} with the same ringdown start time (fixed at the peak for the time-domain analysis, and a Gaussian prior centered on the peak in Hanford for the frequency-domain) but without the overtone included.
We can then compute the evidence, $\mathcal{B}_{1\mathrm{QNM}}^{2\mathrm{QNM}}$, in favor of the 2QNM analysis (with an overtone) over the 1QNM analysis (fundamental mode only) keeping every other part of the analysis identical. 
The log-Bayes' factors for each of the different SNR injections are shown in Table.~\ref{tab:bayes_factors} where it can be seen that the time- and frequency-domain approaches are equally sensitive to the overtone mode.


\section{Conclusions}\label{ch3:sec:discussion}

BH ringdown and QNMs are a key area of study in the burgeoning field of GW astronomy and are particularly important for testing GR.
Ringdown analyses are usually performed in the time domain as this provides a natural way to work with discontinuous models and to apply sharp cuts to the data. 
However, in these analyses the ringdown start time and sky position usually have been fixed beforehand.
The log-likelihood is also more computationally expensive than in the frequency domain.

We have presented a novel approach for analyzing the ringdown in the frequency domain. 
Our approach uses a flexible combination of sine-Gaussian wavelets, truncated at the start of the ringdown, to effectively marginalize over the inspiral and merger parts of the signal. 
The benefits of performing the analysis in the frequency domain include being able to easily vary the source sky position and ringdown start time model parameters as part of the analysis.
As virtually all other GW data analysis is already performed in the frequency domain, a further benefit of our approach is that it allows us to utilize standard, and now very well-tested, GW analysis software packages for performing the Bayesian inference and also for estimating the noise properties.

We have tested our frequency-domain approach by analyzing a series of numerical relativity surrogate injections and by comparing our results with those from a time-domain analysis. 
We find that our frequency-domain approach is equally sensitive to additional QNMs compared to the time-domain approach.
However, we find that it generally yields more precise measurements of the remnant BH mass and spin parameters which we speculate is due to some small coupling with the inspiral and merger signal.
We also paid particular attention to the choice of prior on $t_0$; although this appears as a model parameter in our approach it was found that, unfortunately, it was not possible to reliably determine it solely from the data.

In future we hope to test our method on a larger set of simulated signals, including those with more extreme mass ratios and different spin configurations, and to apply the method to real GW data.

\section{GW150914-like Injection}\label{app:GW150914}

In the main body of the paper the frequency-domain ringdown analysis was tested on GW190521-like injections with varying SNR and observed using a network of two or three interferometers. It was found to perform well. In this appendix we test the frequency-domain approach further by analyzing a GW150914-like injection.

The surrogate was initialized with a total mass of $72.2\,M_\odot$ and a mass ratio of $1.16$. 
As before, all of the component spins were set to zero for simplicity. 
The simulated sky location and GW polarization angle were taken to be $\alpha = 1.95$, $\delta = -1.27$, and $\psi = 0.82$. These are consistent with the GW150914 posterior and were chosen to match the values used in \cite{Isi:2019aib}.
The distance to the binary was set to $471.4\,\mathrm{Mpc}$, which gave an optimal SNR in Hanford of 25.

The inclination angle was chosen to be $\pi$ (i.e\ the source is injected ``face-off'') which is consistent with the GW150914 posterior.
The source inclination affects the GW polarization, and this is handled via the introduction of a ``ellipticity parameter'' $\epsilon$ which has the effect of transforming $h_\times(t) \rightarrow \epsilon h_\times(t)$.
For the ``face-on'' injections in the main text $\epsilon=1$ was used, while for the ``face-off'' injections considered here $\epsilon=-1$.
A more general analysis would allow $\epsilon$ to vary as a free parameter, such as what was done in Ref.~\cite{Isi:2021iql}.

We perform zero-noise injections into the two-interferometer H-L LIGO network that was operating at the time of the first detection. 
We use the PSDs associated with the data surrounding GW150914 (available at Ref.~\cite{gwtc1psds}).
These different parameters (particularly the lower total mass) result in a signal with a longer inspiral.
This is an important test case for our model as there is a much larger fraction of the SNR in the inspiral-merger (compared to the GW190521-like injection) which has to be ``marginalized out'' in the analysis.

As was done initially for the GW190521-like injection, we fix the sky location and polarization angle to the injected values to simplify the problem and aid comparison to the time-domain analysis.
The ringdown start time is also fixed to $3\,\mathrm{ms}$ after the time of the peak strain ($\sim 10\,M_f$ in geometric units).

Following the same procedure as in Sec.~\ref{sec:injection_study}, a time-domain analysis was first carried out to recover the fundamental QNM. 
The prior on the remnant mass was adjusted to reflect the lower injected value (flat between $50\,M_\odot$ and $100\,M_\odot$), but otherwise the analysis was unchanged from the time-domain analyses described in the main text.
The resultant remnant mass and spin posterior is shown by the blue solid line in Fig.~\ref{fig:GW150914_mass_spin_corner}.

\begin{figure}[b]
	\centering
	\includegraphics[width=0.8\columnwidth]{FrequencyDomainAnalysisofBlackHoleRingdowns/GW150914_mass_spin_corner.pdf}
	\caption[Similar to Fig.~\ref{fig:mass_spin_corner_fixed_sky}, posteriors on the recovered remnant mass and spin for the GW150914-like injection using the fundamental QNM and a varying numbers of wavelets]{ 
		Similar to Fig.~\ref{fig:mass_spin_corner_fixed_sky}, posteriors on the recovered remnant mass and spin for the GW150914-like injection using the fundamental QNM and a varying numbers of wavelets. 
		Also shown for comparison is the result of a time-domain analysis (solid blue line).
	}
	\label{fig:GW150914_mass_spin_corner}
\end{figure}

\begin{figure*}[t]
	\centering
	\includegraphics[width=0.8\columnwidth]{FrequencyDomainAnalysisofBlackHoleRingdowns/GW150914_whitened_waveforms.pdf}
	\caption[Whitened waveform reconstructions (in Hanford) corresponding to the results of Fig.~\ref{fig:GW150914_mass_spin_corner}]{ 
		Whitened waveform reconstructions (in Hanford) corresponding to the results of Fig.~\ref{fig:GW150914_mass_spin_corner}. 
		The top panel shows the waveform from a time-domain analysis. 
		In the time-domain approach the data before the ringdown start time is excluded, but here the waveform is plotted for all times.
		This highlights one of the problems with using a ringdown-only model in the frequency domain: the abrupt start of the model leads to spectral leakage when Fourier transforming (visible as oscillations before the ringdown start time).
		The following panels show waveforms from the frequency-domain approach.
		Problems with spectral leakage are avoided, due to the wavelets smoothly connecting to the ringdown part of the model.
		Just a single wavelet fails to model the full GW150914-like inspiral-merger, which is to be expected because of its longer duration in-band.
		As more wavelets are included in the model, more of the inspiral-merger is captured by the model.
		The difference in the reconstruction for three and five wavelets is minimal, showing the model is converging on the signal.
	}
	\label{fig:GW150914_whitened_waveforms}
\end{figure*}

Secondly, a series of frequency-domain analyses were carried out using an increasing number of wavelets.
Results for $W=1$, 3, and 5 are shown in Fig.~\ref{fig:GW150914_mass_spin_corner}.
We found a slightly more restrictive prior on the wavelet central times, $\eta_w$, was required to aid the inference; the Gaussian width was reduced to $10\,\mathrm{ms}$ ($\sim 30\,M_f$).
This encouraged the wavelets to fit near the ringdown start time, which is the part of the signal we are most interested in.
The upper bound on the wavelet frequencies was also increased to $500\,\mathrm{Hz}$ ($\sim 0.17\,M_f^{-1}$), as the lower binary mass means the merger-ringdown occurs at a higher frequency.
We see that a single wavelet is not quite sufficient to avoid bias in the remnant parameters, which may be expected when working with a longer inspiral.
As the number of wavelets is increased the bias disappears, and the remnant posteriors seem to converge to a solution that is stable against the inclusion of additional wavelets. 
As was the case for the GW190521-like analysis, we find the frequency-domain model achieves tighter constraints on the remnant parameters in comparison to the time-domain analysis. 

Finally, we inspect the whitened waveform reconstructions for all four analyses shown in Fig.~\ref{fig:GW150914_mass_spin_corner}.
We focus on the waveform in the Hanford detector.
We take samples from the posterior of each run and use these to compute the projected waveform $F^\mathrm{H}_{+} h_+(t+\Delta t_\mathrm{H})+F^\mathrm{H}_{\times} h_\times(t+\Delta t_\mathrm{H})$, see Eq.~(\ref{eq:projection_antenna}). This quantity is always discontinuous for both the time- and frequency-domain analyses.
We then whiten this waveform (and the data) using the Hanford PSD. After whitening, the projected waveform is continuous.
In Fig.~\ref{fig:GW150914_whitened_waveforms} we plot the median and $5\%-95\%$ credible region of the whitened waveform reconstructions.
The figure highlights the problem of spectral leakage, which occurs when taking Fourier transforms of discontinuous models. 
The time-domain model (top panel) has a discontinuity at the ringdown start time and this causes oscillations to appear before the start time in the whitened waveform.
The following panels, which include wavelets to model the inspiral-merger signal, remove this discontinuity and prevent Fourier transform artifacts.
A single wavelet is not sufficient to capture the full inspiral-merger signal, which likely causes the bias seen in Fig.~\ref{fig:GW150914_mass_spin_corner}. 
Increasing the number of wavelets makes the model flexible enough to model the inspiral-merger signal, and also to remove bias in the mass-spin posterior.


% Chapter 4

\chapter{Searching for a Ringdown Overtone in GW150914}

\label{Chapter4}

\section{Introduction}\label{ch4:sec:introduction}

The very first GW event, GW150914~\cite{LIGOScientific:2016aoc}, remains probably the best candidate for studying the ringdown.
This is a result of several factors, including its large SNR of $\rho\sim 24$ and its total mass of $M\sim 70\,M_\odot$ which places the merger and ringdown in the center of the LIGO~\cite{LIGOScientific:2014pky} sensitive frequency band at $\sim 200\,\mathrm{Hz}$. 
Additionally, GW150914 is by now the most well-studied GW event and therefore the signal and the properties of the noise in the surrounding data are extremely well understood.

The first tests of GR performed using GW150914 included an investigation of the ringdown~\cite{LIGOScientific:2016lio}. 
The ringdown signal, after a fixed starting time $t_0$, was modelled using a single damped sinusoid; the parameters of which were checked for consistency with the predicted least-damped QNM of the remnant BH.
This first attempt at a ringdown analysis was performed using the standard Whittle frequency-domain log-likelihood~\cite{10.2307/2983994}, commonly used in GW data analysis.
The ringdown was isolated by choosing a lower limit of $\sim 130\, \mathrm{Hz}$ in the frequency integral, effectively cutting the data mid-signal.
This approach suffers from several shortcomings. 
Firstly the frequency-domain cut at $\sim 130\, \mathrm{Hz}$ only approximately separates the ringdown from the early-time signal due to the breakdown of the stationary phase approximation near merger. 
Secondly the nonzero amplitude at the start of the signal model breaks the assumption of circularity for the Fourier transform, thereby introducing contamination in the form of spectral leakage. 
Therefore, this approach does not scale well to higher SNRs where noise will no longer dominate over the systematic errors introduced by the sharp frequency-domain cut.
Despite these drawbacks, this approach was successfully used in Ref.~\cite{LIGOScientific:2016lio} to identify the fundamental QNM in the GW150914 signal.

Since this initial attempt, several groups have developed new time-domain frameworks specifically for ringdown analyses~\cite{Carullo:2019flw, Isi:2019aib, Capano:2021etf}.
The principle motivation for working in the time domain is that it is easy to impose sharp cuts on the data at specific times (without any spectral leakage) and to analyse only data after a chosen start time (see Ref.~\cite{Isi:2021iql} for details of time-domain analysis methods).
These approaches have also enabled going beyond the fundamental mode. 
Generically, the ringdown can be modelled as a superposition of QNMs with complex frequencies $\omega_{\ell m n} = 2\pi f_{\ell m n} - i/\tau_{\ell m n}$, labelled with angular indices $\ell\geq 2$, $\abs{m}\leq\ell$, and an overtone index $n \geq 0$ [the fundamental mode has $(\ell, \abs{m}, n) = (2, 2, 0)$].
Detecting additional QNMs beyond the fundamental increases the scientific potential of ringdown studies, especially for fundamental tests of the Kerr metric, the no-hair theorems, and the BH area law~\cite{Dreyer:2003bv, Berti:2005ys, Gossan:2011ha, Brito:2018rfr, Carullo:2019flw, Isi:2019aib, Isi:2020tac}.

Outside of the testing-GR catalog papers mentioned previously~\cite{LIGOScientific:2020tif, LIGOScientific:2021sio}, other groups have attempted to identify additional QNMs in the ringdown data and perform tests of the no-hair theorem. 
This includes claims of detection of the $(2,2,1)$ overtone in GW150914~\cite{Isi:2019aib}, and claims of detection of the $(3,3,0)$ harmonic in GW190521~\cite{Capano:2021etf}. 
By allowing the QNM frequency of the secondary mode to deviate from the GR Kerr prediction, the above works found the measured spectrum to be in agreement with the no-hair hypothesis to within $\sim 20\%$ ($68\%$ credibility) and $\sim 1\%$ ($90\%$ credibility) respectively.

An early application of the time-domain framework was in Ref.~\cite{Isi:2019aib}, where Isi et al. claimed a detection of the first overtone of the fundamental QNM in the GW150914 signal [that is, the $(2, 2, 1)$ mode]. 
This was quickly followed by a separate detection claim of the $(3,3,0)$ harmonic mode in the signal of the $\sim 150M_\odot$ binary merger GW190521~\cite{LIGOScientific:2020iuh} by Capano et al.~\cite{Capano:2021etf} (this was done using an equivalent formulation of the time-domain method, although expressed in the frequency domain).
The claimed detection of an overtone was made possible partly because, compared to earlier studies, the authors chose to use an earlier start time for the ringdown; this was motivated by contemporary numerical relativity studies~\cite{Giesler:2019uxc} (see also Refs.~\cite{Bhagwat:2019dtm, Ota:2019bzl, Cook:2020otn, JimenezForteza:2020cve, Dhani:2020nik, Finch:2021iip, Forteza:2021wfq, Dhani:2021vac, MaganaZertuche:2021syq}) that demonstrated that when overtones are included the ringdown can be considered to start as early as the time of peak strain amplitude. 

However, a recent paper by Cotesta et al.~\cite{Cotesta:2022pci} reanalysed the GW150914 signal using very similar methods and found no significant evidence for an overtone.
It was also suggested that the earlier detection claims of Ref.~\cite{Isi:2019aib} were noise dominated.
(This prompted a response from Isi et al.~\cite{Isi:2022mhy} where they restated their claim to have detected an overtone in GW150914.) 
Ref.~\cite{Bustillo:2020buq} also found weaker evidence for an overtone using an analysis method closer to that of Ref.~\cite{LIGOScientific:2016lio}.
Similarly, the claim in Ref.~\cite{Capano:2021etf} that a harmonic had been detected in GW190521 has also been debated and no evidence for a harmonic was found by Ref.~\cite{LIGOScientific:2021sio}.
Amid this confusion, it is particularly concerning that the supposedly identical analyses in Refs.~\cite{Isi:2019aib, Isi:2022mhy}, and~\cite{Cotesta:2022pci} come to such different conclusions concerning which QNMs are in the data. 
Discrepancies of this sort risk jeopardising the science that can be done using future ringdown observations.

These discrepancies highlight some of the difficulties inherent in time-domain ringdown analysis, where important choices (that affect the results) for fixed quantities such as the ringdown start time have to be made and care must be taken with the noise covariance estimation.
If ringdown studies are to be used to make precision measurements of BH properties or as reliable tests of GR we must first be able to make reliable and reproducible determinations of the QNM content.
This is also not a problem that will be removed in the future with observations at higher SNR. Even if an event has a higher SNR that is sufficient for a clear detection of the first QNM overtone, the focus will then simply shift to trying to identify the next overtone (or else the next QNM harmonic) in the countably infinite ringdown sum~\cite{Bustillo:2020buq}.

To complement the time-domain analysis frameworks, in the last chapter we proposed a new method for ringdown analyses which works in the frequency domain.
A flexible sum of sine-Gaussian wavelets, truncated at the ringdown start time, is used to effectively marginalise over the inspiral-merger (i.e.\ pre-ringdown) part of the signal.
The model is completed by attaching this to the usual sum of QNMs which model the ringdown.
No continuity is enforced between the two parts of the model in order to keep the ringdown inference independent from the rest of the signal.
However, we find the continuity is effectively learned from the data, and any remaining discontinuities disappear entirely when the signal is ``whitened’’ according to the instrumental noise.
In a particular limit, this approach can be shown to be formally equivalent to the time-domain analyses described above.
However, this frequency-domain approach can be generalised and offers several advantages over time-domain approaches:
well-established GW data analysis methods and pipelines can be used (which are all built in the frequency domain), 
the inspiral-merger data informs the noise estimation at the start of the ringdown (improving parameter estimation accuracy), 
and the ringdown start time and the source sky position can be easily treated as free parameters and marginalised over as part of a Bayesian analysis (instead of being fixed).
We note, however, that (as discussed in Chapter~\ref{Chapter3}) a narrow and informative prior on the ringdown start time must be used.
Reweighting techniques can be employed to investigate different ringdown start time prior choices computationally efficiently in post processing (see Section~\ref{subsec:reweighting}) obviating the need for the large number of analyses performed in Refs.~\cite{Cotesta:2022pci, Isi:2022mhy}.

In this chapter the new frequency-domain method is applied to reanalysing the ringdown of GW150914 paying particular attention to the presence (or absence) of an overtone. 
We perform analyses with and without an overtone and investigate different choices of the ringdown start time. 
We also perform additional analyses with varying data sampling frequencies and integration limits to verify the stability of our results. Finally, a mock injection study into real detector noise is also performed to further assess the significance of any overtone detection.
Section~\ref{sec:analysis} describes the signal model, the data, and the analysis methods used in this chapter.
Section~\ref{sec:results} presents our main results including posteriors on the remnant BH properties and overtone amplitude, and Bayes' factors for the overtone model.
The results are discussed further in Section~\ref{ch4:sec:discussion}.
Throughout this chapter we make use of natural units where $G=c=1$.

All data products and plotting scripts used to make the figures in this chapter are made publicly available at Ref.~\cite{finch_eliot_2022_6949492}.


\section{Methods}\label{sec:analysis}

This section briefly describes the frequency-domain method for analysing BH ringdowns introduced in Chapter~\ref{Chapter3}:
the wavelet-ringdown model is described in Section~\ref{sec:model}; the data, likelihood and priors are described in Section~\ref{sec:details}; and our approach for dealing with changes to the ringdown start time is described in Section~\ref{subsec:reweighting}.


\subsection{Wavelet-ringdown model}\label{sec:model}

Our model consists of two parts: one for early times before $t_0$ which is referred to here as the \emph{inspiral-merger}, and another for the \emph{ringdown} after the start time $t_0$.

First, we describe the ringdown part of the model.
After a ringdown start time $t_0$, which is itself a parameter in the model, the model takes the form
\begin{equation}\label{eq:ringdown_model}
    h^\mathrm{R}(t) = h_+^\mathrm{R}(t) + ih_\times^\mathrm{R}(t) = \sum_{n=0}^N A_n e^{-i[\omega_{22n}(t-t_0) + \phi_{n}]}, \quad t \geq t_0. 
\end{equation}
Because our focus in this chapter is on the presence of an overtone, we fix the angular indices to $\ell = m = 2$ and vary only the number of QNM overtones, $N$, in the model ($N$ is always taken to be either 0 or 1 in this chapter). 
Note that the form of this equation differs slightly from Eq.\ref{ch3:eq:hr} in Chapter~\ref{Chapter3}. 
This is because the source inclination angle is fixed to be ``face-off'' (i.e.\ $\iota=\pi$).
In the notation of (for example) Refs.~\cite{Dhani:2020nik, Finch:2021iip, MaganaZertuche:2021syq}, this is equivalent to using the $\ell = -m = 2$ mirror modes. 
Or, in notation of Ref.~\cite{Isi:2021iql}, using an ellipticity of $\epsilon = -1$.
The complex QNM frequencies, $\omega_{\ell m n} = 2\pi f_{\ell m n} - i/\tau_{\ell m n}$, are functions of the remnant BH mass $M_f$ (detector frame) and dimensionless spin $\chi_f$.
Additionally, each QNM is further described by an amplitude, $A_{n}$, and a phase, $\phi_{n}$. 

Second, we describe the inspiral-merger part of the model.
This is modelled as a truncated sum of $W$ wavelets.
At early times the model takes the form
\begin{align}\label{eq:wavelets}
	h^\mathrm{IM}(t) &=  h_+^\mathrm{IM}(t) + ih_\times^\mathrm{IM}(t) \nonumber \\
	&= \sum_{w=1}^{W} \mathcal{A}_w \exp \Bigg[-2\pi i \nu_w(t-\eta_w) - \qty(\frac{t-\eta_w}{\tau_w})^2 - i\varphi_w \Bigg], \quad t < t_0. 
\end{align}
Again, the minor differences in sign conventions compared to Chapter~\ref{Chapter3} come from fixing the inclination angle to be face-off. 
The wavelets are each described by five parameters: $\mathcal{A}_w$ and $\varphi_w$ are the wavelet amplitudes and phases, $\tau_w$ are the wavelet widths, $\nu_w$ are the wavelet frequencies, and $\eta_w$ are the wavelet central times. 
In this chapter we use $W=3$ (three wavelets) in our model.
This number was empirically found to be sufficient (see Section~\ref{app:GW150914}, where the number of wavelets was varied for a GW150914-like injection).

The full signal model is given by discontinuously joining the inspiral-merger to the ringdown at $t_0$,
\begin{equation}
	h(t) = h^\mathrm{IM}(t) + h^\mathrm{R}(t).
\end{equation}

Finally, the detector response must be considered.
We project the waveform polarisations onto each interferometer (IFO) with the antenna patterns, $F^\mathrm{IFO}_{+,\times}$.
The detector response for each ${\mathrm{IFO}\in \{\mathrm{H}, \mathrm{L}\}}$ is given by
\begin{align} \label{ch4:eq:projection_antenna}
	h^\mathrm{IFO}(t) = F^\mathrm{IFO}_+(\alpha, \delta, \psi) ~ &h_+(t + \Delta t_\mathrm{IFO}) \nonumber \\
	+ F^\mathrm{IFO}_\times(\alpha, \delta, \psi) ~ &h_\times(t + \Delta t_\mathrm{IFO}),
\end{align}
where $\alpha$, $\delta$ are the source right ascension and declination, and $\psi$ is the GW polarisation angle.
The time delay $\Delta t_\mathrm{IFO}(\alpha, \delta)$ accounts for the different signal arrival times at the detectors and is also a function of the source sky location.
Throughout this chapter we quote times in the Hanford frame.
So, in particular, $t_0$ refers to the ringdown start time in Hanford.
By definition, $h_+(t) = \Re\{ h(t) \}$, and $h_\times(t) = \Im \{ h(t) \}$.


\subsection{Data and priors}
\label{sec:details}

We use the GW150914 strain data sampled at $4096\, \mathrm{Hz}$ for both the Hanford and Livingston interferometers, which was obtained from Refs.~\cite{gwosc, LIGOScientific:2019lzm}.
A total of $4096\,\mathrm{s}$ of data around the event was downloaded, from which the mean was subtracted (this is effectively equivalent to applying a $\sim 1\, \mathrm{Hz}$ highpass filter). 
Pre-computed power spectral densities (PSDs) associated with GW150914 from the GWTC-1 release were used~\cite{gwtc1psds}. 
It has been verified our results are insensitive to the exact noise PSD used; for example, our results are unchanged when using a PSD estimated from a length of off-source data.
The analysis data consists of $4\,\mathrm{s}$ of data centred on the event GPS time ($1126259462.4\,\mathrm{s}$), and a Tukey window with an alpha parameter of 0.2 was applied to this analysis data.
The Bayesian analysis used the standard frequency-domain log-likelihood function (see, e.g., Eq.~\ref{eq:logL_FD_continuous}), with the limits of the frequency integration between $20$ and $1000\, \mathrm{Hz}$.
The choices of sampling rate and upper limit of frequency integration are discussed further in Section~\ref{subsec:noise}.

All the model parameters described in Section~\ref{sec:analysis} were sampled over as part of a Bayesian analysis.
For the wavelet parameters, uniform priors are used for the amplitudes $(\mathcal{A}_w \in [0,10^{-20}])$, phases $(\varphi_w \in [0,2\pi])$, frequencies $(\nu_w \in [20,200]\, \mathrm{Hz})$, and widths $(\tau_w \in [4,80]\, \tilde{M_f}$, or equivalently $\sim[1.4,27]\, \mathrm{ms})$.
Here, $\tilde{M_f}=68.779M_\odot=0.33875\,\mathrm{ms}$ is a fixed point estimate of the final, detector-frame mass (obtained using the median value from Ref.~\cite{LIGOScientific:2018mvr}) and should not be confused with the varying model parameter $M_f$.
The label-switching ambiguity among the wavelets was removed by enforcing the ordering 
$ \nu_w \leq \nu_{w+1} $ via the hypertriangulation transformation described in Ref.~\cite{Buscicchio:2019rir}.
We sample over the wavelet central times ($\eta_w$) using a Gaussian prior in the Hanford frame with a width of $50\,\tilde{M_f}$ ($\sim 17\,\mathrm{ms}$) centred on $t_\mathrm{ref} = 1126259462.423\,\mathrm{s}$.
This choice was found to be sufficiently flexible, whilst at the same time encouraging the wavelets to accurately model the signal near the peak (see the discussion in Section~\ref{app:GW150914}).

For the ringdown, uniform priors are used for the amplitudes $(A_n \in [0,10^{-19}])$, phases $(\phi_n \in [0,2\pi])$, remnant mass $(M_f \in [40,100]\,M_\odot )$, remnant spin $(\chi_f \in [0,0.99])$ and ringdown start time $(t_0-t_\mathrm{ref} \in [-15, 15]\,\tilde{M_f},$ which in SI units corresponds to $\sim[-5.1,5.1]\, \mathrm{ms})$.
We use a uniform prior on $t_0$ so that the samples can be easily reweighted in post-processing (see Section~\ref{subsec:reweighting}). 
For the remaining parameters, we used a uniform prior over the sphere of the sky (parameterised using $\alpha$ and $\delta$) for the source location and a flat, periodic prior on the polarisation angle $\psi$ in the range $0$ to $\pi$.

The nested sampling~\cite{Skilling:2006gxv} algorithm as implemented in \textsc{dynesty}~\cite{Speagle:2019ivv} was used to sample the posterior with 4000 live points and using the random walk sampling method with a walk length parameter of 2000.


\subsection{Reweighting} \label{subsec:reweighting}

An ever present issue in ringdown analyses is the choice of ringdown start time, $t_0$, and this choice is closely related to the issue of the presence of an overtone.
To address this issue, previous time-domain analyses~\cite{Isi:2019aib, Cotesta:2022pci, Isi:2022mhy} perform large numbers of Bayesian analysis runs with different choices of start time.

One key conceptual benefit of the frequency-domain approach of Chapter~\ref{Chapter3} is that the ringdown start time enters as a parameter of the model and can therefore be easily marginalised over, instead of simply being fixed (although, see Ref.~\cite{Carullo:2019flw} where the ringdown start time was varied in a time-domain analysis). 
However, it is necessary to choose an informative (narrow) prior for the parameter $t_0$.

A related computational benefit of our approach is that we can do a single Bayesian analysis run with a broad uniform prior on $t_0$. 
We can then explore different, narrower priors by reweighting the results in post processing. 
This is an example of importance sampling (see, for example, Ref.~\cite{RobertChristian2013MCsm}) and is the approach adopted here.
This removes the need to perform the large number of runs used to explore the effect of varying the ringdown start time when performing time-domain ringdown analyses.

Given a model that depends on parameters $\params$, a likelihood $\mathcal{L}(\mathrm{data}|\params)$, and a prior $\pi(\params)$, nested sampling can be used to draw a large number of samples $\params_i$ from the posterior, which is given by Bayes' theorem $P(\params|\mathrm{data})\propto \mathcal{L}(\mathrm{data}|\params) \pi(\params)$.
Samples from the posterior have associated weights $w_i$ (samples may often be equally weighted with $w_i=1$, but we do not require this to be the case). 
Such samples can be used to approximate integrals via a Monte-Carlo sum; $\int \mathrm{d}\params\,P(\params|\mathrm{data})f(\params)=\sum_{i}w_i f(\params_i)/W$, where $W=\sum_{i}w_i$.
If we choose a new prior $\hat{\pi}(\params)$, then the Bayesian posterior is given instead by $\hat{P}(\params|\mathrm{data})\propto \mathcal{L}(\mathrm{data}|\params) \hat{\pi}(\params)$.
We can define the new weights via
\begin{align}
	\hat{w}_i = w_i \frac{\hat{\pi}(\params_i)}{\pi(\params_i)}.
\end{align}
The same samples can then be used to approximate integrals of the form $\int \mathrm{d}\params\,\hat{P}(\params|\mathrm{data})f(\params)$ via the Monte-Carlo sum $\sum_{i}\hat{w}_i f(\params_i)/\hat{W}$, where $\hat{W}=\sum_{i}\hat{w}_i$.

It is also possible to reweight the Bayesian evidence for the new choice of prior.
In a GW context this approach has been used previously for inference with higher-order modes~\cite{Payne:2019wmy}.
The Bayesian evidence (i.e.\ the normalisation denominator in Bayes' theorem) under the original prior is given by $Z=P(\mathrm{data})=\int\mathrm{d}\params\,\mathcal{L}(\mathrm{data}|\params)\pi(\params)$.
The Bayesian evidence under the new prior, $\hat{\pi}(\params)$, is $\hat{Z}=\int\mathrm{d}\params\,\mathcal{L}(\mathrm{data}|\params)\hat{\pi}(\params)$. Using the reweighted samples to approximate the integral, it can be shown that the new evidence is given by
\begin{align}\label{eq:new_evidence}
	\hat{Z} = Z\frac{\hat{W}}{W}.
\end{align}

\begin{figure}[t]
    \centering
    \includegraphics[width=0.6\columnwidth]{SearchingforaRingdownOvertoneinGW150914/start_time_plot.pdf}
    \caption[Different prior choices for the ringdown start time used for the GW150914 analysis]{ 
    Our ringdown inference is run initially using a flat, uniform prior on the ringdown start time, $t_0$, over the plot range $\pm 15 \tilde{M_f}$ relative to $t_\mathrm{ref}$ (Hanford frame).
    In post processing, the posterior samples can be reweighted to a different choice of prior on $t_0$ (see Section~\ref{subsec:reweighting}). 
    The different prior choices used in this chapter are shown in this figure. 
    We use a sequence of narrow Gaussian priors (with different means $\bar{t_0}$ defined relative to $t_\mathrm{ref}$ and fixed standard deviation, $\sigma=1\tilde{M_f}$) as well as using the posterior on the time of peak strain from a full IMR analysis as a prior.
    }
    \label{fig:start_time}
\end{figure}

The process of reweighting to the new, target prior reduces the effective number of posterior samples available.
For this not to be a problem, we require the original prior to have significant support across the target prior.
Here, we reweight on just a single parameter, the ringdown start time $t_0$.
As described above, we use a uniform prior on $t_0$ as the original prior, $\pi$, in our analyses.
For the target prior we use a variety of different choices, this removes the need for performing a large number of runs with different start times. 
Our prior choices are plotted in Fig.~\ref{fig:start_time}.
Narrow Gaussians centred at different start times are used to explore the start time dependence on the results, and we use the notation $\bar{t_0}$ to indicate the mean of the Gaussian relative to $t_\mathrm{ref}$. 
For more details on the $t_0$ reweighting, see Section~\ref{app:t0_posterior_prior}.

We also use the posterior on $t_\mathrm{peak}$ from a full inspiral-merger-ringdown (IMR) analysis from Ref.~\cite{Isi:2022mhy}, obtained with the \textsc{IMRPhenomPv2} (IMRP) waveform model~\cite{Hannam:2013oca}, as another prior on $t_0$. 
Our aim in doing this is to marginalise over our uncertainty on the ringdown start time, $t_0$. 
We emphasise that this is achieved here by using the posterior on the time of peak strain as a prior on $t_0$; this is motivated by the observations of Refs.~\cite{Giesler:2019uxc, Bhagwat:2019dtm, Ota:2019bzl, Cook:2020otn, JimenezForteza:2020cve, Dhani:2020nik, Finch:2021iip, Forteza:2021wfq, Dhani:2021vac, MaganaZertuche:2021syq} described above, which show that generically the ringdown can be considered to start at around this time.


\section{Results}\label{sec:results}

There are several ways to investigate and quantify the evidence for additional QNMs in the ringdown.
Section~\ref{subsec:overtone} contains the results of a series of analyses designed to study the presence of a possible overtone in GW150914.
Section~\ref{subsec:verify} contains the results of a series of analyses designed to test whether or not what has been detected really is an overtone and is not the accumulation of other effects.
Section~\ref{subsec:noise} describes further checks on the stability of the results, and Section~\ref{subsec:other_results} contains some additional results that further demonstrate the capabilities of the frequency-domain approach to ringdown analysis.

Throughout Secs.~\ref{subsec:overtone} and \ref{subsec:verify}, we compare our results with those in Refs.~\cite{Cotesta:2022pci} and~\cite{Isi:2022mhy}. 
This is done in the hope of helping to resolve the controversy over the evidence for a ringdown overtone in GW150914. 
However, it should be stressed that our results are produced using a very different method and care should therefore be taken in making direct comparisons.
Although the frequency-domain analysis is formally equivalent to the time-domain analysis in a particular limit (as discussed in the introduction, and in more detail in Section~\ref{subsec:motivation}) we do not take this limit in a practical analysis. Furthermore, the frequency-domain analysis is further generalised with respect to the time-domain analysis in that it marginalises over parameters such as the sky position and ringdown start time (which are fixed in the analyses of Refs.~\cite{Cotesta:2022pci, Isi:2022mhy}).
Results from our frequency-domain analyses should therefore not be expected to agree perfectly with those from previous time-domain analyses.

\subsection{Presence of an overtone}\label{subsec:overtone}

In order to investigate the presence of an overtone in the GW150914 ringdown, we initially perform two analyses using the model described in Section~\ref{sec:model}: one analysis uses only the fundamental QNM ($N=0$) and the other includes the first overtone ($N=1$).
Aside from the inclusion of the overtone in the ringdown (which introduces two additional parameters: an amplitude and a phase), these two analyses are otherwise identical.

\begin{figure}[t]
    \captionsetup[subfigure]{labelformat=empty}
    \centering
    \;\subfloat{\includegraphics[width=.49\linewidth]{SearchingforaRingdownOvertoneinGW150914/220_mass_spin_plot.pdf}}
    \;\subfloat{\includegraphics[width=.49\linewidth]{SearchingforaRingdownOvertoneinGW150914/220221_mass_spin_plot.pdf}}
    \caption[Posterior distributions on GW150914's remnant mass and dimensionless spin for different choices of $t_0$ prior]{  
    Posterior distributions on the remnant mass, $M_f$, and dimensionless spin, $\chi_f$, for different choices of $t_0$ prior (the colours and line styles correspond to those used in Fig.~\ref{fig:start_time}). 
    \emph{Left:} the results from the $(2,2,0)$ fundamental-mode-only analysis (i.e.\ $N=0$).
    \emph{Right:} the results from the overtone analysis including the $(2,2,0)$ and $(2,2,1)$ modes (i.e.\ $N=1$).
    Each line corresponds to a different choice of $t_0$ prior. 
    Coloured lines correspond to Gaussians with widths of $1 \tilde{M_f}$ and means $\bar{t_0}$ (see Fig.~\ref{fig:start_time}).
    The dashed black line corresponds to using the posterior on time of peak strain (from a full IMR analysis) as our prior, which marginalises over uncertainty on the time of peak strain.
    Also shown for reference (dotted line) is the posterior from a full IMR analysis. 
    The main panel shows the 90\% confidence contours while the side panels show the one-dimensional marginalised posteriors.
    }
    \label{fig:mass_spin_post}
\end{figure}

In Fig.~\ref{fig:mass_spin_post} we plot the posterior distributions on the remnant BH mass, $M_f$, and dimensionless spin, $\chi_f$, for both of these analyses.
Results are shown for the different choices of the prior on the ringdown start time shown in Fig.~\ref{fig:start_time} (these results were obtained by reweighting the samples obtained with a flat prior using the approach described in Section~\ref{subsec:reweighting}).
The earliest start time ($\bar{t_0}=-2\tilde{M_f}$) is omitted from the fundamental-only ($N=0$) plot in the left-hand panel of Fig.~\ref{fig:mass_spin_post} because of a low number of posterior samples at these time (see Section~\ref{app:t0_posterior_prior}).
Also shown for comparison are the much tighter constraints resulting from the full IMR analysis.
These IMR posterior samples were obtained from Ref.~\cite{maximiliano_isi_2022_5965773}, which (as detailed in Refs.~\cite{Isi:2019aib,Isi:2022mhy}) are obtained from applying fitting formulas to the samples available at Ref.~\cite{gwtc1datarelease}. 
When only the fundamental QNM is used ($N=0$), and when the analysis is started at early times (e.g.\ $t_0 - t_\mathrm{ref}\lesssim -2\tilde{M_f}$) our posteriors on the remnant parameters are biased towards high values of $M_f$ and $\chi_f$.
This behaviour is expected; a single QNM is only able to model the ringdown signal starting well after the time of peak strain.
Including an overtone ($N=1$) allows the ringdown analysis to start at earlier times, as can be seen by the removal of the bias in the right panel. 
This improvement is suggestive that the data supports the inclusion of an overtone.
Additionally, using an earlier ringdown start time increases the SNR in the ringdown and reduces the posterior width; this effect can be seen in both the $N=0$ and $N=1$ analyses.

\begin{figure*}[t!]
    \centering
    \includegraphics[width=0.9\columnwidth]{SearchingforaRingdownOvertoneinGW150914/overtone_amplitude_plot.pdf}
    \caption[Posteriors on the GW150914 overtone amplitude and Bayes' factors in favour of the overtone model for different choices of ringdown start time prior]{ 
    Posteriors on the overtone amplitude, and Bayes' factors in favour of the overtone model for different choices of $t_0$ prior (the colours and line styles correspond to those used in Fig.~\ref{fig:start_time}).
    \emph{Top:} posterior on the time of peak strain in the Hanford frame, from a \textsc{IMRPhenomPv2} analysis, as in Fig.~\ref{fig:start_time} (and originally from Ref.~\cite{Isi:2022mhy}). 
    \emph{Middle:} overtone amplitude posteriors for different choices of $t_0$ prior. The left panel corresponds to Gaussian priors with standard deviation $1\tilde{M_f}$, centred at the time they are plotted.
    The dotted line indicates the expected exponential decay of the $A_1$ mode; this is included merely to guide the eye and was produced using the median mass and spin values from the full IMR analysis and the median value of $A_1$ from the $\bar{t_0} = 0$ prior.
    The right panel corresponds to using the \textsc{IMRPhenomPv2} time of peak strain as a prior.
    For earlier start times the posteriors on the amplitude are peaked further away from zero; this is quantified in the inset plot where the ratio of the median to the standard deviation of the $A_1$ posterior is plotted.
    \emph{Bottom:} the Bayes' factor in favour of the overtone model for each prior choice; circles with error bars show the Bayes' factor calculated from nested sampling (with errors estimated by the sampler) while the crosses show the results calculated using the Savage-Dickey density ratio.    
    }
    \label{fig:overtone_amplitude}
\end{figure*}

Our results in Fig.~\ref{fig:mass_spin_post} can be compared to the corresponding results of the time-domain analyses shown in Fig.~1 from Cotesta et al.~\cite{Cotesta:2022pci} and Figs.~4 and 5 from Isi \& Farr~\cite{Isi:2022mhy}.
In general terms, there is broad agreement between all three sets of results. 
In particular, all three sets of authors find that the overtone analyses ($N=1$) always gives results that are more consistent with the IMR result and get increasingly broader for later choices of the ringdown start time.
All sets of authors also find that for the fundamental-only analysis ($N=0$) starting at early times (i.e.\ $t_0  - t_\mathrm{ref}\lesssim 0$) leads to posteriors that are inconsistent with the IMR result.
However, there are subtle differences between the various results.
Our results with $N=0$ and early start times gives posteriors biased to large values of $M_f$ and $\chi_f$; this is also seen in Ref.~\cite{Isi:2022mhy}, but not in Ref.~\cite{Cotesta:2022pci} (where the posterior consistently reaches lower values of $\chi_f$).
Our results with $N=0$ and late start times (i.e.\ $t_0 - t_\mathrm{ref}\gtrsim 4\tilde{M_f}$) are partially consistent with the IMR results; this is also seen in Ref.~\cite{Cotesta:2022pci}, but not in Ref.~\cite{Isi:2022mhy} who never find consistency with the IMR result for any choice of start time.
Finally, when including the overtone ($N=1$) and starting at late times, Ref.~\cite{Cotesta:2022pci} find results that are consistent with $\chi_f=0$ (i.e.\ a Schwarzschild BH) at 90\% confidence, in stark disagreement with Ref.~\cite{Isi:2022mhy} who find $\chi_f\gtrsim 0.2$. 
Our results are in better agreement with those of Ref.~\cite{Isi:2022mhy}.

In the middle panel of Fig.~\ref{fig:overtone_amplitude} we investigate our $N=1$ overtone analysis further by plotting the one-dimensional marginalised posteriors for the amplitude, $A_1$, of the QNM overtone.
An amplitude posterior peaked away from zero has been suggested (particularly by Ref.~\cite{Isi:2019aib}) as one good indication for the presence of an overtone in the data.
As expected, the QNM overtone decays quickly and when starting at later times we find a small value for the amplitude.
The degree to which the $A_1$ posterior is peaked away from zero can be quantified using the ratio between the median and standard deviation; this is plotted in the inset of the middle panel of Fig.~\ref{fig:overtone_amplitude}.
For values of $\bar{t_0}$ between $-2\tilde{M_f}$ and $+6\tilde{M_f}$, we find posteriors on $A_1$ that are peaked away from zero at between $1.44$ and $3.34\sigma$.
If we reweight using the IMRP $t_{\rm peak}$ prior, we find a posterior peaked away from zero at $1.79\sigma$.

Our results in the middle panel of Fig.~\ref{fig:overtone_amplitude} can be compared to the corresponding results of the time-domain analyses shown in Fig.~1 of Ref.~\cite{Isi:2022mhy} and Fig.~2 of Ref.~\cite{Cotesta:2022pci}.
All three sets of authors find values of $A_1$ that are smaller at later times, consistent with the expected exponential decay of the overtone, but they disagree on the absolute value of the amplitude and the significance with which a zero amplitude can be excluded.
Refs.~\cite{Isi:2019aib, Isi:2022mhy} find the largest values; they report a posterior peaked $3.6\sigma$ away from zero.
Ref.~\cite{Cotesta:2022pci} finds much smaller values which are consistent with zero for many choices of start time.
These analyses use essentially the same method and should therefore agree exactly.
Our result, produced using a different method, lies somewhere in between; we do find nonzero values are preferred for a range of start times, but only with a modest significance of $\sim 1.79\sigma$ for our preferred IMRP $t_{\rm peak}$ prior which we consider to be the best description of our uncertainty on the ringdown start time.

The comparison of our results with those of Refs.~\cite{Isi:2019aib, Cotesta:2022pci, Isi:2022mhy} is complicated by the fact that we use subtly different definitions for the amplitude. 
The time-domain analyses naturally define the mode amplitudes at a fixed time, usually $t_0$.
Our frequency-domain analysis also defines the mode amplitudes at $t_0$, but this start time is then varied as part of the analysis, blurring the exact time at which the amplitude is defined.
This is a fairly small effect for the narrow Gaussian priors, but more significant for the wider IMRP $t_{\rm peak}$ prior.
We can correct for this effect by rescaling all the overtone amplitudes to any fixed reference time (here we use $t_{\rm ref}$) using the known decay rate for the QNMs;
\begin{align}
	A_{1,\mathrm{ref}} = A_1 \exp\left(\frac{t_0-t_{\mathrm{ref}}}{\tau_{221}(M_f,\chi_f)}\right),
\end{align}
where $\tau_{221}(M_f, \chi_f)$ is the exponential decay time of the $(2,2,1)$ QNM and is a function of the remnant mass and spin.
This rescaling can be done for any QNM and the resulting amplitude parameters $A_{\ell m n,\mathrm{ref}}$ are more directly comparable with the amplitudes used in time-domain analyses.
Posteriors on $A_{1,\mathrm{ref}}$ are shown in  
Fig.~\ref{fig:amp_at_tref}.

\begin{figure}[t]
    \centering
    \includegraphics[width=0.6\columnwidth]{SearchingforaRingdownOvertoneinGW150914/overtone_amplitude_at_tref.pdf}
    \caption[Posteriors on the GW150914 overtone amplitude, rescaled to a fixed reference time]{ 
    Posteriors on the overtone amplitude from our $N=1$ overtone analysis, rescaled to a fixed reference time of $t_{\rm ref}$.
    The rescaling does not significantly affect the significance with which the posteriors are peaked away from zero.
    The colours and line styles indicate the prior used on $t_0$ and correspond to those used in Fig.~\ref{fig:start_time}.
    }
    \label{fig:amp_at_tref}
\end{figure}

In the bottom panel of Fig.~\ref{fig:overtone_amplitude} we plot the Bayes' factors between the fundamental only ($N=0$) and overtone ($N=1$) analyses.
This is defined as $\mathcal{B}_{\rm 1QNM}^{\rm 2QNM}=Z_{N=1}/Z_{N=0}$.
The Bayes' factor has been suggested (particularly by Ref.~\cite{Cotesta:2022pci}) as another good way for quantifying the support for an overtone in the data.
The Bayes' factor was computed in two different ways.
Firstly, \textsc{dynesty} was used to calculate the evidences $Z_{N=0}$ and $Z_{N=1}$ for both of the analyses described above, and these were reweighted to the desired $t_0$ prior using Eq.~\ref{eq:new_evidence}. 
Nested sampling also returns an estimate for the error on the evidences, and these are used to plot the error bars in Fig.~\ref{fig:overtone_amplitude}.
Secondly, exploiting the fact that the $N=0$ model is nested within the $N=1$ model, the Bayes' factors were computed using the posterior on $A_1$ from the $N=1$ analysis to find the Savage-Dickey density ratio~\cite{10.2307/2958475}. 

Our results in the bottom panel of Fig.~\ref{fig:overtone_amplitude} can be compared to the corresponding results of the time-domain analyses shown in Fig.~7 of Ref.~\cite{Isi:2022mhy} and Fig.~2 of Ref.~\cite{Cotesta:2022pci}.
Ref.~\cite{Cotesta:2022pci} computes the Bayes' factors using the ratio of evidences evaluated with nested sampling, whereas Ref.~\cite{Isi:2022mhy} computes Bayes' factors using Savage-Dickey density ratios.
All sets of authors find Bayes' factors that decrease for later ringdown start times, although they disagree on the exact value.
Ref.~\cite{Isi:2022mhy} finds the strongest log-evidence of $\sim 1.7$ at $t_0-t_{\rm ref} \sim 0$.
Ref.~\cite{Cotesta:2022pci} finds slightly negative log-evidence starting at this time.
Again, our result lies somewhere in between, we find a moderate log-evidence of $\sim 1.0$ when marginalising over a narrow prior on $t_0$ centred at this time.
If we instead marginalise over the time of peak strain using the broader IMRP $t_{\rm peak}$ prior, the evidence is slightly negative.
However, as discussed in Section~\ref{ch4:sec:discussion} below, we consider the actual values of the Bayes factors to be less important than their trend with varying start time.


\subsection{The nature of the overtone}\label{subsec:verify}

The results of the previous section show that there is tentative evidence for something beyond the fundamental $(2,2,0)$ mode in the GW150914 data. 
In the previous section it was assumed that this is the $(2,2,1)$ QNM overtone; this is motivated by our expectations from numerical relativity experiments (see, for example, Ref.~\cite{Giesler:2019uxc}). 
In this section, we address this assumption by measuring the frequency and amplitude of the QNM overtone and comparing with the expectations from GR.

Fig.~\ref{fig:delta_f} shows the results of a third ringdown analysis that also includes two QNMs.
In this analysis the complex frequency of the second QNM is allowed to deviate from the Kerr overtone value. 
This differs from the $N=1$ overtone analysis described above, where the frequency of the overtone was fixed by the remnant mass and spin to the Kerr value, $\omega_{221} = 2\pi f_{221} - i/\tau_{221}$.
Recovering a value of $\delta f$ consistent with zero has been suggested (particularly by Ref.~\cite{Isi:2022mhy}) as further evidence for the presence of an overtone; otherwise, it might be expected that the extra parameters would fit to the noise and would not recover the Kerr value.
We use the parameterisation from Ref.~\cite{Isi:2022mhy}; the complex frequency of the second QNM is now $\omega_{221} = 2\pi f-i/\tau$, where $f=f_{221}\exp(\delta f)$ and $\tau=\tau_{221}\exp(\delta \tau)$. 
This introduces the two new dimensionless parameters $\delta f$ and $\delta \tau$ into the model, for which we use uniform priors in the range $[-0.5,\, 0.5]$.
The $\delta \tau$ parameter is not well constrained, therefore we focus initially on $\delta f$.
We find posteriors on $\delta f$ consistent with zero for all choices of $t_0$ prior with standard deviations $\sim 0.2$. 
This is consistent with what was found in Ref.~\cite{Isi:2019aib} and can be viewed as a test of the no-hair theorem at the $\sim 20\%$ level.

\begin{figure}[t]
    \centering
    \includegraphics[width=0.6\columnwidth]{SearchingforaRingdownOvertoneinGW150914/220221_deviation_plot.pdf}
    \caption[Posteriors on the deviation from the Kerr for the real part of the GW150914 overtone frequency]{ 
    Posteriors on the deviation parameter from the Kerr value for the real part of the overtone frequency.
    The colours and line styles distinguish different choices for the $t_0$ prior and correspond to those used in Fig.~\ref{fig:start_time}.
    The mode frequency is given by $f_{221}^{\rm Kerr} \exp(\delta f)$, so that $\delta_f=0$ is the expected result for the Kerr metric.
    For all choices of $t_0$ prior the data is consistent with $\delta f=0$.
    }
    \label{fig:delta_f}
\end{figure}

Our results in Fig.~\ref{fig:delta_f} can be compared with Fig.~2 of Ref.~\cite{Isi:2022mhy}. 
Our preferred run, using the IMRP $t_{\rm peak}$ prior on $t_0$, is broadly consistent with that result.
However, what is notable about our results is that we do not find a significant broadening of the posterior for later choices of the start time. 
This was found by Ref.~\cite{Isi:2022mhy} and would be expected if an overtone was present, as both the overtone amplitude and ringdown SNR decaying with later ringdown start times.

\begin{figure}
    \centering
    \includegraphics[width=0.6\columnwidth]{SearchingforaRingdownOvertoneinGW150914/kerr_spectrum_and_deviation.pdf}
    \caption[Posterior on the dimensionless complex frequency of the second GW150914 QNM assuming the first is the fundamental mode]{ 
    The posterior on the dimensionless complex frequency of the second QNM (50\% and 90\% regions), assuming the first is the fundamental $(\ell,|m|,n)=(2,2,0)$ mode.
    Lines indicate the Kerr frequencies parameterised by the remnant spin; dots and crosses indicate points with $\chi_f=0.7$ and $0$ respectively.
    Lines are coloured according to their $\ell$ and $n$ indices and the $m$ index increases left to right in each set.
    The frequency of the second QNM is consistent with the expected $(2,2,1)$ overtone, but also with several other modes.
    However, all fundamental modes (those with $n=0$) are excluded.
    }
    \label{fig:other_QNMs}
\end{figure}

\begin{figure}[t]
    \centering
    \includegraphics[width=0.6\columnwidth]{SearchingforaRingdownOvertoneinGW150914/amplitude_ratio.pdf}
    \caption[Posteriors on the amplitude ratio $A_1/A_0$ from the GW150914 overtone analysis]{ 
    Posteriors on the amplitude ratio $A_1/A_0$ from our $N=1$ overtone analysis. 
    The 90\% contours are plotted, with the colours and line styles indicating the $t_0$ prior and correspond to those used in Fig.~\ref{fig:start_time}.
    The solid grey curve shows the results of a two-QNM fit to the numerical relativity simulation SXS:BBH:0305 which has parameters consistent with GW150914.
    The dashed grey curve shows the results of a multi-QNM fit to SXS:BBH:0305 which follows closely the expected exponential decay rate for the amplitude ratio.
    }
    \label{fig:amp_ratio}
\end{figure}

To investigate this further, we use the results of the ringdown analysis where the frequency of the overtone is allowed to vary freely to address another important question. 
If the data does indeed contain a second QNM, can we determine which mode it is?
Theoretical studies of numerical relativity simulations suggest that the $(\ell,|m|,n)=(2,2,1)$ will be the next most prominent, especially for early start times~\cite{Giesler:2019uxc}. 
In Fig.~\ref{fig:other_QNMs} we plot the posterior on the dimensionless complex frequency (allowing both the real and imaginary parts to vary freely) of the second QNM, $\omega M_f$.
This plot uses the value for $M_f$ calculated from the complex frequency of the first QNM, assuming this is the expected $(2,2,0)$ fundamental mode of Kerr.
We find that we can confidently conclude that the second mode is an overtone ($n\geq 1$) but that it is not possible to say from the data alone exactly which overtone. 
For example, the modes $(2,2,1)$ and $(2,1,1)$ are both equally compatible with the data. 
In general, when searching for additional QNMs it is necessary to be guided by our prior expectations regarding which modes are expected to be excited with the highest amplitudes.

We now turn our attention to the measured amplitude $A_1$ and whether this matches the theoretical expectations for the $(2,2,1)$ overtone. 
For convenience, we choose to work with the amplitude ratio $A_1/A_0$ which eliminates factors common to all modes, such as the distance to the source. 
Because the two QNMs decay exponentially at different rates, the amplitude ratio depends strongly on the chosen ringdown start time.
Our two-dimensional posteriors on the amplitude ratio and ringdown start time are plotted in Fig.~\ref{fig:amp_ratio}.
As expected we find that the amplitude ratio decreases for later start times, and the error on the amplitude ratio increases for later start times because of the decreasing SNR in the ringdown.

In order to check whether this is consistent with the theoretical expectation for an overtone we compare with fits to the numerical relativity simulation SXS:BBH:0305~\cite{Lovelace:2016uwp} which has parameters consistent with GW150914.
Fixing the remnant mass and spin to the values reported in the simulation metadata, we perform QNM least-squares fits to this simulation for a range of ringdown start times using the code previously developed in Ref.~\cite{Finch:2021iip}.
Results are shown in Fig.~\ref{fig:amp_ratio} for two such fits. 
Firstly, we performed a two-QNM fit intended to mimic the analysis of the real GW150914 data described in Section~\ref{subsec:overtone} above. 
In this analysis the ${}_{-2}Y_{22}$ spherical harmonic mode of the simulation is modelled as a sum of the $(2,2,0)$ and $(2,2,1)$ QNMs and the amplitude ratio is recorded. 
The results from this two-QNM fit agree very well with what is seen in the real data giving us further confidence that there is nothing unexpected present in the data and that our results are not unduly affected by noise fluctuations (see also the discussion in Section~\ref{subsec:noise}). 
Secondly, we perform a full multi-QNM fit to all the spherical harmonic modes (up to and including $\ell = 8$) with a ringdown model that includes all QNMs (including both prograde and retrograde modes) up to $\ell = 8$ and $n = 7$ (1232 QNMs in total).
The ratio of the amplitudes of the $(2,2,0)$ and $(2,2,1)$ prograde modes from this fit behaves very differently; the ratio follows very closely a exponential time evolution which can be understood in terms of the difference between the two QNM decay times.

The fact that the two-QNM analysis gives a very different amplitude ratio compared to the full multi-QNM analysis for ringdown start times near the peak strain is related to the extreme destructive interference observed in the QNM overtone fits of Refs.~\cite{Giesler:2019uxc, Bhagwat:2019dtm, Ota:2019bzl, Cook:2020otn, JimenezForteza:2020cve, Dhani:2020nik, Finch:2021iip, Forteza:2021wfq, Dhani:2021vac, MaganaZertuche:2021syq} with large values of $N$.
This shows that the amplitude $A_1$ recovered from a two-QNM analysis is not purely the amplitude of the first overtone but also includes significant contributions from higher overtones and other harmonics. 
However, absorbing these contributions into the first overtone introduces a systematic bias in the remnant properties that is smaller than the statistical uncertainty; this can be seen in, for example, Fig.~\ref{fig:mass_spin_post} and Section IV\,C of Ref.~\cite{Giesler:2019uxc}. 
For this reason, it still makes sense to describe the results of the two-QNM analysis as a measurement of the overtone, even though there are undoubtedly other contributions present in the signal.


\subsection{The effect of noise and sampling frequency}\label{subsec:noise}

One of the key claims made in Ref.~\cite{Cotesta:2022pci} was the overtone detection was highly sensitive to noise fluctuations.
This was disputed by Ref.~\cite{Isi:2022mhy}.
In order to address this issue, we performed a noise injection study mirroring closely what was done in Ref.~\cite{Cotesta:2022pci}.
The results of this injection study are presented in Section~\ref{app:inj}.
As expected, the results of injecting into different noise realisations show some scatter.
However, this scatter is not larger than expected and we are unable to reproduce the claim in Ref.~\cite{Cotesta:2022pci} with our (very different) analysis method. 

It has been suggested~\cite{WillMaxTGRtelecon} that the results of ringdown analyses, in particular those including overtones, might be sensitive to aliasing effects when using downsampled strain data due to the reduced Nyquist frequency.
QNM overtones, $(\ell, m, n \geq 1)$, have roughly the same real part of the frequency as the corresponding fundamental, $(\ell, m, 0)$, mode, but they have a shorter damping time. 
See, for example, Fig.~\ref{fig:other_QNMs}. 
This means that if a single, isolated, mode is viewed in Fourier space, the power spectrum is broader and contains significant power at higher frequencies.

Early ringdown studies, including those in Refs.~\cite{LIGOScientific:2016lio, Isi:2019aib, Isi:2022mhy}, generally used strain data that had been downsampled to $2\,$kHz. 
This was done for convenience and computational speed and was not originally anticipated to be a problem because the merger of GW150914 occurs at $\sim 200\,\mathrm{Hz}$, safely below the Nyquist frequency.

In this chapter the $4\,$kHz data is used for all the analyses in the main text. 
Additionally, a frequency-domain log-likelihood with an upper integration limit of $f_{\rm high}=1000\,$Hz was used.
Our method is very different from the time-domain analyses, and do not expect our results to be sensitive to small changes in these choices.
To check that this is the case we have repeated the $N=1$ overtone analysis using the $16\,$kHz sampled data (obtained from Ref.~\cite{gwosc}) and we find no significant changes in our results.
Using reweighting techniques (this time applied to the likelihood) we have also investigated the effect of changing the upper limit of integration in the likelihood. 
By re-evaluating the $N=1$ posterior chain on a likelihood with $f_\mathrm{high}=1500\,$Hz and $2000\,$Hz, and then reweighting, we again find no significant changes in our results.


\subsection{Posteriors on the ringdown start time}\label{app:t0_posterior_prior}

As described in Section~\ref{subsec:reweighting}, we initially perform Bayesian inference on the ringdown using a broad, flat prior on the ringdown start time parameter $t_0$. 
The posteriors on $t_0$ from both the $N=0$ and $N=1$ analyses are shown in Fig.~\ref{fig:t0_posterior}.
We do not consider these posteriors to be physically meaningful results because they were obtained with a prior that does not correctly describe our state of knowledge about when the ringdown should start.
These results are produced merely as an intermediate step in our analysis, before the reweighting was applied, and are shown here only to further illustrate the reweighting procedure described in Section~\ref{subsec:reweighting}.

\begin{figure}[b!]
    \centering
    \includegraphics[width=0.6\columnwidth]{SearchingforaRingdownOvertoneinGW150914/unweighted_t0_posteriors.pdf}
    \caption[Posteriors on the GW150914 ringdown start time]{ 
    Posteriors on the ringdown start time obtained from our initial analysis using a flat prior over the range shown in the plot.
    Results are shown for the fundamental only ($N=0$) and overtone ($N=1$) analyses.
    Vertical coloured lines show the locations of the means $\bar{t_0}$ of the narrow Gaussian priors used for the subsequent reweighting (see Fig.~\ref{fig:start_time}).
    The $N=1$ posterior has ample support across the entire range of interest, as required for the reweighting to remain accurate.
    The $N=0$ posterior has enough support everywhere except the $\bar{t_0}=-2\ \tilde{M_f}$ prior.
    }
    \label{fig:t0_posterior}
\end{figure}

In order for the subsequent reweighting step to be accurate, it is necessary for the posterior chains (particularly for the $N=1$ overtone analysis) to contain samples across the range of start times that we consider. 
For this reason, the \textsc{dynesty} sampler settings described in Section~\ref{sec:details} were chosen to ensure a large number of posterior samples were produced; we obtained 203697 and 218882 posterior samples from the $N=0$ and $N=1$ analyses respectively. 
This is sufficient for the reweighting to remain accurate everywhere except for the earliest start time in the $N=0$ analysis. 
This is the reason why this result is omitted from Fig.~\ref{fig:mass_spin_post}.


\subsection{Injection study}\label{app:inj}

Closely following the injection study performed in Ref.~\cite{Cotesta:2022pci}, we inject GW150914-like signals in the instrumental noise surrounding the true GW150914 event and reperform our overtone analysis ($N=1$).

The $\ell=2$ spin-weighted spherical harmonic of the numerical relativity simulation SXS:BBH:0305~\cite{Lovelace:2016uwp} was used as the mock signal, scaled to a total mass of $72\,M_\odot$ and injected with a face-off orientation at a luminosity distance of $410\,\mathrm{Mpc}$. 
The sky position was taken to be $\alpha = 1.95\,\mathrm{rad}$, $\delta=-1.27\,\mathrm{rad}$.
This signal was injected into the data surrounding GW150914, such that the peak of the absolute value of the strain occurred at times $[-20, -15, -10, 5, 15, 20, 25, 30, 35, 40]\,\mathrm{s}$ relative to $t_\mathrm{ref}$.
These choices ensure the mock signal does not overlap with the real event.
Additionally, a zero-noise injection was performed for comparison. 

\begin{figure}[t]
    \centering
    \includegraphics[width=0.9\columnwidth]{SearchingforaRingdownOvertoneinGW150914/injection_study_amps_only.pdf}
    \caption[Posteriors on the overtone amplitude from a noise injection study]{ 
    This is similar to the middle panel of Fig.~\ref{fig:overtone_amplitude} in the main text, but shows the posteriors on the overtone amplitude from the noise injection study.
    The different violin plots are for the different priors on the ringdown start time and the colours are the same as those used in Figs.~\ref{fig:start_time} and \ref{fig:overtone_amplitude}.
    On the right-hand side of each set of violin plots, the filled posterior shows the result obtained using the real GW150914 data (this is the same as what is plotted in Fig.~\ref{fig:overtone_amplitude}). 
    On the left-hand side are all the posteriors from the injection campaign, which indicate the spread in results due to different noise realisations.  
    Finally, the dashed lines on the right-hand side are the posteriors from the zero-noise injection.
    This plot is intended to be compared to Fig.~2 of Ref.~\cite{Cotesta:2022pci}, and Fig.~6 of Ref.~\cite{Isi:2022mhy}.
    }
    \label{fig:injection_study}
\end{figure}

We performed the frequency-domain ringdown analysis on these mock datasets using the same setup as was used for the real data and as described in Section~\ref{sec:details}.
This includes using the same PSD in the likelihood for all datasets.
We plot the resulting posteriors on the overtone amplitudes in Fig.~\ref{fig:injection_study}.
As with the real data, prior reweighting (see Section~\ref{subsec:reweighting}) has been used to show results for different choices of the ringdown start time prior.
We also investigated the Bayes' factors and found the same declining trend.

As expected, different noise realisations introduce some scatter into the results and we observe a spread in the locations of the maximum posterior values for the overtone amplitudes. 
However, this spread is consistent with the width of the posterior. 
The analysis described in the main text found only tentative evidence for the overtone, but there is no indication that this is overly effected by noise fluctuations.


\subsection{Wavelet posteriors}\label{app:W3}

The frequency-domain ringdown analysis method described in Chapter~\ref{Chapter3} and used here marginalises over the early-time inspiral-merger signal using a flexible combination of sine-Gaussian wavelets (see Eq.~\ref{eq:wavelets}).
In the GW150914 analyses presented in this chapter $W=3$ wavelets were used.
This choice was found empirically to be large enough to model the inspiral-merger signal without biasing the ringdown inference.
We have also verified that no strong correlations are observed between the wavelet and QNM parameters, and that further increasing the number of wavelets does not significantly affect the results for physically meaningful parameters (such as remnant parameters $M_f$ and $\chi_f$). 
These tests are described further in Section~\ref{app:GW150914}.

The whitened strain posterior on the sum of these wavelets, together with the QNMs, can be seen in the early-time signal in Fig.~\ref{fig:waveform} where the fit to the data is seen to be excellent.
The wavelet parameters themselves are not physical; the wavelets are being used here solely to marginalise out the inspiral-merger. 
Nevertheless, in this section we show some additional posterior plots on the wavelet parameters, see Fig.~\ref{fig:wavelet}.
As expected, in order to describe the ``chirping'' inspiral signal, the wavelets naturally order themselves with their amplitudes and frequencies increasing with time. 

\begin{figure}[t]
    \centering
    \includegraphics[width=0.9\columnwidth]{SearchingforaRingdownOvertoneinGW150914/wavelet_posterior_figure.pdf}
    \caption[Posteriors on selected wavelet parameters used in the GW150914 overtone analysis]{ 
    Posteriors on selected parameters for the $W=3$ wavelets used in the $N=1$ overtone analysis, reweighted using the IMRP $t_{\rm peak}$ prior on the ringdown start time.
    \emph{Left:} the wavelet central times, $\eta_w$.
    \emph{Middle:} the wavelet widths, $\tau_w$.
    \emph{Right:} the wavelet frequencies, $\nu_w$. 
    The index runs over values $w=1,\,2$ and $3$, where the numbering of the wavelets is chosen to enforce the ordering $\nu_{w}<\nu_{w+1}$.
    All plots use SI units on the upper $x$-axis and natural units on the lower $x$-axis. 
    }
    \label{fig:wavelet}
\end{figure}



\subsection{Other results}\label{subsec:other_results}

\begin{figure}[b!]
    \centering
    \includegraphics[width=0.9\columnwidth]{SearchingforaRingdownOvertoneinGW150914/skymap.pdf}
    \caption[Posterior on the GW150914 source sky position]{ 
    Posterior on the source sky position using geocentric coordinates in Mollweide projection.
    Shown in blue is the results from the $N=1$ overtone analysis using the IMRP $t_{\rm peak}$ reweighting for the prior on the ringdown start time.
    The LIGO skymap for this event is shown by the dashed black line for comparison.
    The inset plot shows a zoomed-in map plotted using right ascension and the sine of the declination.
    In both cases, 50\% and 90\% contours are plotted.
    }
    \label{fig:skymap}
\end{figure}

One important benefit of the frequency-domain approach to ringdown data analysis introduced in Chapter~\ref{Chapter3} and used here is that it naturally allows us to search (and hence to numerically marginalise) over source sky position and ringdown start time. 
This should be contrasted with the treatment of these parameters in most time-domain analyses where these parameters are fixed, potentially biasing the results. (Although it is technically possible to search over the sky in a time-domain analysis~\cite{Carullo:2019flw, Isi:2021iql}, this is rarely done in practice.)
To emphasise this, we plot the posterior on the sky location of GW150914 from our $N=1$ overtone analysis reweighted to the IMRP $t_{\rm peak}$ prior on the ringdown start time.
This can be compared with the publicly available LIGO sky posterior for GW150914 obtained using the samples from Ref.~\cite{skysamples}.
This is shown in Fig.~\ref{fig:skymap}.
As discussed in Chapter~\ref{Chapter3} (see the discussion around Fig.~\ref{fig:t0_geocent_posterior}), it should be emphasised that this sky posterior is not a ringdown-only result because much of the information is also coming from the wavelets used to model the inspiral-merger portion of the signal.

\begin{figure}[t]
    \centering
    \includegraphics[width=0.9\columnwidth]{SearchingforaRingdownOvertoneinGW150914/waveform_plot.pdf}
    \caption[Posterior on the GW150914 reconstructed whitened waveform]{ 
    Posterior on the reconstructed whitened waveform.
    Shown in grey is the strain data from both LIGO interferometers (\emph{top}: Hanford, \emph{bottom}: Livingston) whitened according to the noise amplitude spectral density in the detector and bandpass filtered between 32 and $512\,\mathrm{Hz}$ for clarity.
    Shown in blue is the waveform reconstruction from the $N=1$ overtone analysis with the IMRP $t_{\rm peak}$ reweighting for the prior on the ringdown start time.
    The blue lines and shaded regions indicate median and the 90\% credible interval. 
    The signal is plotted as a function of time from $t_{\rm ref}$ using both SI and natural units on the upper and lower $x$-axis respectively.
    }
    \label{fig:waveform}
\end{figure}

Because the inspiral and merger parts of the signal are being modelled using truncated wavelets as part of the frequency-domain ringdown analysis, this allows us to plot a full waveform reconstruction from our results.
This reconstruction is shown in Fig.~\ref{fig:waveform} for our $N=1$ overtone analysis reweighted to the IMRP $t_{\rm peak}$ prior.
The full waveform model used in our analysis is discontinuous at $t_0$. However, as discussed in Chapter~\ref{Chapter3}, the whitened waveform reconstruction plotted here is smooth; this is a result of marginalising over the location of the discontinuity at $t_0$, the waveform model ``learning'' the continuity from the data, and the whitening process used to make the figure.
This waveform reconstruction uses the posterior on all of the model parameters, including those for the wavelets; more details on these parameters are given in Section~\ref{app:W3}.


\section{Conclusions}\label{ch4:sec:discussion}

The main motivation for this work comes from the ongoing discussion in the literature about whether a ringdown overtone can be confidently detected in the GW150914 data. 
In particular, the detection claim made in Ref.~\cite{Isi:2019aib} was disputed by Ref.~\cite{Cotesta:2022pci} where a nearly identical time-domain analysis was reperformed (see also the reply Ref.~\cite{Isi:2022mhy}).
Applying the frequency-domain ringdown analysis originally presented in Chapter~\ref{Chapter3}, we contribute to this discussion with a thorough reanalysis of the GW150914 data. 
This includes performing analyses with and without an overtone while considering different ringdown start times, as well as performing a noise injection study and studying the effects of different data sampling rates and frequency integration limits on our results.
Although the method used here differs significantly from previous time-domain analyses, we present our results in a way that makes it as easy as possible to compare with earlier work.
In conclusion, we do find tentative evidence for a ringdown overtone, but not at the high level of significance originally claimed in Ref.~\cite{Isi:2019aib}.

In order to be more quantitative, it is first necessary to be able to say clearly what it even means to ``detect a overtone''. 
Although intuitively obvious, it is not clear how to make this notion precise (this issue has previously been discussed in Ref.~\cite{Isi:2022mhy}). 
Several approaches have been suggested: looking to see if including the overtone improves the posterior on the remnant parameters (see Fig.~\ref{fig:mass_spin_post}); looking at the posterior on the overtone amplitude for a range of start times (see middle panel of Fig.~\ref{fig:overtone_amplitude}); computing the Bayes' factor in favour of an overtone (see bottom panel of Fig.~\ref{fig:overtone_amplitude}); and allowing the frequency of the second QNM to vary freely to see if the data prefers, or at least is consistent with, the expected Kerr value (see Figs.~\ref{fig:delta_f} and \ref{fig:other_QNMs}).
Although these are not all independent from one another, they all help shed light on which QNMs are present. 
The results of all of these tests can also be compared to results from a noise injection study.

As well as not being completely independent of each other, none of these tests are, by themselves, sufficient to justify a claim of a detection.
For example, one issue that has been raised is that the Bayes' factor can be made to take any value with a suitable adjustment to the prior range.
There are also conceptual problems regarding what it means to compare two models, neither of which is expected to fully describe the data. Here we are comparing the fundamental-only mode model (with a single QNM) to the overtone model (with two QNMs) when our firm prior belief is that the true signal should contain an infinite number of QNMs plus additional corrections (e.g.\ from nonlinearities in the merger, tails, and memory effects).

From the above discussion, it is clear that ringdown analyses are rather subtle. 
We think our frequency-domain method has some important advantages over what has been done before. 
For example, it marginalises over the ringdown start time and sky position which is preferable to fixing these parameters (which potentially introduces systematic biases). 
Ideally, we should also marginalise over the uncertainties in the noise power spectral density (see, e.g.,\ Ref.~\cite{Cornish:2020dwh}) and detector calibration (see, e.g.,\ Ref.~\cite{LIGOScientific:2017aaj}) as part of a ringdown analysis. 
The ability to do this is, in principle, another benefit of the frequency-domain analysis approach used here as this can be done using techniques that are standard in the field.

We stress that while our results have been compared with those of previous time-domain studies, our frequency-domain method is rather different and therefore we do not expect to find perfect agreement. 
In contrast, the results of Refs.~\cite{Isi:2019aib, Cotesta:2022pci, Isi:2022mhy} are produced using essentially identical methods and should therefore be expected to agree exactly. 
The reason for the disagreement that is seen there is currently unknown and the subject of an ongoing investigation by both sets of authors.
It is vitally important for QNM science that all results are reproducible. To that end we have made all our data products and plotting scripts publicly available at Ref.~\cite{finch_eliot_2022_6949492}.

If QNMs are going to fulfill their promise for testing GR, fundamental physics and the Kerr metric hypothesis, then the community must be able to agree on standards for what it means to detect them and to be able to robustly quantify their significance. 
This field is still very young, and that there is already significant controversy regarding the QNM content of GW150914 and GW190521 is concerning, and we risk the situation becoming more confused with many more suitable events expected in O4.
And, as discussed in the introduction, this is a conceptual issue that will not be resolved with more observations, even at higher SNRs.
This issue needs input from the whole community; however, we suggest that (as a minimum) future claims of an overtone detection are accompanied by the investigations in Fig.~\ref{fig:mass_spin_post}, both panels of Fig.~\ref{fig:overtone_amplitude} and Fig.~\ref{fig:delta_f}.
That is, posteriors on the remnant properties with and without the overtone, posteriors on the overtone amplitude, a study of the Bayes' factor trends for different start times, and posteriors on deviations from Kerr when the overtone frequency is allowed to vary.

All data products and plot scripts associated with this work are made publicly available at Ref.~\cite{finch_eliot_2022_6949492}.
 

%----------------------------------------------------------------------------------------
%	THESIS CONTENT - CONCLUSIONS
%----------------------------------------------------------------------------------------

\chapter{Conclusions and Prospects}

In this thesis we have investigated the ringdown in two ways; fits of ringdown models to noiseless NR simulations inform us about what to look for in GW observations, and analysis of GW data tests our expectations.
For the former, we extended the work of Giesler et al.~\cite{Giesler:2019uxc} to precessing BBH mergers.
It was shown that overtones alone are not sufficient to model the ringdown at early times in this extended parameter space, and even the inclusion of mirror modes and mode mixing (although improving the model in many instances) fail in some cases.
Although beyond the limits of current detector sensitivities, these studies are valuable for understanding the nature of the ringdown, and also for testing our waveform models.
For example, we also showed that a current state-of-the-art surrogate waveform, NRSur7dq4, is not sufficiently accurate in the ringdown for studies of this sort.

Regarding the nature of the ringdown, there remain many unanswered questions in ringdown modelling. 
It is still unclear if ringdown overtones are actually physical, despite convincing evidence for their effectiveness in fitting ringdown waveforms and improving constraints on remnant properties.
Their tendency to destructively interfere with each other, resulting in inflated amplitudes, and their rapid decay naturally raises suspicions of over-fitting.
Where, exactly, the nonlinearities from the merger go (if the linear ringdown model can indeed be extended to early times) is also the subject of ongoing study.

In principle, the QNM amplitudes encode information about how the remnant BH was perturbed, i.e., the initial binary configuration.
Mappings between aligned-spin binary properties and QNM amplitudes have been found~\cite{London:2014cma}, but a similar relationship for precessing systems remains elusive.
This is closely related to the issue of ringdown start time, for which choosing a reliable value becomes complicated in the precessing case.
With the code~\cite{qnmfits} developed for the work in Chapter~\ref{Chapter2}, studies of QNM amplitudes is a natural target.
Since the publication of Ref.~\cite{Finch:2021iip} the code has been developed to incorporate multimode ringdown fitting, and also has the capability to do fits of nonlinear QNM frequencies.
A second-order effect, nonlinear QNMs can be thought of as QNMs sourced by their first-order counterparts.
These quadratic QNMs have now been identified in NR waveforms~\cite{Cheung:2022rbm, Mitman:2022qdl}, hinting at their importance in waveform modelling.

But, it is now clear that QNM studies are being limited by the accuracy of state-of-the-art NR waveforms. 
Typically, NR waveforms extrapolate the signal to large distances from the source.
However, this is known to be sub-optimal for the ringdown.
One solution is to work with CCE waveforms~\cite{MaganaZertuche:2021syq}, which allows for more control over subtle choices of the waveform frame.
This machinery provides opportunities for studying in detail topics such as nonlinear QNMs and GW memory, and is readily compatible with our code.

Turning to the analysis of real data, the ringdown presents some unique challenges. 
These include low SNR signals, uncertainties with model mode content and start time, and that it is more natural to work in the time domain (due to the ringdown's abrupt start) instead of the more commonly used frequency domain.
Differences in the choices made for the analyses have already led to disagreements in the literature regarding the QNM content present in the data; in particular, the presence of an overtone in GW150914, and the presence of a higher harmonic in GW190521. 
To resolve this uncertainty I have developed a frequency-domain ringdown analysis pipeline.
This approach is introduced in Chapter~\ref{Chapter3}, and the code is available at Ref.~\cite{fdringdown}.
By employing sine-Gaussian wavelets to model the pre-ringdown signal we can move back into the frequency domain and make use of well-established analysis methods. 
The method was then employed to quantify evidence for a GW150914 overtone in Chapter~\ref{Chapter4}, contributing to an ongoing discussion in the literature with a method which marginalises over multiple uncertainties to report a single significance which was not possible before.

There is also scope for further studies, in particular for GW190521; the method can be readily applied to other GW events with any choice of QNM content. 


%----------------------------------------------------------------------------------------
%	THESIS CONTENT - APPENDICES
%----------------------------------------------------------------------------------------

%\appendix % Cue to tell LaTeX that the following "chapters" are Appendices

% Include the appendices of the thesis as separate files from the Appendices folder
% Uncomment the lines as you write the Appendices

%\include{Appendices/AppendixA}
%\include{Appendices/AppendixB}
%\include{Appendices/AppendixC}

%----------------------------------------------------------------------------------------
%	BIBLIOGRAPHY
%----------------------------------------------------------------------------------------

\singlespacing
\printbibliography

%----------------------------------------------------------------------------------------

\end{document}  

